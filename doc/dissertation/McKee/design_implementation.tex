% design_implementation.tex
\newpage
\chapter{Design and Implementation} 
\section{Design}
This body of work is mainly interested in common classifiers, and in trying to use them in a novel way. It therefore makes sense to extend a package in which these are already implemented, and in which they are easily accessible. The Weka project therefore suits this purpose since it has well maintained, efficient implementations of most machine learning algorithms, as well as handling file I/O, and having both command line and graphical user interfaces. Since these are written in Java, it follows that the plugin should also be written in Java.

It also follows that any files which are used for testing should be in either .csv, or .arff format. The former is commonly used in data processing while the latter is a proprietary Weka format which is very similar, but differs in that it contains additional data at the top of the file pertaining to the dataset. Both are supported by Weka, and therefore either is usable.

Since the project relies upon the idea that predictions for missing training data values should converge, it is assumed that the developed classifier will initially only work on nominal data sets. This is because nominal data sets should eventually settle on concrete values, whilst numeric values are not guaranteed to every settle on a particular value at all. It is more likely that numeric predicitions would just change less and less between iterations, until the changes were miniscule. While it would be possible to implement a solution which stops iterating once a numeric data set only changes to a certain degree, as a proof of concept it is sensible to stick to nominal data.

\section{Implementation}
As an existing piece of software is being extended, much of the work to implement the various classifiers to be tested is already done. The main aim of this work is to integrate successfully into Weka as a plugin, and to interface with it as necessary.
\subsection{Weka Architecture}
Weka is a Java package, with a source code layout which looks somewhat like this if unused packages are omitted: \\
\dirtree{%
.1 weka. 
.2 associations. 
.2 attributeSelection. 
.2 classifiers. 
.3 AbstractClassifier. 
.3 IterativeClassifier. 
.3 SingleClassifierEnhancer. 
.2 clusterers. 
.2 core. 
.3 Attribute. 
.3 Capabilities. 
.3 Instance. 
.3 Instances. 
.3 Option. 
.3 Utils. 
.2 datagenerators.
.2 estimators.
.2 experiment.
.2 filters.
.2 gui.
.2 knowledgeflow. }
\hfill \break
Each of these subpackages is reasonably self explanatory. This project is mostly concerned with classifications, and thus code implemented for it lives in the \textit{weka.classifiers} package, although there are some dependcies on \textit{weka.core} since it contains much of the shared code for Weka. Provided that the rest of the project works as intended and the documentation is followed, there is no need to modify or discuss the other packages.

\subsection{weka.core}
From \textit{weka.core}, the following classes are used:

\begin{itemize}
\item \textbf{Instances} - An Instances object in Weka is used to represent a complete data set. It is roughly equivalent to the object representation of an ARFF file. It is comprised of a list of Instance objects,  a list of Attribute objects and some other metadata such as the class attribute for the data set and its name.
\item \textbf{Attribute} - An Attribute in Weka refers to a single column in a data set. This object is mostly responsible for tracking the column's data type (e.g. nominal, numeric, etc), and other related information common to the entire column.
\item \textbf{Instance} - An Instance is the representation of a single row of data, roughly analogous to a row in a CSV file. Belongs to a particular Instances object. Each attribute value is stored as a Java double, which is then used in connection with an Attribute object to determine the actual value of an attribute.
\item \textbf{Option} - An Option object represents a particular command line parameter, and the way in which it should be handled. This also contains a description to be printed in the help dialog if incorrect parameters are passed to the CLI.
\end{itemize}

\subsection{AbstractClassifier}
Stuff about deriving from same base classifier
\subsection{IterativeClassifier}
Stuff about next() and done()
\subsection{SingleClassifierEnhancer}
Stuff about mClassifier and stuff

\subsection{Weka Command Line Interface}
Stuff about passing options to classifiers
\subsection{Weka Graphical User Interface}
Stuff about setter methods etc
\subsection{StateAnalyser}
Stuff about how we're measuring sameness
\subsection{ProjectClassifier}
Stuff about implementation, mostly pseudocode
