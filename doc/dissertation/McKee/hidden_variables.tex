% Hidden Variables appendix

In neural networks, many hidden layers generally exist between the input and output layer and each is trained to determine particular characteristics of an input, either from the actual input or from a preceding layer \cite{stackexchange_nn}. This appendix attempts to apply a similar concept to the iterative training method described in Section \ref{design} by adding additional attributes to the training set before the algorithm begins, and investigating the effect of these on classification accuracy.

\section{Implementation}

The implemented \textit{ProjectClassifier} class from Section \ref{project_class} accepts an optional parameter (-N followed by a number), which represents the number of 'hidden variables' which will be added to the test data set during training. It is hoped that each of these hidden variables will correspond to some trend within the training data. In order to simplify the scope of the experiments, the added attributes will each be binary attributes where a 1 represents the presence of the unknown trend, while a 0 will represent its absence. 

The training algorithm then becomes the following:

\begin{enumerate}
\item Add \textit{x} additional attributes to the training set, where \textit{x} is the specified number of hidden variables
\item Fill each of these added columns with random data by adding either a 1 or a 0
\item Train the model normally
\item Once the data set has converged, remove the added columns
\end{enumerate}

This should cause the model to behave slightly differently during classification than if it had been trained without the hidden attributes.

\section{Experiments}

\subsubsection{Bayesian Network}


\begin{table}[thb]
\caption{\label{bnpi3}Bayesian Network Percent Incorrect - Three Hidden Variables}
\footnotesize
{\centering \begin{tabular}{lrr@{\hspace{0.1cm}}cr@{\hspace{0.1cm}}cr@{\hspace{0.1cm}}cr@{\hspace{0.1cm}}c}
\\
\hline
Dataset & (1)& (2) & & (3) & & (4) & & (5) & \\
\hline
audiology & 37.68 & 38.23 &           & 41.04 &   $\circ$ & 50.09 &   $\circ$ & 37.87 &          \\
autos &  9.28 &  9.34 &           & 10.83 &   $\circ$ &  9.62 &           &  9.67 &          \\
balance-scale & 29.40 & 29.61 &           & 28.49 & $\bullet$ & 28.49 & $\bullet$ & 30.05 &          \\
breast-cancer & 27.52 & 27.36 &           & 29.46 &   $\circ$ & 27.41 &           & 27.40 &          \\
bridges-version1 & 34.66 & 36.77 &   $\circ$ & 38.89 &   $\circ$ & 36.00 &           & 36.91 &   $\circ$\\
car & 21.16 & 21.41 &           & 15.41 & $\bullet$ & 23.23 &   $\circ$ & 21.24 &          \\
cmc & 52.46 & 52.96 &   $\circ$ & 49.26 & $\bullet$ & 50.24 & $\bullet$ & 50.61 & $\bullet$\\
colic & 21.11 & 21.42 &           & 19.84 & $\bullet$ & 19.94 & $\bullet$ & 20.03 & $\bullet$\\
cylinder-bands & 33.71 & 33.73 &           & 28.38 & $\bullet$ & 32.69 & $\bullet$ & 33.20 &          \\
dermatology &  4.69 &  4.75 &           &  6.10 &   $\circ$ &  3.58 & $\bullet$ &  3.43 & $\bullet$\\
diabetes & 29.30 & 29.49 &           & 27.28 & $\bullet$ & 27.52 & $\bullet$ & 27.94 & $\bullet$\\
ecoli & 35.47 & 34.66 & $\bullet$ & 33.03 & $\bullet$ & 34.53 & $\bullet$ & 34.14 & $\bullet$\\
flags & 37.56 & 38.86 &   $\circ$ & 42.66 &   $\circ$ & 40.51 &   $\circ$ & 38.81 &          \\
glass & 49.56 & 49.12 &           & 42.56 & $\bullet$ & 46.58 & $\bullet$ & 47.03 & $\bullet$\\
haberman & 29.48 & 30.00 &           & 31.03 &   $\circ$ & 27.27 & $\bullet$ & 28.48 & $\bullet$\\
hayes-roth-train & 54.88 & 55.59 &           & 46.97 & $\bullet$ & 49.23 & $\bullet$ & 49.30 & $\bullet$\\
heart-h & 18.81 & 18.47 &           & 20.18 &   $\circ$ & 19.23 &           & 17.79 & $\bullet$\\
heart-statlog & 26.90 & 26.74 &           & 26.62 &           & 27.51 &           & 26.02 &          \\
hepatitis & 18.84 & 18.63 &           & 20.20 &           & 19.20 &           & 19.13 &          \\
hypothyroid & 10.59 & 10.84 &           &  8.15 & $\bullet$ &  8.38 & $\bullet$ &  8.61 & $\bullet$\\
ionosphere & 30.67 & 32.00 &   $\circ$ & 19.62 & $\bullet$ & 23.94 & $\bullet$ & 23.66 & $\bullet$\\
iris & 26.30 & 26.37 &           & 19.66 & $\bullet$ & 23.54 & $\bullet$ & 21.27 & $\bullet$\\
kr-vs-kp & 17.31 & 17.43 &           & 11.73 & $\bullet$ & 16.69 & $\bullet$ & 15.22 & $\bullet$\\
labor &  5.60 &  5.20 &           &  8.00 &   $\circ$ &  5.20 &           &  5.00 &          \\
letter & 63.01 & 63.20 &   $\circ$ & 56.25 & $\bullet$ & 62.59 & $\bullet$ & 61.77 & $\bullet$\\
liver-disorders & 40.60 & 41.09 &           & 37.99 & $\bullet$ & 39.17 &           & 38.07 & $\bullet$\\
lung-cancer & 50.23 & 53.10 &           & 57.43 &   $\circ$ & 45.55 &           & 55.90 &   $\circ$\\
lymph & 16.07 & 16.96 &           & 16.14 &           & 16.49 &           & 16.37 &          \\
molecular-biology-promoters & 11.88 & 12.59 &           & 23.31 &   $\circ$ & 14.00 &   $\circ$ & 11.11 &          \\
mushroom &  6.90 &  7.11 &   $\circ$ &  0.57 & $\bullet$ &  7.89 &   $\circ$ &  4.16 & $\bullet$\\
nursery & 18.62 & 19.07 &   $\circ$ & 15.16 & $\bullet$ & 16.51 & $\bullet$ & 16.47 & $\bullet$\\
optdigits & 17.39 & 17.55 &   $\circ$ & 10.67 & $\bullet$ & 16.32 & $\bullet$ & 15.19 & $\bullet$\\
page-blocks & 17.54 & 23.49 &   $\circ$ &  9.23 & $\bullet$ & 14.00 & $\bullet$ & 14.18 & $\bullet$\\
pendigits & 29.71 & 29.30 & $\bullet$ & 18.20 & $\bullet$ & 26.14 & $\bullet$ & 24.65 & $\bullet$\\
postoperative-patient-data & 31.30 & 32.04 &           & 33.61 &   $\circ$ & 30.36 &           & 30.93 &          \\
primary-tumor & 55.52 & 55.68 &           & 57.97 &   $\circ$ & 54.94 &           & 54.68 & $\bullet$\\
segment & 38.36 & 39.05 &           & 26.16 & $\bullet$ & 33.04 & $\bullet$ & 30.88 & $\bullet$\\
shuttle-landing-control & 46.67 & 44.67 &           & 40.00 & $\bullet$ & 36.67 & $\bullet$ & 44.00 &          \\
sick &  9.19 &  9.28 &           &  7.47 & $\bullet$ &  8.84 & $\bullet$ &  9.52 &   $\circ$\\
solar-flare-2 & 28.68 & 28.72 &           & 28.22 &           & 28.39 &           & 28.41 &          \\
sonar & 25.39 & 25.86 &           & 23.43 & $\bullet$ & 25.33 &           & 24.09 & $\bullet$\\
soybean & 11.09 & 11.42 &           & 11.23 &           & 11.28 &           &  9.46 & $\bullet$\\
spambase & 26.17 & 25.86 &           & 22.76 & $\bullet$ & 24.14 & $\bullet$ & 25.17 & $\bullet$\\
tae & 52.15 & 51.85 &           & 53.69 &           & 51.64 &           & 49.43 & $\bullet$\\
tic-tac-toe & 30.31 & 30.37 &           & 27.29 & $\bullet$ & 29.23 & $\bullet$ & 30.08 &          \\
trains & 36.00 & 32.00 &           & 20.00 & $\bullet$ & 39.00 &           & 37.00 &          \\
vehicle & 57.05 & 57.32 &           & 42.62 & $\bullet$ & 55.79 & $\bullet$ & 53.53 & $\bullet$\\
vote &  9.59 &  9.57 &           &  6.78 & $\bullet$ &  9.34 & $\bullet$ &  9.80 &   $\circ$\\
vowel & 72.66 & 73.35 &           & 63.98 & $\bullet$ & 72.30 &           & 71.36 & $\bullet$\\
waveform-5000 & 23.91 & 24.08 &           & 24.25 &           & 23.63 &           & 23.33 & $\bullet$\\
zoo & 12.06 & 13.06 &           & 13.50 &           & 11.33 &           & 11.99 &          \\
\hline
\multicolumn{10}{c}{$\circ$, $\bullet$ statistically significant improvement or degradation}\\
\end{tabular} \footnotesize \par}
\end{table}
\begin{table}[thb]
\caption{\label{bnmae3}Bayesian Network Mean Absolute Error - Three Hidden Variables}
\footnotesize
{\centering \begin{tabular}{lrr@{\hspace{0.1cm}}cr@{\hspace{0.1cm}}cr@{\hspace{0.1cm}}cr@{\hspace{0.1cm}}c}
\\
\hline
Dataset & (1)& (2) & & (3) & & (4) & & (5) & \\
\hline
audiology & 0.03 & 0.03 &           & 0.04 &   $\circ$ & 0.04 &   $\circ$ & 0.03 &    $\circ$\\
autos & 0.10 & 0.10 &           & 0.11 &   $\circ$ & 0.10 &           & 0.10 &           \\
balance-scale & 0.26 & 0.26 &           & 0.26 & $\bullet$ & 0.28 &   $\circ$ & 0.27 &    $\circ$\\
breast-cancer & 0.32 & 0.32 &           & 0.35 &   $\circ$ & 0.32 &           & 0.33 &    $\circ$\\
bridges-version1 & 0.14 & 0.15 &   $\circ$ & 0.16 &   $\circ$ & 0.15 &   $\circ$ & 0.15 &    $\circ$\\
car & 0.12 & 0.12 &           & 0.12 &           & 0.14 &   $\circ$ & 0.14 &    $\circ$\\
cmc & 0.38 & 0.38 &           & 0.38 &   $\circ$ & 0.39 &   $\circ$ & 0.39 &    $\circ$\\
colic & 0.23 & 0.23 &           & 0.23 &           & 0.22 & $\bullet$ & 0.22 &  $\bullet$\\
cylinder-bands & 0.35 & 0.35 &   $\circ$ & 0.32 & $\bullet$ & 0.35 & $\bullet$ & 0.34 &  $\bullet$\\
dermatology & 0.02 & 0.02 &           & 0.03 &   $\circ$ & 0.02 & $\bullet$ & 0.02 &  $\bullet$\\
diabetes & 0.33 & 0.33 &           & 0.34 &   $\circ$ & 0.33 &           & 0.33 &           \\
ecoli & 0.10 & 0.10 &           & 0.11 &   $\circ$ & 0.11 &   $\circ$ & 0.10 &    $\circ$\\
flags & 0.10 & 0.10 &   $\circ$ & 0.11 &   $\circ$ & 0.11 &   $\circ$ & 0.10 &    $\circ$\\
glass & 0.16 & 0.16 &           & 0.16 &           & 0.16 &   $\circ$ & 0.16 &    $\circ$\\
haberman & 0.37 & 0.37 &           & 0.38 &   $\circ$ & 0.37 &           & 0.38 &    $\circ$\\
hayes-roth-train & 0.31 & 0.31 &           & 0.30 & $\bullet$ & 0.31 &           & 0.31 &  $\bullet$\\
heart-h & 0.09 & 0.09 &           & 0.11 &   $\circ$ & 0.10 &   $\circ$ & 0.09 &    $\circ$\\
heart-statlog & 0.30 & 0.30 &           & 0.33 &   $\circ$ & 0.31 &   $\circ$ & 0.31 &    $\circ$\\
hepatitis & 0.20 & 0.20 &           & 0.23 &   $\circ$ & 0.21 &   $\circ$ & 0.21 &    $\circ$\\
hypothyroid & 0.07 & 0.07 &           & 0.06 & $\bullet$ & 0.07 & $\bullet$ & 0.07 &  $\bullet$\\
ionosphere & 0.31 & 0.32 &   $\circ$ & 0.22 & $\bullet$ & 0.26 & $\bullet$ & 0.26 &  $\bullet$\\
iris & 0.20 & 0.20 &   $\circ$ & 0.22 &   $\circ$ & 0.21 &   $\circ$ & 0.19 &           \\
kr-vs-kp & 0.24 & 0.24 &           & 0.18 & $\bullet$ & 0.25 &   $\circ$ & 0.23 &  $\bullet$\\
labor & 0.08 & 0.08 &           & 0.11 &   $\circ$ & 0.09 &   $\circ$ & 0.08 &           \\
letter & 0.06 & 0.06 &           & 0.05 & $\bullet$ & 0.06 &   $\circ$ & 0.06 &    $\circ$\\
liver-disorders & 0.46 & 0.46 &           & 0.44 & $\bullet$ & 0.46 &           & 0.46 &           \\
lung-cancer & 0.33 & 0.35 &           & 0.39 &   $\circ$ & 0.31 &           & 0.37 &    $\circ$\\
lymph & 0.10 & 0.10 &           & 0.10 &           & 0.11 &   $\circ$ & 0.10 &           \\
molecular-biology-promoters & 0.14 & 0.14 &           & 0.24 &   $\circ$ & 0.15 &   $\circ$ & 0.13 &           \\
mushroom & 0.07 & 0.07 &   $\circ$ & 0.01 & $\bullet$ & 0.08 &   $\circ$ & 0.04 &  $\bullet$\\
nursery & 0.11 & 0.11 &           & 0.10 & $\bullet$ & 0.13 &   $\circ$ & 0.11 &           \\
optdigits & 0.04 & 0.04 &   $\circ$ & 0.03 & $\bullet$ & 0.04 &   $\circ$ & 0.03 &  $\bullet$\\
page-blocks & 0.09 & 0.10 &   $\circ$ & 0.06 & $\bullet$ & 0.07 & $\bullet$ & 0.08 &  $\bullet$\\
pendigits & 0.07 & 0.07 &           & 0.05 & $\bullet$ & 0.07 &   $\circ$ & 0.06 &  $\bullet$\\
postoperative-patient-data & 0.28 & 0.28 &           & 0.28 &           & 0.27 & $\bullet$ & 0.28 &           \\
primary-tumor & 0.06 & 0.06 &           & 0.06 &   $\circ$ & 0.06 &   $\circ$ & 0.06 &    $\circ$\\
segment & 0.12 & 0.12 &           & 0.10 & $\bullet$ & 0.11 & $\bullet$ & 0.10 &  $\bullet$\\
shuttle-landing-control & 0.44 & 0.43 &           & 0.45 &   $\circ$ & 0.44 &   $\circ$ & 0.43 &  $\bullet$\\
sick & 0.11 & 0.11 &           & 0.10 & $\bullet$ & 0.11 &   $\circ$ & 0.12 &    $\circ$\\
solar-flare-2 & 0.11 & 0.11 &   $\circ$ & 0.12 &   $\circ$ & 0.12 &   $\circ$ & 0.11 &    $\circ$\\
sonar & 0.26 & 0.26 &           & 0.27 &           & 0.26 &           & 0.26 &           \\
soybean & 0.01 & 0.01 &           & 0.01 &   $\circ$ & 0.01 &   $\circ$ & 0.01 &  $\bullet$\\
spambase & 0.28 & 0.27 &           & 0.30 &   $\circ$ & 0.28 &   $\circ$ & 0.28 &    $\circ$\\
tae & 0.40 & 0.40 &           & 0.42 &   $\circ$ & 0.41 &   $\circ$ & 0.40 &    $\circ$\\
tic-tac-toe & 0.36 & 0.36 &           & 0.36 &           & 0.37 &   $\circ$ & 0.38 &    $\circ$\\
trains & 0.37 & 0.34 &           & 0.23 & $\bullet$ & 0.35 &           & 0.34 &           \\
vehicle & 0.29 & 0.29 &           & 0.25 & $\bullet$ & 0.29 & $\bullet$ & 0.28 &  $\bullet$\\
vote & 0.10 & 0.10 &           & 0.08 & $\bullet$ & 0.10 &   $\circ$ & 0.10 &           \\
vowel & 0.14 & 0.14 &           & 0.13 & $\bullet$ & 0.15 &   $\circ$ & 0.14 &    $\circ$\\
waveform-5000 & 0.17 & 0.17 &           & 0.19 &   $\circ$ & 0.17 &   $\circ$ & 0.17 &  $\bullet$\\
zoo & 0.04 & 0.04 &   $\circ$ & 0.05 &   $\circ$ & 0.04 &           & 0.04 &           \\
\hline
\multicolumn{10}{c}{$\circ$, $\bullet$ statistically significant improvement or degradation}\\
\end{tabular} \footnotesize \par}
\end{table}
\newpage
{\centering \footnotesize \begin{longtable}{lrr@{\hspace{0.1cm}}cr@{\hspace{0.1cm}}cr@{\hspace{0.1cm}}cr@{\hspace{0.1cm}}c}
\caption{\label{bnrmse3}Bayesian Network Root Mean Squared Error - Three Hidden Variables}
\\
\hline
Dataset & (1)& (2) & & (3) & & (4) & & (5) & \\
\hline
audiology & 0.16 & 0.16 &           & 0.16 &   $\circ$ & 0.17 &   $\circ$ & 0.15 & $\bullet$\\
autos & 0.28 & 0.28 &           & 0.30 &   $\circ$ & 0.27 &           & 0.28 &          \\
balance-scale & 0.35 & 0.35 &           & 0.35 &           & 0.36 &   $\circ$ & 0.35 &          \\
breast-cancer & 0.46 & 0.46 &           & 0.45 &           & 0.44 & $\bullet$ & 0.45 & $\bullet$\\
bridges-version1 & 0.29 & 0.30 &   $\circ$ & 0.30 &   $\circ$ & 0.29 &           & 0.29 &          \\
car & 0.27 & 0.27 &           & 0.23 & $\bullet$ & 0.26 & $\bullet$ & 0.26 & $\bullet$\\
cmc & 0.47 & 0.47 &           & 0.45 & $\bullet$ & 0.45 & $\bullet$ & 0.46 & $\bullet$\\
colic & 0.42 & 0.43 &           & 0.40 & $\bullet$ & 0.41 & $\bullet$ & 0.41 & $\bullet$\\
cylinder-bands & 0.50 & 0.50 &   $\circ$ & 0.45 & $\bullet$ & 0.47 & $\bullet$ & 0.47 & $\bullet$\\
dermatology & 0.10 & 0.10 &           & 0.12 &   $\circ$ & 0.09 & $\bullet$ & 0.09 & $\bullet$\\
diabetes & 0.44 & 0.44 &           & 0.42 & $\bullet$ & 0.42 & $\bullet$ & 0.43 & $\bullet$\\
ecoli & 0.24 & 0.24 &           & 0.23 & $\bullet$ & 0.23 & $\bullet$ & 0.23 & $\bullet$\\
flags & 0.27 & 0.27 &           & 0.28 &   $\circ$ & 0.27 &           & 0.27 &          \\
glass & 0.30 & 0.30 &           & 0.29 & $\bullet$ & 0.29 & $\bullet$ & 0.29 & $\bullet$\\
haberman & 0.45 & 0.45 &           & 0.45 &           & 0.44 & $\bullet$ & 0.45 &          \\
hayes-roth-train & 0.40 & 0.40 &           & 0.38 & $\bullet$ & 0.39 & $\bullet$ & 0.39 & $\bullet$\\
heart-h & 0.24 & 0.24 &           & 0.24 &   $\circ$ & 0.23 & $\bullet$ & 0.23 & $\bullet$\\
heart-statlog & 0.44 & 0.44 &           & 0.42 & $\bullet$ & 0.42 & $\bullet$ & 0.42 & $\bullet$\\
hepatitis & 0.39 & 0.39 &           & 0.39 &           & 0.39 & $\bullet$ & 0.40 &          \\
hypothyroid & 0.20 & 0.20 &           & 0.18 & $\bullet$ & 0.18 & $\bullet$ & 0.18 & $\bullet$\\
ionosphere & 0.50 & 0.51 &   $\circ$ & 0.39 & $\bullet$ & 0.43 & $\bullet$ & 0.43 & $\bullet$\\
iris & 0.35 & 0.36 &   $\circ$ & 0.33 & $\bullet$ & 0.34 & $\bullet$ & 0.33 & $\bullet$\\
kr-vs-kp & 0.34 & 0.34 &   $\circ$ & 0.29 & $\bullet$ & 0.34 & $\bullet$ & 0.33 & $\bullet$\\
labor & 0.18 & 0.18 &           & 0.22 &   $\circ$ & 0.19 &           & 0.17 &          \\
letter & 0.18 & 0.18 &           & 0.16 & $\bullet$ & 0.17 & $\bullet$ & 0.17 & $\bullet$\\
liver-disorders & 0.49 & 0.49 &           & 0.48 & $\bullet$ & 0.48 & $\bullet$ & 0.48 & $\bullet$\\
lung-cancer & 0.51 & 0.53 &           & 0.56 &   $\circ$ & 0.48 & $\bullet$ & 0.54 &   $\circ$\\
lymph & 0.24 & 0.25 &   $\circ$ & 0.24 &           & 0.24 & $\bullet$ & 0.24 & $\bullet$\\
molecular-biology-promoters & 0.30 & 0.30 &           & 0.43 &   $\circ$ & 0.31 &           & 0.29 &          \\
mushroom & 0.25 & 0.25 &   $\circ$ & 0.07 & $\bullet$ & 0.26 &   $\circ$ & 0.18 & $\bullet$\\
nursery & 0.24 & 0.24 &           & 0.21 & $\bullet$ & 0.23 & $\bullet$ & 0.22 & $\bullet$\\
optdigits & 0.17 & 0.17 &   $\circ$ & 0.13 & $\bullet$ & 0.16 & $\bullet$ & 0.15 & $\bullet$\\
page-blocks & 0.24 & 0.26 &   $\circ$ & 0.17 & $\bullet$ & 0.20 & $\bullet$ & 0.21 & $\bullet$\\
pendigits & 0.21 & 0.21 &           & 0.16 & $\bullet$ & 0.20 & $\bullet$ & 0.19 & $\bullet$\\
postoperative-patient-data & 0.40 & 0.40 &           & 0.41 &   $\circ$ & 0.39 &           & 0.39 & $\bullet$\\
primary-tumor & 0.18 & 0.18 &           & 0.18 &           & 0.18 & $\bullet$ & 0.18 & $\bullet$\\
segment & 0.29 & 0.29 &           & 0.23 & $\bullet$ & 0.26 & $\bullet$ & 0.25 & $\bullet$\\
shuttle-landing-control & 0.48 & 0.48 &           & 0.49 &   $\circ$ & 0.48 &           & 0.48 &          \\
sick & 0.26 & 0.26 &           & 0.23 & $\bullet$ & 0.25 & $\bullet$ & 0.26 &          \\
solar-flare-2 & 0.26 & 0.26 &   $\circ$ & 0.25 & $\bullet$ & 0.25 & $\bullet$ & 0.25 & $\bullet$\\
sonar & 0.46 & 0.46 &           & 0.42 & $\bullet$ & 0.45 & $\bullet$ & 0.44 & $\bullet$\\
soybean & 0.10 & 0.10 &           & 0.09 & $\bullet$ & 0.10 &           & 0.09 & $\bullet$\\
spambase & 0.44 & 0.44 &           & 0.39 & $\bullet$ & 0.42 & $\bullet$ & 0.42 & $\bullet$\\
tae & 0.47 & 0.47 &           & 0.47 &   $\circ$ & 0.46 & $\bullet$ & 0.46 & $\bullet$\\
tic-tac-toe & 0.44 & 0.44 &           & 0.43 & $\bullet$ & 0.44 & $\bullet$ & 0.44 & $\bullet$\\
trains & 0.47 & 0.43 &           & 0.29 & $\bullet$ & 0.43 &           & 0.43 &          \\
vehicle & 0.46 & 0.46 &           & 0.37 & $\bullet$ & 0.44 & $\bullet$ & 0.43 & $\bullet$\\
vote & 0.29 & 0.29 &           & 0.23 & $\bullet$ & 0.29 & $\bullet$ & 0.29 &          \\
vowel & 0.28 & 0.29 &   $\circ$ & 0.27 & $\bullet$ & 0.28 & $\bullet$ & 0.27 & $\bullet$\\
waveform-5000 & 0.36 & 0.36 &           & 0.34 & $\bullet$ & 0.35 & $\bullet$ & 0.35 & $\bullet$\\
zoo & 0.15 & 0.16 &   $\circ$ & 0.15 &           & 0.14 & $\bullet$ & 0.14 &          \\
\hline
\multicolumn{10}{c}{$\circ$, $\bullet$ statistically significant improvement or degradation}\\
\end{longtable} \footnotesize \par}

\begin{table}[thb]
\caption{\label{bnmeg3}Bayesian Network Mean Entropy Gain - Three Hidden Variables}
\footnotesize
{\centering \begin{tabular}{lrr@{\hspace{0.1cm}}cr@{\hspace{0.1cm}}cr@{\hspace{0.1cm}}cr@{\hspace{0.1cm}}c}
\\
\hline
Dataset & (1)& (2) & & (3) & & (4) & & (5) & \\
\hline
audiology & -1.01 & -1.00 &           & -1.86 & $\bullet$ & -2.14 & $\bullet$ & -0.61 &   $\circ$\\
autos & -0.12 & -0.13 &           &  0.20 &   $\circ$ &  0.11 &   $\circ$ &  0.02 &   $\circ$\\
balance-scale &  0.36 &  0.36 &           &  0.37 &   $\circ$ &  0.32 & $\bullet$ &  0.35 & $\bullet$\\
breast-cancer & -0.09 & -0.09 &           & -0.02 &   $\circ$ & -0.00 &   $\circ$ & -0.05 &   $\circ$\\
bridges-version1 &  0.47 &  0.43 &           &  0.61 &   $\circ$ &  0.67 &   $\circ$ &  0.60 &   $\circ$\\
car &  0.25 &  0.25 &           &  0.67 &   $\circ$ &  0.57 &   $\circ$ &  0.60 &   $\circ$\\
cmc & -0.11 & -0.11 &           &  0.09 &   $\circ$ &  0.04 &   $\circ$ &  0.03 &   $\circ$\\
colic & -0.43 & -0.44 &           &  0.00 &   $\circ$ & -0.17 &   $\circ$ & -0.21 &   $\circ$\\
cylinder-bands & -0.17 & -0.20 & $\bullet$ &  0.08 &   $\circ$ &  0.04 &   $\circ$ &  0.02 &   $\circ$\\
dermatology &  2.25 &  2.25 &           &  2.15 & $\bullet$ &  2.29 &   $\circ$ &  2.30 &   $\circ$\\
diabetes &  0.11 &  0.10 &           &  0.17 &   $\circ$ &  0.16 &   $\circ$ &  0.16 &   $\circ$\\
ecoli &  0.75 &  0.78 &           &  0.97 &   $\circ$ &  0.96 &   $\circ$ &  0.93 &   $\circ$\\
flags & -0.08 & -0.05 &           & -0.09 &           &  0.27 &   $\circ$ &  0.18 &   $\circ$\\
glass &  0.37 &  0.37 &           &  0.49 &   $\circ$ &  0.53 &   $\circ$ &  0.49 &   $\circ$\\
haberman & -0.04 & -0.04 &           & -0.03 &           & -0.01 &   $\circ$ & -0.02 &   $\circ$\\
hayes-roth-train &  0.07 &  0.06 &           &  0.15 &   $\circ$ &  0.10 &   $\circ$ &  0.13 &   $\circ$\\
heart-h &  0.22 &  0.23 &           &  0.26 &   $\circ$ &  0.30 &   $\circ$ &  0.29 &   $\circ$\\
heart-statlog &  0.11 &  0.11 &           &  0.21 &   $\circ$ &  0.22 &   $\circ$ &  0.20 &   $\circ$\\
hepatitis & -0.25 & -0.25 &           & -0.14 &   $\circ$ & -0.09 &   $\circ$ & -0.18 &   $\circ$\\
hypothyroid &  0.04 &  0.02 &           &  0.10 &   $\circ$ &  0.09 &   $\circ$ &  0.09 &   $\circ$\\
ionosphere & -0.64 & -0.76 & $\bullet$ &  0.01 &   $\circ$ & -0.08 &   $\circ$ & -0.09 &   $\circ$\\
iris &  0.64 &  0.61 &           &  0.79 &   $\circ$ &  0.76 &   $\circ$ &  0.79 &   $\circ$\\
kr-vs-kp &  0.48 &  0.47 & $\bullet$ &  0.60 &   $\circ$ &  0.48 &   $\circ$ &  0.52 &   $\circ$\\
labor &  0.68 &  0.70 &   $\circ$ &  0.65 &           &  0.70 &           &  0.72 &   $\circ$\\
letter &  1.38 &  1.37 &           &  2.03 &   $\circ$ &  1.58 &   $\circ$ &  1.68 &   $\circ$\\
liver-disorders &  0.02 &  0.01 &           &  0.06 &   $\circ$ &  0.03 &   $\circ$ &  0.05 &   $\circ$\\
lung-cancer & -2.13 & -2.08 &           & -1.97 &           & -1.21 &   $\circ$ & -1.94 &          \\
lymph &  0.68 &  0.64 &           &  0.62 &           &  0.72 &   $\circ$ &  0.71 &   $\circ$\\
molecular-biology-promoters &  0.41 &  0.38 &           & -0.28 & $\bullet$ &  0.43 &           &  0.49 &   $\circ$\\
mushroom &  0.48 &  0.47 & $\bullet$ &  0.97 &   $\circ$ &  0.57 &   $\circ$ &  0.80 &   $\circ$\\
nursery &  0.76 &  0.78 &           &  1.14 &   $\circ$ &  1.00 &   $\circ$ &  1.08 &   $\circ$\\
optdigits &  2.10 &  2.09 & $\bullet$ &  2.78 &   $\circ$ &  2.37 &   $\circ$ &  2.41 &   $\circ$\\
page-blocks & -0.14 & -0.27 & $\bullet$ &  0.24 &   $\circ$ &  0.11 &   $\circ$ &  0.09 &   $\circ$\\
pendigits &  1.56 &  1.57 &           &  2.49 &   $\circ$ &  2.00 &   $\circ$ &  2.04 &   $\circ$\\
postoperative-patient-data & -0.22 & -0.23 &           & -0.24 &           & -0.17 &   $\circ$ & -0.17 &   $\circ$\\
primary-tumor &  0.51 &  0.51 &           &  0.62 &   $\circ$ &  0.71 &   $\circ$ &  0.70 &   $\circ$\\
segment &  0.64 &  0.65 &           &  1.81 &   $\circ$ &  1.25 &   $\circ$ &  1.30 &   $\circ$\\
shuttle-landing-control &  0.04 &  0.05 &           & -0.01 & $\bullet$ &  0.04 &           &  0.05 &          \\
sick & -0.00 & -0.00 &           &  0.08 &   $\circ$ &  0.04 &   $\circ$ &  0.02 &   $\circ$\\
solar-flare-2 &  1.21 &  1.17 & $\bullet$ &  1.32 &   $\circ$ &  1.28 &   $\circ$ &  1.33 &   $\circ$\\
sonar & -0.68 & -0.68 &           &  0.09 &   $\circ$ & -0.33 &   $\circ$ & -0.30 &   $\circ$\\
soybean &  3.10 &  3.09 &           &  3.31 &   $\circ$ &  3.23 &   $\circ$ &  3.27 &   $\circ$\\
spambase &  0.02 &  0.01 & $\bullet$ &  0.28 &   $\circ$ &  0.17 &   $\circ$ &  0.15 &   $\circ$\\
tae & -0.02 & -0.03 &           & -0.02 &           &  0.03 &   $\circ$ &  0.03 &   $\circ$\\
tic-tac-toe &  0.11 &  0.11 &           &  0.16 &   $\circ$ &  0.12 &   $\circ$ &  0.13 &   $\circ$\\
trains & -0.73 & -0.37 &   $\circ$ &  0.32 &   $\circ$ &  0.01 &   $\circ$ & -0.11 &   $\circ$\\
vehicle & -1.48 & -1.49 &           &  0.51 &   $\circ$ & -0.68 &   $\circ$ & -0.79 &   $\circ$\\
vote &  0.05 &  0.05 &           &  0.66 &   $\circ$ &  0.28 &   $\circ$ &  0.18 &   $\circ$\\
vowel &  0.38 &  0.30 & $\bullet$ &  1.00 &   $\circ$ &  0.63 &   $\circ$ &  0.69 &   $\circ$\\
waveform-5000 &  0.31 &  0.32 &           &  0.78 &   $\circ$ &  0.60 &   $\circ$ &  0.55 &   $\circ$\\
zoo &  1.96 &  1.88 & $\bullet$ &  1.94 &           &  2.06 &   $\circ$ &  2.04 &   $\circ$\\
\hline
\multicolumn{10}{c}{$\circ$, $\bullet$ statistically significant improvement or degradation}\\
\end{tabular} \footnotesize \par}
\end{table}
\FloatBarrier
