% Hidden Variables appendix

In neural networks, many hidden layers generally exist between the input and output layer and each is trained to determine particular characteristics of an input, either from the actual input or from a preceding layer \cite{stackexchange_nn}. This appendix attempts to apply a similar concept to the iterative training method described in Section \ref{design} by adding additional attributes to the training set before the algorithm begins, and investigating the effect of these on classification accuracy.

\section{Implementation}

The implemented \textit{ProjectClassifier} class from Section \ref{project_class} accepts an optional parameter (-N followed by a number), which represents the number of 'hidden variables' which will be added to the test data set during training. It is hoped that each of these hidden variables will correspond to some trend within the training data. In order to simplify the scope of the experiments, the added attributes will each be binary attributes where a 1 represents the presence of the unknown trend, while a 0 will represent its absence. 

The training algorithm then becomes the following:

\begin{enumerate}
\item Add \textit{x} additional attributes to the training set, where \textit{x} is the specified number of hidden variables
\item Fill each of these added columns with random data by adding either a 1 or a 0
\item Train the model normally
\item Once the data set has converged, remove the added columns
\end{enumerate}

This should cause the model to behave slightly differently during classification than if it had been trained without the hidden attributes.

\section{Experiments}

The experiments described here follow much the same format as those described in section \ref{experiments}, with only slight alterations. Exactly the same test cases and data sets are used as those described there, except hidden variables are added to Cases 1 and 2. Experiments will be carried out with three and five added hidden variables, and these will then be compared with relevant results found with 0 hidden variables earlier in the work.

\subsection{Naive Bayes}

The results for the Naive Bayes classifier with various numbers of hidden variables can be seen in figures \ref{nbpi3} - \ref{nbmeg5}.

From the percent incorrect results (figures \ref{nbpi}, \ref{nbpi3} and \ref{nbpi5}), it can be seen that increasing the number of hidden variables decreased the accuracy of the classifier quite significantly. With no hidden variables, Case 1 was roughly as accurate as Case 3 in 30 tests, and less accurate in 19. With three hidden variables it was as accurate in 19 tests and less accurate in 28 cases, and with 5 hidden variables it was even in 17 tests and performed significantly worse in 32 cases. At five hidden variables Case 1 was roughly as accurate as Case 4, and actually less accurate than Case 5.

The Naive Bayes MAE results (figures \ref{nbmae}, \ref{nbmae3} and \ref{nbmae5}) show a similar trend. While Case 1 outperformed Cases 3 and 5 with 0 hidden variables, increasing the number of hidden variables caused Case 1 to perform less well relative to these classifiers. Case 1 still performed better than Case 4 in all three experiments.

The Naive Bayes RMSE results (figures \ref{nbrmse}, \ref{nbrmse3} and \ref{nbrmse5}) show that adding hidden variables during training appears to significantly lower RMSE. Case 1 was outperformed by the other cases in all three experiments, and the number of tests in which they were outperformed increased from 36 to 41 when three hidden variables were added. This is probably because the iterative training method will generally lead to very strong predictions one way or the other, and adding additional random data to this is likely to cause more extreme predictions.

From the Naive Bayes MEG results in figures \ref{nbmeg}, \ref{nbmeg3} and \ref{nbmeg5}, it can be seen that adding additional hidden variables caused large reductions in entropy. With three added hidden variables Case 1 showed larger entropy reductions than Cases 4 and 5, with large increases in the number of tests in which this was observed. A similar trend was observed when five hidden variables were added. This is probably due to so the amount of additional data being added to the system (${X~hidden~variables}\times{Y~rows~of~data}$). It is assumed that this causes the iterative training to find and favour certain trends, and therefore reduce the entropy of the whole data set.

\linespread{1.0}
\newpage
{\centering \footnotesize \begin{longtable}{lrr@{\hspace{0.1cm}}cr@{\hspace{0.1cm}}cr@{\hspace{0.1cm}}cr@{\hspace{0.1cm}}c}
\caption{\label{nbpi3}Naive Bayes Percent Incorrect - Three Hidden Variables}
\\
\hline
Dataset & (1)& (2) & & (3) & & (4) & & (5) & \\
\hline
audiology & 37.44 & 38.34 &   $\circ$ & 37.28 &           & 50.09 &    $\circ$ & 37.28 &          \\
autos &  9.89 &  9.94 &           & 10.22 &           &  9.62 &            & 10.22 &          \\
balance-scale & 25.78 & 26.03 &           & 26.26 &           & 28.49 &    $\circ$ & 26.26 &          \\
breast-cancer & 27.33 & 27.52 &           & 27.14 &           & 27.41 &            & 27.14 &          \\
bridges-version1 & 34.68 & 34.57 &           & 35.45 &           & 36.00 &            & 35.45 &          \\
car & 19.51 & 19.75 &           & 19.46 &           & 23.23 &    $\circ$ & 19.46 &          \\
cmc & 51.92 & 52.39 &   $\circ$ & 50.69 & $\bullet$ & 50.24 &  $\bullet$ & 50.69 & $\bullet$\\
colic & 22.65 & 22.22 &           & 20.92 & $\bullet$ & 19.94 &  $\bullet$ & 20.92 & $\bullet$\\
cylinder-bands & 32.93 & 32.73 &           & 31.79 & $\bullet$ & 32.69 &            & 31.79 & $\bullet$\\
dermatology &  3.10 &  3.16 &           &  2.95 &           &  3.58 &            &  2.95 &          \\
diabetes & 29.18 & 28.88 &           & 27.79 & $\bullet$ & 27.52 &  $\bullet$ & 27.79 & $\bullet$\\
ecoli & 33.39 & 32.96 &           & 32.67 & $\bullet$ & 34.53 &    $\circ$ & 32.67 & $\bullet$\\
flags & 36.86 & 37.05 &           & 38.46 &   $\circ$ & 40.51 &    $\circ$ & 38.46 &   $\circ$\\
glass & 47.78 & 45.17 & $\bullet$ & 46.42 &           & 46.58 &            & 46.42 &          \\
haberman & 28.34 & 27.93 &           & 27.71 &           & 27.27 &  $\bullet$ & 27.71 &          \\
hayes-roth-train & 53.76 & 55.09 &           & 47.16 & $\bullet$ & 49.23 &  $\bullet$ & 47.16 & $\bullet$\\
heart-h & 17.75 & 18.13 &   $\circ$ & 18.39 &   $\circ$ & 19.23 &    $\circ$ & 18.39 &   $\circ$\\
heart-statlog & 26.74 & 27.03 &           & 27.10 &           & 27.51 &            & 27.10 &          \\
hepatitis & 18.84 & 18.99 &           & 19.06 &           & 19.20 &            & 19.06 &          \\
hypothyroid & 11.07 & 12.42 &   $\circ$ &  8.74 & $\bullet$ &  8.38 &  $\bullet$ &  8.74 & $\bullet$\\
ionosphere & 31.46 & 30.99 & $\bullet$ & 24.93 & $\bullet$ & 23.94 &  $\bullet$ & 24.93 & $\bullet$\\
iris & 21.89 & 21.43 & $\bullet$ & 21.51 &           & 23.54 &    $\circ$ & 21.51 &          \\
kr-vs-kp & 16.53 & 16.50 &           & 14.01 & $\bullet$ & 16.69 &            & 14.01 & $\bullet$\\
labor &  5.40 &  5.00 &           &  4.80 &           &  5.20 &            &  4.80 &          \\
letter & 62.20 & 62.01 & $\bullet$ & 59.84 & $\bullet$ & 62.59 &    $\circ$ & 59.84 & $\bullet$\\
liver-disorders & 39.38 & 40.30 &           & 37.31 & $\bullet$ & 39.17 &            & 37.31 & $\bullet$\\
lung-cancer & 47.87 & 51.21 &           & 53.02 &   $\circ$ & 45.55 &            & 53.02 &   $\circ$\\
lymph & 16.23 & 16.23 &           & 15.65 &           & 16.49 &            & 15.65 &          \\
molecular-biology-promoters &  9.95 &  9.53 &           &  8.56 &           & 14.00 &    $\circ$ &  8.56 &          \\
mushroom &  6.89 &  6.86 &           &  5.06 & $\bullet$ &  7.89 &    $\circ$ &  5.06 & $\bullet$\\
nursery & 16.11 & 16.06 &           & 15.87 & $\bullet$ & 16.51 &    $\circ$ & 15.87 & $\bullet$\\
optdigits & 14.03 & 14.05 &           & 13.10 & $\bullet$ & 16.32 &    $\circ$ & 13.10 & $\bullet$\\
page-blocks & 27.29 & 19.30 & $\bullet$ & 16.82 & $\bullet$ & 14.00 &  $\bullet$ & 16.82 & $\bullet$\\
pendigits & 23.03 & 22.49 & $\bullet$ & 21.21 & $\bullet$ & 26.14 &    $\circ$ & 21.21 & $\bullet$\\
postoperative-patient-data & 31.16 & 32.42 &           & 29.98 &           & 30.36 &            & 29.98 &          \\
primary-tumor & 55.55 & 55.16 &           & 54.13 & $\bullet$ & 54.94 &            & 54.13 & $\bullet$\\
segment & 30.06 & 30.79 &   $\circ$ & 28.12 & $\bullet$ & 33.04 &    $\circ$ & 28.12 & $\bullet$\\
shuttle-landing-control & 41.33 & 44.67 &           & 46.67 &   $\circ$ & 36.67 &            & 46.67 &   $\circ$\\
sick & 10.73 & 10.23 & $\bullet$ &  9.81 & $\bullet$ &  8.84 &  $\bullet$ &  9.81 & $\bullet$\\
solar-flare-2 & 28.29 & 28.33 &           & 28.16 &           & 28.39 &            & 28.16 &          \\
sonar & 26.16 & 25.85 &           & 24.14 & $\bullet$ & 25.33 &            & 24.14 & $\bullet$\\
soybean & 10.44 & 10.34 &           &  9.77 & $\bullet$ & 11.28 &    $\circ$ &  9.77 & $\bullet$\\
spambase & 25.42 & 24.84 & $\bullet$ & 24.51 & $\bullet$ & 24.14 &  $\bullet$ & 24.51 & $\bullet$\\
tae & 51.29 & 49.86 &           & 48.92 & $\bullet$ & 51.64 &            & 48.92 & $\bullet$\\
tic-tac-toe & 29.98 & 30.07 &           & 29.49 & $\bullet$ & 29.23 &  $\bullet$ & 29.49 & $\bullet$\\
trains & 36.00 & 33.00 &           & 31.00 &           & 39.00 &            & 31.00 &          \\
vehicle & 56.72 & 56.82 &           & 53.71 & $\bullet$ & 55.79 &  $\bullet$ & 53.71 & $\bullet$\\
vote &  9.57 &  9.57 &           &  9.47 & $\bullet$ &  9.34 &            &  9.47 & $\bullet$\\
vowel & 71.54 & 71.71 &           & 69.27 & $\bullet$ & 72.30 &            & 69.27 & $\bullet$\\
waveform-5000 & 22.62 & 22.65 &           & 22.44 & $\bullet$ & 23.63 &    $\circ$ & 22.44 & $\bullet$\\
zoo &  9.11 &  9.12 &           &  8.78 &           & 11.33 &    $\circ$ &  8.78 &          \\
\hline
\multicolumn{10}{c}{$\circ$, $\bullet$ statistically significant improvement or degradation}\\
\end{longtable} \footnotesize \par}
\newpage
{\centering \footnotesize \begin{longtable}{lrr@{\hspace{0.1cm}}cr@{\hspace{0.1cm}}cr@{\hspace{0.1cm}}cr@{\hspace{0.1cm}}c}
\caption{\label{nbpi5}Naive Bayes Percent Incorrect - Five Hidden Variables}
\\
\hline
Dataset & (1)& (2) & & (3) & & (4) & & (5) & \\
\hline
audiology & 38.18 & 38.44 &           & 37.28 & $\bullet$ & 50.09 &    $\circ$ & 37.28 & $\bullet$\\
autos &  9.94 &  9.94 &           & 10.22 &           &  9.62 &            & 10.22 &          \\
balance-scale & 26.08 & 26.27 &           & 26.26 &           & 28.49 &    $\circ$ & 26.26 &          \\
breast-cancer & 27.86 & 27.67 &           & 27.14 & $\bullet$ & 27.41 &            & 27.14 & $\bullet$\\
bridges-version1 & 34.85 & 33.87 &           & 35.45 &           & 36.00 &            & 35.45 &          \\
car & 19.51 & 19.71 &           & 19.46 &           & 23.23 &    $\circ$ & 19.46 &          \\
cmc & 52.00 & 52.32 &           & 50.69 & $\bullet$ & 50.24 &  $\bullet$ & 50.69 & $\bullet$\\
colic & 22.74 & 22.25 & $\bullet$ & 20.92 & $\bullet$ & 19.94 &  $\bullet$ & 20.92 & $\bullet$\\
cylinder-bands & 33.08 & 32.89 &           & 31.79 & $\bullet$ & 32.69 &            & 31.79 & $\bullet$\\
dermatology &  3.19 &  3.04 &           &  2.95 &           &  3.58 &            &  2.95 &          \\
diabetes & 29.15 & 28.90 &           & 27.79 & $\bullet$ & 27.52 &  $\bullet$ & 27.79 & $\bullet$\\
ecoli & 32.90 & 33.35 &           & 32.67 &           & 34.53 &    $\circ$ & 32.67 &          \\
flags & 37.22 & 37.26 &           & 38.46 &   $\circ$ & 40.51 &    $\circ$ & 38.46 &   $\circ$\\
glass & 48.25 & 44.82 & $\bullet$ & 46.42 & $\bullet$ & 46.58 &            & 46.42 & $\bullet$\\
haberman & 28.53 & 28.12 &           & 27.71 & $\bullet$ & 27.27 &  $\bullet$ & 27.71 & $\bullet$\\
hayes-roth-train & 55.26 & 52.65 &           & 47.16 & $\bullet$ & 49.23 &  $\bullet$ & 47.16 & $\bullet$\\
heart-h & 17.82 & 18.09 &           & 18.39 &   $\circ$ & 19.23 &    $\circ$ & 18.39 &   $\circ$\\
heart-statlog & 26.87 & 27.03 &           & 27.10 &           & 27.51 &            & 27.10 &          \\
hepatitis & 18.98 & 18.70 &           & 19.06 &           & 19.20 &            & 19.06 &          \\
hypothyroid & 13.06 & 12.31 &           &  8.74 & $\bullet$ &  8.38 &  $\bullet$ &  8.74 & $\bullet$\\
ionosphere & 32.03 & 30.92 & $\bullet$ & 24.93 & $\bullet$ & 23.94 &  $\bullet$ & 24.93 & $\bullet$\\
iris & 21.90 & 21.74 &           & 21.51 &           & 23.54 &    $\circ$ & 21.51 &          \\
kr-vs-kp & 16.82 & 16.53 & $\bullet$ & 14.01 & $\bullet$ & 16.69 &            & 14.01 & $\bullet$\\
labor &  5.00 &  5.00 &           &  4.80 &           &  5.20 &            &  4.80 &          \\
letter & 62.56 & 62.00 & $\bullet$ & 59.84 & $\bullet$ & 62.59 &            & 59.84 & $\bullet$\\
liver-disorders & 39.77 & 40.37 &           & 37.31 & $\bullet$ & 39.17 &            & 37.31 & $\bullet$\\
lung-cancer & 49.79 & 49.54 &           & 53.02 &           & 45.55 &            & 53.02 &          \\
lymph & 15.95 & 16.23 &           & 15.65 &           & 16.49 &            & 15.65 &          \\
molecular-biology-promoters &  9.85 &  9.95 &           &  8.56 &           & 14.00 &    $\circ$ &  8.56 &          \\
mushroom &  7.04 &  6.86 & $\bullet$ &  5.06 & $\bullet$ &  7.89 &    $\circ$ &  5.06 & $\bullet$\\
nursery & 16.07 & 16.06 &           & 15.87 & $\bullet$ & 16.51 &    $\circ$ & 15.87 & $\bullet$\\
optdigits & 14.06 & 14.05 &           & 13.10 & $\bullet$ & 16.32 &    $\circ$ & 13.10 & $\bullet$\\
page-blocks & 34.97 & 19.25 & $\bullet$ & 16.82 & $\bullet$ & 14.00 &  $\bullet$ & 16.82 & $\bullet$\\
pendigits & 23.29 & 22.49 & $\bullet$ & 21.21 & $\bullet$ & 26.14 &    $\circ$ & 21.21 & $\bullet$\\
postoperative-patient-data & 32.90 & 32.76 &           & 29.98 & $\bullet$ & 30.36 &  $\bullet$ & 29.98 & $\bullet$\\
primary-tumor & 55.36 & 55.52 &           & 54.13 & $\bullet$ & 54.94 &            & 54.13 & $\bullet$\\
segment & 30.05 & 30.75 &   $\circ$ & 28.12 & $\bullet$ & 33.04 &    $\circ$ & 28.12 & $\bullet$\\
shuttle-landing-control & 44.00 & 45.33 &           & 46.67 &           & 36.67 &  $\bullet$ & 46.67 &          \\
sick & 10.94 & 10.23 & $\bullet$ &  9.81 & $\bullet$ &  8.84 &  $\bullet$ &  9.81 & $\bullet$\\
solar-flare-2 & 28.40 & 28.21 &           & 28.16 &           & 28.39 &            & 28.16 &          \\
sonar & 26.37 & 26.26 &           & 24.14 & $\bullet$ & 25.33 &            & 24.14 & $\bullet$\\
soybean & 10.39 & 10.62 &           &  9.77 & $\bullet$ & 11.28 &    $\circ$ &  9.77 & $\bullet$\\
spambase & 25.44 & 24.85 & $\bullet$ & 24.51 & $\bullet$ & 24.14 &  $\bullet$ & 24.51 & $\bullet$\\
tae & 50.50 & 49.64 &           & 48.92 & $\bullet$ & 51.64 &            & 48.92 & $\bullet$\\
tic-tac-toe & 29.97 & 30.21 &           & 29.49 & $\bullet$ & 29.23 &  $\bullet$ & 29.49 & $\bullet$\\
trains & 34.00 & 33.00 &           & 31.00 &           & 39.00 &            & 31.00 &          \\
vehicle & 56.94 & 56.72 &           & 53.71 & $\bullet$ & 55.79 &  $\bullet$ & 53.71 & $\bullet$\\
vote &  9.57 &  9.57 &           &  9.47 & $\bullet$ &  9.34 &            &  9.47 & $\bullet$\\
vowel & 71.90 & 71.27 & $\bullet$ & 69.27 & $\bullet$ & 72.30 &            & 69.27 & $\bullet$\\
waveform-5000 & 22.66 & 22.63 &           & 22.44 & $\bullet$ & 23.63 &    $\circ$ & 22.44 & $\bullet$\\
zoo &  8.69 &  9.44 &   $\circ$ &  8.78 &           & 11.33 &    $\circ$ &  8.78 &          \\
\hline
\multicolumn{10}{c}{$\circ$, $\bullet$ statistically significant improvement or degradation}\\
\end{longtable} \footnotesize \par}
\newpage
{\centering \footnotesize \begin{longtable}{lrr@{\hspace{0.1cm}}cr@{\hspace{0.1cm}}cr@{\hspace{0.1cm}}cr@{\hspace{0.1cm}}c}
\caption{\label{nbmae3}Naive Bayes Mean Absolute Error - Three Hidden Variables}
\\
\hline
Dataset & (1)& (2) & & (3) & & (4) & & (5) & \\
\hline
audiology & 0.03 & 0.03 &   $\circ$ & 0.03 &    $\circ$ & 0.04 &   $\circ$ & 0.03 &    $\circ$\\
autos & 0.10 & 0.10 &           & 0.10 &            & 0.10 &           & 0.10 &           \\
balance-scale & 0.25 & 0.25 &   $\circ$ & 0.27 &    $\circ$ & 0.28 &   $\circ$ & 0.27 &    $\circ$\\
breast-cancer & 0.32 & 0.32 &           & 0.32 &    $\circ$ & 0.32 &           & 0.32 &    $\circ$\\
bridges-version1 & 0.14 & 0.14 &           & 0.15 &    $\circ$ & 0.15 &   $\circ$ & 0.15 &    $\circ$\\
car & 0.13 & 0.13 & $\bullet$ & 0.13 &    $\circ$ & 0.14 &   $\circ$ & 0.13 &    $\circ$\\
cmc & 0.38 & 0.38 & $\bullet$ & 0.38 &    $\circ$ & 0.39 &   $\circ$ & 0.38 &    $\circ$\\
colic & 0.24 & 0.24 & $\bullet$ & 0.23 &  $\bullet$ & 0.22 & $\bullet$ & 0.23 &  $\bullet$\\
cylinder-bands & 0.35 & 0.35 & $\bullet$ & 0.34 &  $\bullet$ & 0.35 &           & 0.34 &  $\bullet$\\
dermatology & 0.01 & 0.01 &           & 0.01 &            & 0.02 &   $\circ$ & 0.01 &           \\
diabetes & 0.33 & 0.33 & $\bullet$ & 0.33 &            & 0.33 &   $\circ$ & 0.33 &           \\
ecoli & 0.10 & 0.10 & $\bullet$ & 0.10 &    $\circ$ & 0.11 &   $\circ$ & 0.10 &    $\circ$\\
flags & 0.10 & 0.10 &           & 0.10 &    $\circ$ & 0.11 &   $\circ$ & 0.10 &    $\circ$\\
glass & 0.16 & 0.16 & $\bullet$ & 0.16 &            & 0.16 &   $\circ$ & 0.16 &           \\
haberman & 0.37 & 0.36 & $\bullet$ & 0.37 &    $\circ$ & 0.37 &           & 0.37 &    $\circ$\\
hayes-roth-train & 0.31 & 0.31 &           & 0.31 &            & 0.31 &           & 0.31 &           \\
heart-h & 0.09 & 0.09 &           & 0.09 &    $\circ$ & 0.10 &   $\circ$ & 0.09 &    $\circ$\\
heart-statlog & 0.29 & 0.29 &   $\circ$ & 0.30 &    $\circ$ & 0.31 &   $\circ$ & 0.30 &    $\circ$\\
hepatitis & 0.21 & 0.21 & $\bullet$ & 0.21 &    $\circ$ & 0.21 &           & 0.21 &    $\circ$\\
hypothyroid & 0.08 & 0.08 &   $\circ$ & 0.07 &  $\bullet$ & 0.07 & $\bullet$ & 0.07 &  $\bullet$\\
ionosphere & 0.31 & 0.31 & $\bullet$ & 0.26 &  $\bullet$ & 0.26 & $\bullet$ & 0.26 &  $\bullet$\\
iris & 0.18 & 0.17 & $\bullet$ & 0.18 &            & 0.21 &   $\circ$ & 0.18 &           \\
kr-vs-kp & 0.24 & 0.24 &           & 0.23 &  $\bullet$ & 0.25 &   $\circ$ & 0.23 &  $\bullet$\\
labor & 0.08 & 0.08 &           & 0.08 &            & 0.09 &   $\circ$ & 0.08 &           \\
letter & 0.06 & 0.05 & $\bullet$ & 0.05 &  $\bullet$ & 0.06 &   $\circ$ & 0.05 &  $\bullet$\\
liver-disorders & 0.46 & 0.46 &           & 0.46 &            & 0.46 &           & 0.46 &           \\
lung-cancer & 0.33 & 0.34 &           & 0.36 &    $\circ$ & 0.31 &           & 0.36 &    $\circ$\\
lymph & 0.10 & 0.10 &           & 0.10 &  $\bullet$ & 0.11 &           & 0.10 &  $\bullet$\\
molecular-biology-promoters & 0.12 & 0.12 &           & 0.11 &  $\bullet$ & 0.15 &   $\circ$ & 0.11 &  $\bullet$\\
mushroom & 0.07 & 0.07 & $\bullet$ & 0.05 &  $\bullet$ & 0.08 &   $\circ$ & 0.05 &  $\bullet$\\
nursery & 0.09 & 0.09 & $\bullet$ & 0.10 &    $\circ$ & 0.13 &   $\circ$ & 0.10 &    $\circ$\\
optdigits & 0.03 & 0.03 &           & 0.03 &  $\bullet$ & 0.04 &   $\circ$ & 0.03 &  $\bullet$\\
page-blocks & 0.12 & 0.09 & $\bullet$ & 0.08 &  $\bullet$ & 0.07 & $\bullet$ & 0.08 &  $\bullet$\\
pendigits & 0.05 & 0.05 & $\bullet$ & 0.05 &  $\bullet$ & 0.07 &   $\circ$ & 0.05 &  $\bullet$\\
postoperative-patient-data & 0.28 & 0.28 &           & 0.28 &            & 0.27 & $\bullet$ & 0.28 &           \\
primary-tumor & 0.06 & 0.06 & $\bullet$ & 0.06 &    $\circ$ & 0.06 &   $\circ$ & 0.06 &    $\circ$\\
segment & 0.10 & 0.10 &   $\circ$ & 0.09 &  $\bullet$ & 0.11 &   $\circ$ & 0.09 &  $\bullet$\\
shuttle-landing-control & 0.45 & 0.46 &           & 0.46 &            & 0.44 & $\bullet$ & 0.46 &           \\
sick & 0.12 & 0.12 & $\bullet$ & 0.12 &  $\bullet$ & 0.11 & $\bullet$ & 0.12 &  $\bullet$\\
solar-flare-2 & 0.11 & 0.11 & $\bullet$ & 0.11 &    $\circ$ & 0.12 &   $\circ$ & 0.11 &    $\circ$\\
sonar & 0.26 & 0.26 &           & 0.25 &  $\bullet$ & 0.26 &           & 0.25 &  $\bullet$\\
soybean & 0.01 & 0.01 &           & 0.01 &  $\bullet$ & 0.01 &   $\circ$ & 0.01 &  $\bullet$\\
spambase & 0.27 & 0.27 & $\bullet$ & 0.27 &  $\bullet$ & 0.28 &   $\circ$ & 0.27 &  $\bullet$\\
tae & 0.40 & 0.40 &           & 0.40 &    $\circ$ & 0.41 &   $\circ$ & 0.40 &    $\circ$\\
tic-tac-toe & 0.36 & 0.36 & $\bullet$ & 0.37 &    $\circ$ & 0.37 &   $\circ$ & 0.37 &    $\circ$\\
trains & 0.34 & 0.34 &           & 0.33 &            & 0.35 &           & 0.33 &           \\
vehicle & 0.29 & 0.29 &           & 0.28 &  $\bullet$ & 0.29 &           & 0.28 &  $\bullet$\\
vote & 0.10 & 0.10 &           & 0.10 &  $\bullet$ & 0.10 &           & 0.10 &  $\bullet$\\
vowel & 0.14 & 0.14 &           & 0.14 &  $\bullet$ & 0.15 &   $\circ$ & 0.14 &  $\bullet$\\
waveform-5000 & 0.16 & 0.16 & $\bullet$ & 0.16 &  $\bullet$ & 0.17 &   $\circ$ & 0.16 &  $\bullet$\\
zoo & 0.03 & 0.03 &   $\circ$ & 0.03 &    $\circ$ & 0.04 &   $\circ$ & 0.03 &    $\circ$\\
\hline
\multicolumn{10}{c}{$\circ$, $\bullet$ statistically significant improvement or degradation}\\
\end{longtable} \footnotesize \par}
\input{tables/naive_bayes_mean_absolute_error_hidden_five.tex}
\newpage
{\centering \footnotesize \begin{longtable}{lrr@{\hspace{0.1cm}}cr@{\hspace{0.1cm}}cr@{\hspace{0.1cm}}cr@{\hspace{0.1cm}}c}
\caption{\label{nbrmse3}Naive Bayes Root Mean Squared Error - Three Hidden Variables}
\\
\hline
Dataset & (1)& (2) & & (3) & & (4) & & (5) & \\
\hline
audiology & 0.16 & 0.16 &            & 0.16 & $\bullet$ & 0.17 &    $\circ$ & 0.16 & $\bullet$\\
autos & 0.28 & 0.28 &            & 0.28 &           & 0.27 &  $\bullet$ & 0.28 &          \\
balance-scale & 0.34 & 0.34 &    $\circ$ & 0.34 &   $\circ$ & 0.36 &    $\circ$ & 0.34 &   $\circ$\\
breast-cancer & 0.45 & 0.45 &            & 0.45 & $\bullet$ & 0.44 &  $\bullet$ & 0.45 & $\bullet$\\
bridges-version1 & 0.29 & 0.29 &            & 0.28 & $\bullet$ & 0.29 &    $\circ$ & 0.28 & $\bullet$\\
car & 0.24 & 0.24 &            & 0.25 &   $\circ$ & 0.26 &    $\circ$ & 0.25 &   $\circ$\\
cmc & 0.47 & 0.47 &            & 0.46 & $\bullet$ & 0.45 &  $\bullet$ & 0.46 & $\bullet$\\
colic & 0.44 & 0.43 &  $\bullet$ & 0.41 & $\bullet$ & 0.41 &  $\bullet$ & 0.41 & $\bullet$\\
cylinder-bands & 0.50 & 0.49 &  $\bullet$ & 0.47 & $\bullet$ & 0.47 &  $\bullet$ & 0.47 & $\bullet$\\
dermatology & 0.08 & 0.08 &            & 0.08 &           & 0.09 &    $\circ$ & 0.08 &          \\
diabetes & 0.44 & 0.44 &            & 0.43 & $\bullet$ & 0.42 &  $\bullet$ & 0.43 & $\bullet$\\
ecoli & 0.23 & 0.23 &            & 0.23 & $\bullet$ & 0.23 &            & 0.23 & $\bullet$\\
flags & 0.27 & 0.27 &            & 0.26 & $\bullet$ & 0.27 &            & 0.26 & $\bullet$\\
glass & 0.29 & 0.29 &  $\bullet$ & 0.29 & $\bullet$ & 0.29 &  $\bullet$ & 0.29 & $\bullet$\\
haberman & 0.44 & 0.44 &  $\bullet$ & 0.44 & $\bullet$ & 0.44 &            & 0.44 & $\bullet$\\
hayes-roth-train & 0.39 & 0.40 &            & 0.38 & $\bullet$ & 0.39 &            & 0.38 & $\bullet$\\
heart-h & 0.23 & 0.23 &            & 0.23 & $\bullet$ & 0.23 &            & 0.23 & $\bullet$\\
heart-statlog & 0.43 & 0.44 &    $\circ$ & 0.43 & $\bullet$ & 0.42 &  $\bullet$ & 0.43 & $\bullet$\\
hepatitis & 0.39 & 0.39 &            & 0.39 & $\bullet$ & 0.39 &  $\bullet$ & 0.39 & $\bullet$\\
hypothyroid & 0.20 & 0.22 &    $\circ$ & 0.18 & $\bullet$ & 0.18 &  $\bullet$ & 0.18 & $\bullet$\\
ionosphere & 0.50 & 0.50 &  $\bullet$ & 0.44 & $\bullet$ & 0.43 &  $\bullet$ & 0.44 & $\bullet$\\
iris & 0.32 & 0.32 &  $\bullet$ & 0.32 & $\bullet$ & 0.34 &    $\circ$ & 0.32 & $\bullet$\\
kr-vs-kp & 0.34 & 0.34 &    $\circ$ & 0.32 & $\bullet$ & 0.34 &            & 0.32 & $\bullet$\\
labor & 0.18 & 0.17 &            & 0.17 &           & 0.19 &    $\circ$ & 0.17 &          \\
letter & 0.18 & 0.17 &  $\bullet$ & 0.17 & $\bullet$ & 0.17 &  $\bullet$ & 0.17 & $\bullet$\\
liver-disorders & 0.49 & 0.49 &            & 0.48 & $\bullet$ & 0.48 &  $\bullet$ & 0.48 & $\bullet$\\
lung-cancer & 0.50 & 0.52 &            & 0.52 &   $\circ$ & 0.48 &  $\bullet$ & 0.52 &   $\circ$\\
lymph & 0.24 & 0.24 &            & 0.23 & $\bullet$ & 0.24 &  $\bullet$ & 0.23 & $\bullet$\\
molecular-biology-promoters & 0.25 & 0.26 &            & 0.24 & $\bullet$ & 0.31 &    $\circ$ & 0.24 & $\bullet$\\
mushroom & 0.25 & 0.25 &            & 0.20 & $\bullet$ & 0.26 &    $\circ$ & 0.20 & $\bullet$\\
nursery & 0.21 & 0.21 &            & 0.21 &   $\circ$ & 0.23 &    $\circ$ & 0.21 &   $\circ$\\
optdigits & 0.15 & 0.15 &    $\circ$ & 0.14 & $\bullet$ & 0.16 &    $\circ$ & 0.14 & $\bullet$\\
page-blocks & 0.28 & 0.24 &  $\bullet$ & 0.21 & $\bullet$ & 0.20 &  $\bullet$ & 0.21 & $\bullet$\\
pendigits & 0.19 & 0.18 &  $\bullet$ & 0.18 & $\bullet$ & 0.20 &    $\circ$ & 0.18 & $\bullet$\\
postoperative-patient-data & 0.40 & 0.40 &    $\circ$ & 0.39 & $\bullet$ & 0.39 &            & 0.39 & $\bullet$\\
primary-tumor & 0.18 & 0.18 &            & 0.18 & $\bullet$ & 0.18 &    $\circ$ & 0.18 & $\bullet$\\
segment & 0.25 & 0.25 &    $\circ$ & 0.24 & $\bullet$ & 0.26 &    $\circ$ & 0.24 & $\bullet$\\
shuttle-landing-control & 0.49 & 0.50 &            & 0.49 &           & 0.48 &  $\bullet$ & 0.49 &          \\
sick & 0.27 & 0.27 &  $\bullet$ & 0.26 & $\bullet$ & 0.25 &  $\bullet$ & 0.26 & $\bullet$\\
solar-flare-2 & 0.25 & 0.25 &    $\circ$ & 0.24 & $\bullet$ & 0.25 &    $\circ$ & 0.24 & $\bullet$\\
sonar & 0.46 & 0.46 &            & 0.44 & $\bullet$ & 0.45 &  $\bullet$ & 0.44 & $\bullet$\\
soybean & 0.10 & 0.10 &            & 0.09 & $\bullet$ & 0.10 &    $\circ$ & 0.09 & $\bullet$\\
spambase & 0.43 & 0.43 &  $\bullet$ & 0.42 & $\bullet$ & 0.42 &  $\bullet$ & 0.42 & $\bullet$\\
tae & 0.46 & 0.46 &            & 0.46 & $\bullet$ & 0.46 &            & 0.46 & $\bullet$\\
tic-tac-toe & 0.44 & 0.44 &            & 0.43 & $\bullet$ & 0.44 &            & 0.43 & $\bullet$\\
trains & 0.43 & 0.43 &            & 0.41 &           & 0.43 &            & 0.41 &          \\
vehicle & 0.46 & 0.46 &    $\circ$ & 0.44 & $\bullet$ & 0.44 &  $\bullet$ & 0.44 & $\bullet$\\
vote & 0.30 & 0.30 &            & 0.29 & $\bullet$ & 0.29 &  $\bullet$ & 0.29 & $\bullet$\\
vowel & 0.28 & 0.28 &    $\circ$ & 0.27 & $\bullet$ & 0.28 &  $\bullet$ & 0.27 & $\bullet$\\
waveform-5000 & 0.35 & 0.35 &  $\bullet$ & 0.35 & $\bullet$ & 0.35 &  $\bullet$ & 0.35 & $\bullet$\\
zoo & 0.12 & 0.13 &    $\circ$ & 0.12 &           & 0.14 &    $\circ$ & 0.12 &          \\
\hline
\multicolumn{10}{c}{$\circ$, $\bullet$ statistically significant improvement or degradation}\\
\end{longtable} \footnotesize \par}
\input{tables/naive_bayes_root_mean_squared_error_hidden_five.tex}
\newpage
{\centering \footnotesize \begin{longtable}{lrr@{\hspace{0.1cm}}cr@{\hspace{0.1cm}}cr@{\hspace{0.1cm}}cr@{\hspace{0.1cm}}c}
\caption{\label{nbmeg3}Naive Bayes Mean Entropy Gain - Three Hidden Variables}
\\
\hline
Dataset & (1)& (2) & & (3) & & (4) & & (5) & \\
\hline
audiology & -1.33 & -1.30 &    $\circ$ & -1.30 &           & -2.14 & $\bullet$ & -1.30 &          \\
autos & -0.12 & -0.13 &            & -0.04 &   $\circ$ &  0.11 &   $\circ$ & -0.04 &   $\circ$\\
balance-scale &  0.42 &  0.41 &            &  0.39 & $\bullet$ &  0.32 & $\bullet$ &  0.39 & $\bullet$\\
breast-cancer & -0.06 & -0.07 &  $\bullet$ & -0.02 &   $\circ$ & -0.00 &   $\circ$ & -0.02 &   $\circ$\\
bridges-version1 &  0.67 &  0.67 &            &  0.75 &   $\circ$ &  0.67 &           &  0.75 &   $\circ$\\
car &  0.65 &  0.65 &            &  0.63 & $\bullet$ &  0.57 & $\bullet$ &  0.63 & $\bullet$\\
cmc & -0.07 & -0.08 &  $\bullet$ &  0.04 &   $\circ$ &  0.04 &   $\circ$ &  0.04 &   $\circ$\\
colic & -0.44 & -0.44 &    $\circ$ & -0.22 &   $\circ$ & -0.17 &   $\circ$ & -0.22 &   $\circ$\\
cylinder-bands & -0.17 & -0.16 &    $\circ$ &  0.02 &   $\circ$ &  0.04 &   $\circ$ &  0.02 &   $\circ$\\
dermatology &  2.32 &  2.32 &            &  2.33 &           &  2.29 & $\bullet$ &  2.33 &          \\
diabetes &  0.12 &  0.12 &            &  0.16 &   $\circ$ &  0.16 &   $\circ$ &  0.16 &   $\circ$\\
ecoli &  0.96 &  0.96 &            &  1.00 &   $\circ$ &  0.96 &           &  1.00 &   $\circ$\\
flags &  0.12 &  0.12 &            &  0.31 &   $\circ$ &  0.27 &   $\circ$ &  0.31 &   $\circ$\\
glass &  0.47 &  0.48 &            &  0.55 &   $\circ$ &  0.53 &   $\circ$ &  0.55 &   $\circ$\\
haberman & -0.01 & -0.01 &    $\circ$ &  0.01 &   $\circ$ & -0.01 &           &  0.01 &   $\circ$\\
hayes-roth-train &  0.08 &  0.07 &            &  0.14 &   $\circ$ &  0.10 &           &  0.14 &   $\circ$\\
heart-h &  0.26 &  0.25 &            &  0.31 &   $\circ$ &  0.30 &   $\circ$ &  0.31 &   $\circ$\\
heart-statlog &  0.15 &  0.14 &  $\bullet$ &  0.20 &   $\circ$ &  0.22 &   $\circ$ &  0.20 &   $\circ$\\
hepatitis & -0.21 & -0.21 &            & -0.14 &   $\circ$ & -0.09 &   $\circ$ & -0.14 &   $\circ$\\
hypothyroid &  0.04 & -0.12 &  $\bullet$ &  0.10 &   $\circ$ &  0.09 &   $\circ$ &  0.10 &   $\circ$\\
ionosphere & -0.68 & -0.65 &    $\circ$ & -0.18 &   $\circ$ & -0.08 &   $\circ$ & -0.18 &   $\circ$\\
iris &  0.88 &  0.89 &    $\circ$ &  0.89 &   $\circ$ &  0.76 & $\bullet$ &  0.89 &   $\circ$\\
kr-vs-kp &  0.49 &  0.49 &  $\bullet$ &  0.53 &   $\circ$ &  0.48 & $\bullet$ &  0.53 &   $\circ$\\
labor &  0.70 &  0.72 &    $\circ$ &  0.71 &           &  0.70 &           &  0.71 &          \\
letter &  1.49 &  1.49 &    $\circ$ &  1.75 &   $\circ$ &  1.58 &   $\circ$ &  1.75 &   $\circ$\\
liver-disorders &  0.01 &  0.01 &            &  0.06 &   $\circ$ &  0.03 &   $\circ$ &  0.06 &   $\circ$\\
lung-cancer & -1.66 & -1.55 &            & -1.40 &   $\circ$ & -1.21 &   $\circ$ & -1.40 &   $\circ$\\
lymph &  0.62 &  0.62 &            &  0.70 &   $\circ$ &  0.72 &   $\circ$ &  0.70 &   $\circ$\\
molecular-biology-promoters &  0.62 &  0.61 &            &  0.66 &   $\circ$ &  0.43 & $\bullet$ &  0.66 &   $\circ$\\
mushroom &  0.52 &  0.51 &            &  0.76 &   $\circ$ &  0.57 &   $\circ$ &  0.76 &   $\circ$\\
nursery &  1.20 &  1.20 &            &  1.19 & $\bullet$ &  1.00 & $\bullet$ &  1.19 & $\bullet$\\
optdigits &  2.37 &  2.36 &  $\bullet$ &  2.46 &   $\circ$ &  2.37 &           &  2.46 &   $\circ$\\
page-blocks & -0.30 & -0.11 &    $\circ$ &  0.07 &   $\circ$ &  0.11 &   $\circ$ &  0.07 &   $\circ$\\
pendigits &  2.05 &  2.06 &    $\circ$ &  2.13 &   $\circ$ &  2.00 & $\bullet$ &  2.13 &   $\circ$\\
postoperative-patient-data & -0.18 & -0.20 &  $\bullet$ & -0.14 &   $\circ$ & -0.17 &           & -0.14 &   $\circ$\\
primary-tumor &  0.73 &  0.74 &            &  0.84 &   $\circ$ &  0.71 &           &  0.84 &   $\circ$\\
segment &  1.17 &  1.16 &  $\bullet$ &  1.32 &   $\circ$ &  1.25 &   $\circ$ &  1.32 &   $\circ$\\
shuttle-landing-control &  0.02 & -0.01 &            &  0.00 &           &  0.04 &   $\circ$ &  0.00 &          \\
sick & -0.02 & -0.01 &    $\circ$ &  0.02 &   $\circ$ &  0.04 &   $\circ$ &  0.02 &   $\circ$\\
solar-flare-2 &  1.35 &  1.35 &            &  1.40 &   $\circ$ &  1.28 & $\bullet$ &  1.40 &   $\circ$\\
sonar & -0.59 & -0.59 &            & -0.37 &   $\circ$ & -0.33 &   $\circ$ & -0.37 &   $\circ$\\
soybean &  3.14 &  3.13 &  $\bullet$ &  3.25 &   $\circ$ &  3.23 &   $\circ$ &  3.25 &   $\circ$\\
spambase &  0.06 &  0.06 &    $\circ$ &  0.14 &   $\circ$ &  0.17 &   $\circ$ &  0.14 &   $\circ$\\
tae &  0.00 &  0.01 &            &  0.05 &   $\circ$ &  0.03 &   $\circ$ &  0.05 &   $\circ$\\
tic-tac-toe &  0.13 &  0.13 &            &  0.14 &   $\circ$ &  0.12 & $\bullet$ &  0.14 &   $\circ$\\
trains & -0.07 & -0.06 &            &  0.07 &   $\circ$ &  0.01 &           &  0.07 &   $\circ$\\
vehicle & -1.53 & -1.54 &  $\bullet$ & -1.00 &   $\circ$ & -0.68 &   $\circ$ & -1.00 &   $\circ$\\
vote &  0.08 &  0.08 &            &  0.18 &   $\circ$ &  0.28 &   $\circ$ &  0.18 &   $\circ$\\
vowel &  0.60 &  0.57 &  $\bullet$ &  0.85 &   $\circ$ &  0.63 &   $\circ$ &  0.85 &   $\circ$\\
waveform-5000 &  0.46 &  0.46 &    $\circ$ &  0.54 &   $\circ$ &  0.60 &   $\circ$ &  0.54 &   $\circ$\\
zoo &  2.19 &  2.15 &  $\bullet$ &  2.18 &           &  2.06 & $\bullet$ &  2.18 &          \\
\hline
\multicolumn{10}{c}{$\circ$, $\bullet$ statistically significant improvement or degradation}\\
\end{longtable} \footnotesize \par}
\newpage
{\centering \footnotesize \begin{longtable}{lrr@{\hspace{0.1cm}}cr@{\hspace{0.1cm}}cr@{\hspace{0.1cm}}cr@{\hspace{0.1cm}}c}
\caption{\label{nbmeg5}Naive Bayes Mean Entropy Gain - Five Hidden Variables}
\\
\hline
Dataset & (1)& (2) & & (3) & & (4) & & (5) & \\
\hline
audiology & -1.36 & -1.30 &   $\circ$ & -1.30 &   $\circ$ & -2.14 &  $\bullet$ & -1.30 &   $\circ$\\
autos & -0.13 & -0.13 &           & -0.04 &   $\circ$ &  0.11 &    $\circ$ & -0.04 &   $\circ$\\
balance-scale &  0.42 &  0.42 &           &  0.39 & $\bullet$ &  0.32 &  $\bullet$ &  0.39 & $\bullet$\\
breast-cancer & -0.07 & -0.07 &           & -0.02 &   $\circ$ & -0.00 &    $\circ$ & -0.02 &   $\circ$\\
bridges-version1 &  0.67 &  0.67 &           &  0.75 &   $\circ$ &  0.67 &            &  0.75 &   $\circ$\\
car &  0.64 &  0.65 &           &  0.63 & $\bullet$ &  0.57 &  $\bullet$ &  0.63 & $\bullet$\\
cmc & -0.07 & -0.08 & $\bullet$ &  0.04 &   $\circ$ &  0.04 &    $\circ$ &  0.04 &   $\circ$\\
colic & -0.45 & -0.43 &   $\circ$ & -0.22 &   $\circ$ & -0.17 &    $\circ$ & -0.22 &   $\circ$\\
cylinder-bands & -0.18 & -0.16 &   $\circ$ &  0.02 &   $\circ$ &  0.04 &    $\circ$ &  0.02 &   $\circ$\\
dermatology &  2.32 &  2.32 &           &  2.33 &   $\circ$ &  2.29 &  $\bullet$ &  2.33 &   $\circ$\\
diabetes &  0.12 &  0.12 &           &  0.16 &   $\circ$ &  0.16 &    $\circ$ &  0.16 &   $\circ$\\
ecoli &  0.97 &  0.96 &           &  1.00 &   $\circ$ &  0.96 &            &  1.00 &   $\circ$\\
flags &  0.12 &  0.12 &           &  0.31 &   $\circ$ &  0.27 &    $\circ$ &  0.31 &   $\circ$\\
glass &  0.45 &  0.48 &   $\circ$ &  0.55 &   $\circ$ &  0.53 &    $\circ$ &  0.55 &   $\circ$\\
haberman & -0.01 & -0.01 &           &  0.01 &   $\circ$ & -0.01 &            &  0.01 &   $\circ$\\
hayes-roth-train &  0.07 &  0.08 &           &  0.14 &   $\circ$ &  0.10 &    $\circ$ &  0.14 &   $\circ$\\
heart-h &  0.26 &  0.25 & $\bullet$ &  0.31 &   $\circ$ &  0.30 &    $\circ$ &  0.31 &   $\circ$\\
heart-statlog &  0.14 &  0.14 &           &  0.20 &   $\circ$ &  0.22 &    $\circ$ &  0.20 &   $\circ$\\
hepatitis & -0.21 & -0.21 &           & -0.14 &   $\circ$ & -0.09 &    $\circ$ & -0.14 &   $\circ$\\
hypothyroid & -0.10 & -0.11 &           &  0.10 &   $\circ$ &  0.09 &    $\circ$ &  0.10 &   $\circ$\\
ionosphere & -0.75 & -0.65 &   $\circ$ & -0.18 &   $\circ$ & -0.08 &    $\circ$ & -0.18 &   $\circ$\\
iris &  0.87 &  0.89 &   $\circ$ &  0.89 &   $\circ$ &  0.76 &  $\bullet$ &  0.89 &   $\circ$\\
kr-vs-kp &  0.49 &  0.49 &           &  0.53 &   $\circ$ &  0.48 &            &  0.53 &   $\circ$\\
labor &  0.71 &  0.71 &           &  0.71 &           &  0.70 &            &  0.71 &          \\
letter &  1.47 &  1.49 &   $\circ$ &  1.75 &   $\circ$ &  1.58 &    $\circ$ &  1.75 &   $\circ$\\
liver-disorders &  0.01 &  0.01 &           &  0.06 &   $\circ$ &  0.03 &    $\circ$ &  0.06 &   $\circ$\\
lung-cancer & -1.63 & -1.54 &           & -1.40 &           & -1.21 &    $\circ$ & -1.40 &          \\
lymph &  0.64 &  0.61 &           &  0.70 &   $\circ$ &  0.72 &    $\circ$ &  0.70 &   $\circ$\\
molecular-biology-promoters &  0.60 &  0.59 &           &  0.66 &   $\circ$ &  0.43 &  $\bullet$ &  0.66 &   $\circ$\\
mushroom &  0.50 &  0.51 &   $\circ$ &  0.76 &   $\circ$ &  0.57 &    $\circ$ &  0.76 &   $\circ$\\
nursery &  1.20 &  1.20 &           &  1.19 & $\bullet$ &  1.00 &  $\bullet$ &  1.19 & $\bullet$\\
optdigits &  2.37 &  2.36 & $\bullet$ &  2.46 &   $\circ$ &  2.37 &            &  2.46 &   $\circ$\\
page-blocks & -0.45 & -0.11 &   $\circ$ &  0.07 &   $\circ$ &  0.11 &    $\circ$ &  0.07 &   $\circ$\\
pendigits &  2.05 &  2.06 &   $\circ$ &  2.13 &   $\circ$ &  2.00 &  $\bullet$ &  2.13 &   $\circ$\\
postoperative-patient-data & -0.19 & -0.19 &           & -0.14 &   $\circ$ & -0.17 &            & -0.14 &   $\circ$\\
primary-tumor &  0.73 &  0.74 &           &  0.84 &   $\circ$ &  0.71 &            &  0.84 &   $\circ$\\
segment &  1.17 &  1.16 & $\bullet$ &  1.32 &   $\circ$ &  1.25 &    $\circ$ &  1.32 &   $\circ$\\
shuttle-landing-control &  0.01 &  0.00 &           &  0.00 &           &  0.04 &    $\circ$ &  0.00 &          \\
sick & -0.02 & -0.01 &   $\circ$ &  0.02 &   $\circ$ &  0.04 &    $\circ$ &  0.02 &   $\circ$\\
solar-flare-2 &  1.35 &  1.35 &           &  1.40 &   $\circ$ &  1.28 &  $\bullet$ &  1.40 &   $\circ$\\
sonar & -0.58 & -0.60 & $\bullet$ & -0.37 &   $\circ$ & -0.33 &    $\circ$ & -0.37 &   $\circ$\\
soybean &  3.13 &  3.13 &           &  3.25 &   $\circ$ &  3.23 &    $\circ$ &  3.25 &   $\circ$\\
spambase &  0.06 &  0.06 &   $\circ$ &  0.14 &   $\circ$ &  0.17 &    $\circ$ &  0.14 &   $\circ$\\
tae &  0.01 &  0.01 &           &  0.05 &   $\circ$ &  0.03 &            &  0.05 &   $\circ$\\
tic-tac-toe &  0.12 &  0.13 &           &  0.14 &   $\circ$ &  0.12 &            &  0.14 &   $\circ$\\
trains & -0.05 & -0.08 &           &  0.07 &           &  0.01 &            &  0.07 &          \\
vehicle & -1.54 & -1.54 &           & -1.00 &   $\circ$ & -0.68 &    $\circ$ & -1.00 &   $\circ$\\
vote &  0.08 &  0.08 &           &  0.18 &   $\circ$ &  0.28 &    $\circ$ &  0.18 &   $\circ$\\
vowel &  0.59 &  0.57 & $\bullet$ &  0.85 &   $\circ$ &  0.63 &    $\circ$ &  0.85 &   $\circ$\\
waveform-5000 &  0.45 &  0.46 &   $\circ$ &  0.54 &   $\circ$ &  0.60 &    $\circ$ &  0.54 &   $\circ$\\
zoo &  2.20 &  2.16 & $\bullet$ &  2.18 &           &  2.06 &  $\bullet$ &  2.18 &          \\
\hline
\multicolumn{10}{c}{$\circ$, $\bullet$ statistically significant improvement or degradation}\\
\end{longtable} \footnotesize \par}
\linespread{1.3}


\subsection{Bayesian Network}

The results for a TAN Bayesian Network with various numbers of hidden variables can be seen in figures \ref{bnpi3} - \ref{bnmeg5}.

The Bayesian Network Percent Incorrect results can be seen in figures \ref{bnpi}, \ref{bnpi3} and \ref{bnpi5}. With zero hidden variables, Case 1 performed slightly worse than Case 4, and roughly as well as Case 5. Once the number of hidden variables was increased to three, Case 1 started to perform much worse than both Cases 4 and 5. This trend continued when five hidden variables were added. Interestingly the performance of Case 1 relative to Case 3 was reasonably stable throughout. The number of tests in which they performed better or worse than each other was approximately the same in all three experiments, with Case 3 performing better in roughly twice as many tests.

The Bayesian Network MAE results (figures \ref{bnmae}, \ref{bnmae3} and \ref{bnmae5}) again show Case 1 and Case 3 performing roughly as well as each other with all numbers of hidden variables. Case 1 does perform slightly better than Case 3, but not significantly so. The most interesting part of these results is the comparison between Case 1 and Case 5. With zero hidden variables Case 1 performs better in 44/51 tests, but with three and five hidden variables this drops to 22/51 tests.

The Bayesian Network RMSE results can be seen in figures \ref{bnrmse}, \ref{bnrmse3} and \ref{bnrmse5}. These figures show Case 1 performing significantly worse as hidden variables are added, especially relative to the control cases. Case 1 again performs approximately the same relative to Case 3 at any number of hidden variables, with Case 3 seeming to be roughly 2.5 times better.

The Bayesian Network MEG results are shown in figures \ref{bnmeg}, \ref{bnmeg3} and \ref{bnmeg5}. Adding hidden variables again caused large entropy reductions, and adding more hidden variables led to larger reductions.

\linespread{1.0}
\begin{table}[thb]
\caption{\label{bnpi3}Bayesian Network Percent Incorrect - Three Hidden Variables}
\footnotesize
{\centering \begin{tabular}{lrr@{\hspace{0.1cm}}cr@{\hspace{0.1cm}}cr@{\hspace{0.1cm}}cr@{\hspace{0.1cm}}c}
\\
\hline
Dataset & (1)& (2) & & (3) & & (4) & & (5) & \\
\hline
audiology & 37.68 & 38.23 &           & 41.04 &   $\circ$ & 50.09 &   $\circ$ & 37.87 &          \\
autos &  9.28 &  9.34 &           & 10.83 &   $\circ$ &  9.62 &           &  9.67 &          \\
balance-scale & 29.40 & 29.61 &           & 28.49 & $\bullet$ & 28.49 & $\bullet$ & 30.05 &          \\
breast-cancer & 27.52 & 27.36 &           & 29.46 &   $\circ$ & 27.41 &           & 27.40 &          \\
bridges-version1 & 34.66 & 36.77 &   $\circ$ & 38.89 &   $\circ$ & 36.00 &           & 36.91 &   $\circ$\\
car & 21.16 & 21.41 &           & 15.41 & $\bullet$ & 23.23 &   $\circ$ & 21.24 &          \\
cmc & 52.46 & 52.96 &   $\circ$ & 49.26 & $\bullet$ & 50.24 & $\bullet$ & 50.61 & $\bullet$\\
colic & 21.11 & 21.42 &           & 19.84 & $\bullet$ & 19.94 & $\bullet$ & 20.03 & $\bullet$\\
cylinder-bands & 33.71 & 33.73 &           & 28.38 & $\bullet$ & 32.69 & $\bullet$ & 33.20 &          \\
dermatology &  4.69 &  4.75 &           &  6.10 &   $\circ$ &  3.58 & $\bullet$ &  3.43 & $\bullet$\\
diabetes & 29.30 & 29.49 &           & 27.28 & $\bullet$ & 27.52 & $\bullet$ & 27.94 & $\bullet$\\
ecoli & 35.47 & 34.66 & $\bullet$ & 33.03 & $\bullet$ & 34.53 & $\bullet$ & 34.14 & $\bullet$\\
flags & 37.56 & 38.86 &   $\circ$ & 42.66 &   $\circ$ & 40.51 &   $\circ$ & 38.81 &          \\
glass & 49.56 & 49.12 &           & 42.56 & $\bullet$ & 46.58 & $\bullet$ & 47.03 & $\bullet$\\
haberman & 29.48 & 30.00 &           & 31.03 &   $\circ$ & 27.27 & $\bullet$ & 28.48 & $\bullet$\\
hayes-roth-train & 54.88 & 55.59 &           & 46.97 & $\bullet$ & 49.23 & $\bullet$ & 49.30 & $\bullet$\\
heart-h & 18.81 & 18.47 &           & 20.18 &   $\circ$ & 19.23 &           & 17.79 & $\bullet$\\
heart-statlog & 26.90 & 26.74 &           & 26.62 &           & 27.51 &           & 26.02 &          \\
hepatitis & 18.84 & 18.63 &           & 20.20 &           & 19.20 &           & 19.13 &          \\
hypothyroid & 10.59 & 10.84 &           &  8.15 & $\bullet$ &  8.38 & $\bullet$ &  8.61 & $\bullet$\\
ionosphere & 30.67 & 32.00 &   $\circ$ & 19.62 & $\bullet$ & 23.94 & $\bullet$ & 23.66 & $\bullet$\\
iris & 26.30 & 26.37 &           & 19.66 & $\bullet$ & 23.54 & $\bullet$ & 21.27 & $\bullet$\\
kr-vs-kp & 17.31 & 17.43 &           & 11.73 & $\bullet$ & 16.69 & $\bullet$ & 15.22 & $\bullet$\\
labor &  5.60 &  5.20 &           &  8.00 &   $\circ$ &  5.20 &           &  5.00 &          \\
letter & 63.01 & 63.20 &   $\circ$ & 56.25 & $\bullet$ & 62.59 & $\bullet$ & 61.77 & $\bullet$\\
liver-disorders & 40.60 & 41.09 &           & 37.99 & $\bullet$ & 39.17 &           & 38.07 & $\bullet$\\
lung-cancer & 50.23 & 53.10 &           & 57.43 &   $\circ$ & 45.55 &           & 55.90 &   $\circ$\\
lymph & 16.07 & 16.96 &           & 16.14 &           & 16.49 &           & 16.37 &          \\
molecular-biology-promoters & 11.88 & 12.59 &           & 23.31 &   $\circ$ & 14.00 &   $\circ$ & 11.11 &          \\
mushroom &  6.90 &  7.11 &   $\circ$ &  0.57 & $\bullet$ &  7.89 &   $\circ$ &  4.16 & $\bullet$\\
nursery & 18.62 & 19.07 &   $\circ$ & 15.16 & $\bullet$ & 16.51 & $\bullet$ & 16.47 & $\bullet$\\
optdigits & 17.39 & 17.55 &   $\circ$ & 10.67 & $\bullet$ & 16.32 & $\bullet$ & 15.19 & $\bullet$\\
page-blocks & 17.54 & 23.49 &   $\circ$ &  9.23 & $\bullet$ & 14.00 & $\bullet$ & 14.18 & $\bullet$\\
pendigits & 29.71 & 29.30 & $\bullet$ & 18.20 & $\bullet$ & 26.14 & $\bullet$ & 24.65 & $\bullet$\\
postoperative-patient-data & 31.30 & 32.04 &           & 33.61 &   $\circ$ & 30.36 &           & 30.93 &          \\
primary-tumor & 55.52 & 55.68 &           & 57.97 &   $\circ$ & 54.94 &           & 54.68 & $\bullet$\\
segment & 38.36 & 39.05 &           & 26.16 & $\bullet$ & 33.04 & $\bullet$ & 30.88 & $\bullet$\\
shuttle-landing-control & 46.67 & 44.67 &           & 40.00 & $\bullet$ & 36.67 & $\bullet$ & 44.00 &          \\
sick &  9.19 &  9.28 &           &  7.47 & $\bullet$ &  8.84 & $\bullet$ &  9.52 &   $\circ$\\
solar-flare-2 & 28.68 & 28.72 &           & 28.22 &           & 28.39 &           & 28.41 &          \\
sonar & 25.39 & 25.86 &           & 23.43 & $\bullet$ & 25.33 &           & 24.09 & $\bullet$\\
soybean & 11.09 & 11.42 &           & 11.23 &           & 11.28 &           &  9.46 & $\bullet$\\
spambase & 26.17 & 25.86 &           & 22.76 & $\bullet$ & 24.14 & $\bullet$ & 25.17 & $\bullet$\\
tae & 52.15 & 51.85 &           & 53.69 &           & 51.64 &           & 49.43 & $\bullet$\\
tic-tac-toe & 30.31 & 30.37 &           & 27.29 & $\bullet$ & 29.23 & $\bullet$ & 30.08 &          \\
trains & 36.00 & 32.00 &           & 20.00 & $\bullet$ & 39.00 &           & 37.00 &          \\
vehicle & 57.05 & 57.32 &           & 42.62 & $\bullet$ & 55.79 & $\bullet$ & 53.53 & $\bullet$\\
vote &  9.59 &  9.57 &           &  6.78 & $\bullet$ &  9.34 & $\bullet$ &  9.80 &   $\circ$\\
vowel & 72.66 & 73.35 &           & 63.98 & $\bullet$ & 72.30 &           & 71.36 & $\bullet$\\
waveform-5000 & 23.91 & 24.08 &           & 24.25 &           & 23.63 &           & 23.33 & $\bullet$\\
zoo & 12.06 & 13.06 &           & 13.50 &           & 11.33 &           & 11.99 &          \\
\hline
\multicolumn{10}{c}{$\circ$, $\bullet$ statistically significant improvement or degradation}\\
\end{tabular} \footnotesize \par}
\end{table}
\input{tables/bayes_net_percent_incorrect_hidden_five.tex}
\input{tables/bayes_net_mean_absolute_error_hidden_three.tex}
\newpage
{\centering \footnotesize \begin{longtable}{lrr@{\hspace{0.1cm}}cr@{\hspace{0.1cm}}cr@{\hspace{0.1cm}}cr@{\hspace{0.1cm}}c}
\caption{\label{bnmae5}Bayesian Network Mean Absolute Error - Five Hidden Variables}
\\
\hline
Dataset & (1)& (2) & & (3) & & (4) & & (5) & \\
\hline
audiology & 0.03 & 0.03 &           & 0.04 &   $\circ$ & 0.04 &   $\circ$ & 0.03 &    $\circ$\\
autos & 0.10 & 0.10 &           & 0.11 &   $\circ$ & 0.10 &           & 0.10 &           \\
balance-scale & 0.26 & 0.26 &           & 0.26 & $\bullet$ & 0.28 &   $\circ$ & 0.27 &    $\circ$\\
breast-cancer & 0.32 & 0.32 &           & 0.35 &   $\circ$ & 0.32 &           & 0.33 &    $\circ$\\
bridges-version1 & 0.14 & 0.15 &   $\circ$ & 0.16 &   $\circ$ & 0.15 &   $\circ$ & 0.15 &    $\circ$\\
car & 0.12 & 0.12 &           & 0.12 &           & 0.14 &   $\circ$ & 0.14 &    $\circ$\\
cmc & 0.38 & 0.38 &           & 0.38 &   $\circ$ & 0.39 &   $\circ$ & 0.39 &    $\circ$\\
colic & 0.23 & 0.23 &           & 0.23 &           & 0.22 & $\bullet$ & 0.22 &  $\bullet$\\
cylinder-bands & 0.35 & 0.35 &   $\circ$ & 0.32 & $\bullet$ & 0.35 & $\bullet$ & 0.34 &  $\bullet$\\
dermatology & 0.02 & 0.02 &           & 0.03 &   $\circ$ & 0.02 & $\bullet$ & 0.02 &  $\bullet$\\
diabetes & 0.33 & 0.33 &           & 0.34 &   $\circ$ & 0.33 &           & 0.33 &           \\
ecoli & 0.10 & 0.10 &           & 0.11 &   $\circ$ & 0.11 &   $\circ$ & 0.10 &    $\circ$\\
flags & 0.10 & 0.10 &   $\circ$ & 0.11 &   $\circ$ & 0.11 &   $\circ$ & 0.10 &    $\circ$\\
glass & 0.16 & 0.16 &           & 0.16 &           & 0.16 &   $\circ$ & 0.16 &    $\circ$\\
haberman & 0.37 & 0.37 &           & 0.38 &   $\circ$ & 0.37 &           & 0.38 &    $\circ$\\
hayes-roth-train & 0.31 & 0.31 &           & 0.30 & $\bullet$ & 0.31 &           & 0.31 &  $\bullet$\\
heart-h & 0.09 & 0.09 &           & 0.11 &   $\circ$ & 0.10 &   $\circ$ & 0.09 &    $\circ$\\
heart-statlog & 0.30 & 0.30 &           & 0.33 &   $\circ$ & 0.31 &   $\circ$ & 0.31 &    $\circ$\\
hepatitis & 0.20 & 0.20 &           & 0.23 &   $\circ$ & 0.21 &   $\circ$ & 0.21 &    $\circ$\\
hypothyroid & 0.07 & 0.07 &           & 0.06 & $\bullet$ & 0.07 & $\bullet$ & 0.07 &  $\bullet$\\
ionosphere & 0.31 & 0.32 &   $\circ$ & 0.22 & $\bullet$ & 0.26 & $\bullet$ & 0.26 &  $\bullet$\\
iris & 0.20 & 0.20 &   $\circ$ & 0.22 &   $\circ$ & 0.21 &   $\circ$ & 0.19 &           \\
kr-vs-kp & 0.24 & 0.24 &           & 0.18 & $\bullet$ & 0.25 &   $\circ$ & 0.23 &  $\bullet$\\
labor & 0.08 & 0.08 &           & 0.11 &   $\circ$ & 0.09 &   $\circ$ & 0.08 &           \\
letter & 0.06 & 0.06 &           & 0.05 & $\bullet$ & 0.06 &   $\circ$ & 0.06 &    $\circ$\\
liver-disorders & 0.46 & 0.46 &           & 0.44 & $\bullet$ & 0.46 &           & 0.46 &           \\
lung-cancer & 0.33 & 0.35 &           & 0.39 &   $\circ$ & 0.31 &           & 0.37 &    $\circ$\\
lymph & 0.10 & 0.10 &           & 0.10 &           & 0.11 &   $\circ$ & 0.10 &           \\
molecular-biology-promoters & 0.14 & 0.14 &           & 0.24 &   $\circ$ & 0.15 &   $\circ$ & 0.13 &           \\
mushroom & 0.07 & 0.07 &   $\circ$ & 0.01 & $\bullet$ & 0.08 &   $\circ$ & 0.04 &  $\bullet$\\
nursery & 0.11 & 0.11 &           & 0.10 & $\bullet$ & 0.13 &   $\circ$ & 0.11 &           \\
optdigits & 0.04 & 0.04 &   $\circ$ & 0.03 & $\bullet$ & 0.04 &   $\circ$ & 0.03 &  $\bullet$\\
page-blocks & 0.09 & 0.10 &   $\circ$ & 0.06 & $\bullet$ & 0.07 & $\bullet$ & 0.08 &  $\bullet$\\
pendigits & 0.07 & 0.07 &           & 0.05 & $\bullet$ & 0.07 &   $\circ$ & 0.06 &  $\bullet$\\
postoperative-patient-data & 0.28 & 0.28 &           & 0.28 &           & 0.27 & $\bullet$ & 0.28 &           \\
primary-tumor & 0.06 & 0.06 &           & 0.06 &   $\circ$ & 0.06 &   $\circ$ & 0.06 &    $\circ$\\
segment & 0.12 & 0.12 &           & 0.10 & $\bullet$ & 0.11 & $\bullet$ & 0.10 &  $\bullet$\\
shuttle-landing-control & 0.44 & 0.43 &           & 0.45 &   $\circ$ & 0.44 &   $\circ$ & 0.43 &  $\bullet$\\
sick & 0.11 & 0.11 &           & 0.10 & $\bullet$ & 0.11 &   $\circ$ & 0.12 &    $\circ$\\
solar-flare-2 & 0.11 & 0.11 &   $\circ$ & 0.12 &   $\circ$ & 0.12 &   $\circ$ & 0.11 &    $\circ$\\
sonar & 0.26 & 0.26 &           & 0.27 &           & 0.26 &           & 0.26 &           \\
soybean & 0.01 & 0.01 &           & 0.01 &   $\circ$ & 0.01 &   $\circ$ & 0.01 &  $\bullet$\\
spambase & 0.28 & 0.27 &           & 0.30 &   $\circ$ & 0.28 &   $\circ$ & 0.28 &    $\circ$\\
tae & 0.40 & 0.40 &           & 0.42 &   $\circ$ & 0.41 &   $\circ$ & 0.40 &    $\circ$\\
tic-tac-toe & 0.36 & 0.36 &           & 0.36 &           & 0.37 &   $\circ$ & 0.38 &    $\circ$\\
trains & 0.37 & 0.34 &           & 0.23 & $\bullet$ & 0.35 &           & 0.34 &           \\
vehicle & 0.29 & 0.29 &           & 0.25 & $\bullet$ & 0.29 & $\bullet$ & 0.28 &  $\bullet$\\
vote & 0.10 & 0.10 &           & 0.08 & $\bullet$ & 0.10 &   $\circ$ & 0.10 &           \\
vowel & 0.14 & 0.14 &           & 0.13 & $\bullet$ & 0.15 &   $\circ$ & 0.14 &    $\circ$\\
waveform-5000 & 0.17 & 0.17 &           & 0.19 &   $\circ$ & 0.17 &   $\circ$ & 0.17 &  $\bullet$\\
zoo & 0.04 & 0.04 &   $\circ$ & 0.05 &   $\circ$ & 0.04 &           & 0.04 &           \\
\hline
\multicolumn{10}{c}{$\circ$, $\bullet$ statistically significant improvement or degradation}\\
\end{longtable} \footnotesize \par}

\newpage
{\centering \footnotesize \begin{longtable}{lrr@{\hspace{0.1cm}}cr@{\hspace{0.1cm}}cr@{\hspace{0.1cm}}cr@{\hspace{0.1cm}}c}
\caption{\label{bnrmse3}Bayesian Network Root Mean Squared Error - Three Hidden Variables}
\\
\hline
Dataset & (1)& (2) & & (3) & & (4) & & (5) & \\
\hline
audiology & 0.16 & 0.16 &           & 0.16 &   $\circ$ & 0.17 &   $\circ$ & 0.15 & $\bullet$\\
autos & 0.28 & 0.28 &           & 0.30 &   $\circ$ & 0.27 &           & 0.28 &          \\
balance-scale & 0.35 & 0.35 &           & 0.35 &           & 0.36 &   $\circ$ & 0.35 &          \\
breast-cancer & 0.46 & 0.46 &           & 0.45 &           & 0.44 & $\bullet$ & 0.45 & $\bullet$\\
bridges-version1 & 0.29 & 0.30 &   $\circ$ & 0.30 &   $\circ$ & 0.29 &           & 0.29 &          \\
car & 0.27 & 0.27 &           & 0.23 & $\bullet$ & 0.26 & $\bullet$ & 0.26 & $\bullet$\\
cmc & 0.47 & 0.47 &           & 0.45 & $\bullet$ & 0.45 & $\bullet$ & 0.46 & $\bullet$\\
colic & 0.42 & 0.43 &           & 0.40 & $\bullet$ & 0.41 & $\bullet$ & 0.41 & $\bullet$\\
cylinder-bands & 0.50 & 0.50 &   $\circ$ & 0.45 & $\bullet$ & 0.47 & $\bullet$ & 0.47 & $\bullet$\\
dermatology & 0.10 & 0.10 &           & 0.12 &   $\circ$ & 0.09 & $\bullet$ & 0.09 & $\bullet$\\
diabetes & 0.44 & 0.44 &           & 0.42 & $\bullet$ & 0.42 & $\bullet$ & 0.43 & $\bullet$\\
ecoli & 0.24 & 0.24 &           & 0.23 & $\bullet$ & 0.23 & $\bullet$ & 0.23 & $\bullet$\\
flags & 0.27 & 0.27 &           & 0.28 &   $\circ$ & 0.27 &           & 0.27 &          \\
glass & 0.30 & 0.30 &           & 0.29 & $\bullet$ & 0.29 & $\bullet$ & 0.29 & $\bullet$\\
haberman & 0.45 & 0.45 &           & 0.45 &           & 0.44 & $\bullet$ & 0.45 &          \\
hayes-roth-train & 0.40 & 0.40 &           & 0.38 & $\bullet$ & 0.39 & $\bullet$ & 0.39 & $\bullet$\\
heart-h & 0.24 & 0.24 &           & 0.24 &   $\circ$ & 0.23 & $\bullet$ & 0.23 & $\bullet$\\
heart-statlog & 0.44 & 0.44 &           & 0.42 & $\bullet$ & 0.42 & $\bullet$ & 0.42 & $\bullet$\\
hepatitis & 0.39 & 0.39 &           & 0.39 &           & 0.39 & $\bullet$ & 0.40 &          \\
hypothyroid & 0.20 & 0.20 &           & 0.18 & $\bullet$ & 0.18 & $\bullet$ & 0.18 & $\bullet$\\
ionosphere & 0.50 & 0.51 &   $\circ$ & 0.39 & $\bullet$ & 0.43 & $\bullet$ & 0.43 & $\bullet$\\
iris & 0.35 & 0.36 &   $\circ$ & 0.33 & $\bullet$ & 0.34 & $\bullet$ & 0.33 & $\bullet$\\
kr-vs-kp & 0.34 & 0.34 &   $\circ$ & 0.29 & $\bullet$ & 0.34 & $\bullet$ & 0.33 & $\bullet$\\
labor & 0.18 & 0.18 &           & 0.22 &   $\circ$ & 0.19 &           & 0.17 &          \\
letter & 0.18 & 0.18 &           & 0.16 & $\bullet$ & 0.17 & $\bullet$ & 0.17 & $\bullet$\\
liver-disorders & 0.49 & 0.49 &           & 0.48 & $\bullet$ & 0.48 & $\bullet$ & 0.48 & $\bullet$\\
lung-cancer & 0.51 & 0.53 &           & 0.56 &   $\circ$ & 0.48 & $\bullet$ & 0.54 &   $\circ$\\
lymph & 0.24 & 0.25 &   $\circ$ & 0.24 &           & 0.24 & $\bullet$ & 0.24 & $\bullet$\\
molecular-biology-promoters & 0.30 & 0.30 &           & 0.43 &   $\circ$ & 0.31 &           & 0.29 &          \\
mushroom & 0.25 & 0.25 &   $\circ$ & 0.07 & $\bullet$ & 0.26 &   $\circ$ & 0.18 & $\bullet$\\
nursery & 0.24 & 0.24 &           & 0.21 & $\bullet$ & 0.23 & $\bullet$ & 0.22 & $\bullet$\\
optdigits & 0.17 & 0.17 &   $\circ$ & 0.13 & $\bullet$ & 0.16 & $\bullet$ & 0.15 & $\bullet$\\
page-blocks & 0.24 & 0.26 &   $\circ$ & 0.17 & $\bullet$ & 0.20 & $\bullet$ & 0.21 & $\bullet$\\
pendigits & 0.21 & 0.21 &           & 0.16 & $\bullet$ & 0.20 & $\bullet$ & 0.19 & $\bullet$\\
postoperative-patient-data & 0.40 & 0.40 &           & 0.41 &   $\circ$ & 0.39 &           & 0.39 & $\bullet$\\
primary-tumor & 0.18 & 0.18 &           & 0.18 &           & 0.18 & $\bullet$ & 0.18 & $\bullet$\\
segment & 0.29 & 0.29 &           & 0.23 & $\bullet$ & 0.26 & $\bullet$ & 0.25 & $\bullet$\\
shuttle-landing-control & 0.48 & 0.48 &           & 0.49 &   $\circ$ & 0.48 &           & 0.48 &          \\
sick & 0.26 & 0.26 &           & 0.23 & $\bullet$ & 0.25 & $\bullet$ & 0.26 &          \\
solar-flare-2 & 0.26 & 0.26 &   $\circ$ & 0.25 & $\bullet$ & 0.25 & $\bullet$ & 0.25 & $\bullet$\\
sonar & 0.46 & 0.46 &           & 0.42 & $\bullet$ & 0.45 & $\bullet$ & 0.44 & $\bullet$\\
soybean & 0.10 & 0.10 &           & 0.09 & $\bullet$ & 0.10 &           & 0.09 & $\bullet$\\
spambase & 0.44 & 0.44 &           & 0.39 & $\bullet$ & 0.42 & $\bullet$ & 0.42 & $\bullet$\\
tae & 0.47 & 0.47 &           & 0.47 &   $\circ$ & 0.46 & $\bullet$ & 0.46 & $\bullet$\\
tic-tac-toe & 0.44 & 0.44 &           & 0.43 & $\bullet$ & 0.44 & $\bullet$ & 0.44 & $\bullet$\\
trains & 0.47 & 0.43 &           & 0.29 & $\bullet$ & 0.43 &           & 0.43 &          \\
vehicle & 0.46 & 0.46 &           & 0.37 & $\bullet$ & 0.44 & $\bullet$ & 0.43 & $\bullet$\\
vote & 0.29 & 0.29 &           & 0.23 & $\bullet$ & 0.29 & $\bullet$ & 0.29 &          \\
vowel & 0.28 & 0.29 &   $\circ$ & 0.27 & $\bullet$ & 0.28 & $\bullet$ & 0.27 & $\bullet$\\
waveform-5000 & 0.36 & 0.36 &           & 0.34 & $\bullet$ & 0.35 & $\bullet$ & 0.35 & $\bullet$\\
zoo & 0.15 & 0.16 &   $\circ$ & 0.15 &           & 0.14 & $\bullet$ & 0.14 &          \\
\hline
\multicolumn{10}{c}{$\circ$, $\bullet$ statistically significant improvement or degradation}\\
\end{longtable} \footnotesize \par}

\input{tables/bayes_net_root_mean_squared_error_hidden_five.tex}
\begin{table}[thb]
\caption{\label{bnmeg3}Bayesian Network Mean Entropy Gain - Three Hidden Variables}
\footnotesize
{\centering \begin{tabular}{lrr@{\hspace{0.1cm}}cr@{\hspace{0.1cm}}cr@{\hspace{0.1cm}}cr@{\hspace{0.1cm}}c}
\\
\hline
Dataset & (1)& (2) & & (3) & & (4) & & (5) & \\
\hline
audiology & -1.01 & -1.00 &           & -1.86 & $\bullet$ & -2.14 & $\bullet$ & -0.61 &   $\circ$\\
autos & -0.12 & -0.13 &           &  0.20 &   $\circ$ &  0.11 &   $\circ$ &  0.02 &   $\circ$\\
balance-scale &  0.36 &  0.36 &           &  0.37 &   $\circ$ &  0.32 & $\bullet$ &  0.35 & $\bullet$\\
breast-cancer & -0.09 & -0.09 &           & -0.02 &   $\circ$ & -0.00 &   $\circ$ & -0.05 &   $\circ$\\
bridges-version1 &  0.47 &  0.43 &           &  0.61 &   $\circ$ &  0.67 &   $\circ$ &  0.60 &   $\circ$\\
car &  0.25 &  0.25 &           &  0.67 &   $\circ$ &  0.57 &   $\circ$ &  0.60 &   $\circ$\\
cmc & -0.11 & -0.11 &           &  0.09 &   $\circ$ &  0.04 &   $\circ$ &  0.03 &   $\circ$\\
colic & -0.43 & -0.44 &           &  0.00 &   $\circ$ & -0.17 &   $\circ$ & -0.21 &   $\circ$\\
cylinder-bands & -0.17 & -0.20 & $\bullet$ &  0.08 &   $\circ$ &  0.04 &   $\circ$ &  0.02 &   $\circ$\\
dermatology &  2.25 &  2.25 &           &  2.15 & $\bullet$ &  2.29 &   $\circ$ &  2.30 &   $\circ$\\
diabetes &  0.11 &  0.10 &           &  0.17 &   $\circ$ &  0.16 &   $\circ$ &  0.16 &   $\circ$\\
ecoli &  0.75 &  0.78 &           &  0.97 &   $\circ$ &  0.96 &   $\circ$ &  0.93 &   $\circ$\\
flags & -0.08 & -0.05 &           & -0.09 &           &  0.27 &   $\circ$ &  0.18 &   $\circ$\\
glass &  0.37 &  0.37 &           &  0.49 &   $\circ$ &  0.53 &   $\circ$ &  0.49 &   $\circ$\\
haberman & -0.04 & -0.04 &           & -0.03 &           & -0.01 &   $\circ$ & -0.02 &   $\circ$\\
hayes-roth-train &  0.07 &  0.06 &           &  0.15 &   $\circ$ &  0.10 &   $\circ$ &  0.13 &   $\circ$\\
heart-h &  0.22 &  0.23 &           &  0.26 &   $\circ$ &  0.30 &   $\circ$ &  0.29 &   $\circ$\\
heart-statlog &  0.11 &  0.11 &           &  0.21 &   $\circ$ &  0.22 &   $\circ$ &  0.20 &   $\circ$\\
hepatitis & -0.25 & -0.25 &           & -0.14 &   $\circ$ & -0.09 &   $\circ$ & -0.18 &   $\circ$\\
hypothyroid &  0.04 &  0.02 &           &  0.10 &   $\circ$ &  0.09 &   $\circ$ &  0.09 &   $\circ$\\
ionosphere & -0.64 & -0.76 & $\bullet$ &  0.01 &   $\circ$ & -0.08 &   $\circ$ & -0.09 &   $\circ$\\
iris &  0.64 &  0.61 &           &  0.79 &   $\circ$ &  0.76 &   $\circ$ &  0.79 &   $\circ$\\
kr-vs-kp &  0.48 &  0.47 & $\bullet$ &  0.60 &   $\circ$ &  0.48 &   $\circ$ &  0.52 &   $\circ$\\
labor &  0.68 &  0.70 &   $\circ$ &  0.65 &           &  0.70 &           &  0.72 &   $\circ$\\
letter &  1.38 &  1.37 &           &  2.03 &   $\circ$ &  1.58 &   $\circ$ &  1.68 &   $\circ$\\
liver-disorders &  0.02 &  0.01 &           &  0.06 &   $\circ$ &  0.03 &   $\circ$ &  0.05 &   $\circ$\\
lung-cancer & -2.13 & -2.08 &           & -1.97 &           & -1.21 &   $\circ$ & -1.94 &          \\
lymph &  0.68 &  0.64 &           &  0.62 &           &  0.72 &   $\circ$ &  0.71 &   $\circ$\\
molecular-biology-promoters &  0.41 &  0.38 &           & -0.28 & $\bullet$ &  0.43 &           &  0.49 &   $\circ$\\
mushroom &  0.48 &  0.47 & $\bullet$ &  0.97 &   $\circ$ &  0.57 &   $\circ$ &  0.80 &   $\circ$\\
nursery &  0.76 &  0.78 &           &  1.14 &   $\circ$ &  1.00 &   $\circ$ &  1.08 &   $\circ$\\
optdigits &  2.10 &  2.09 & $\bullet$ &  2.78 &   $\circ$ &  2.37 &   $\circ$ &  2.41 &   $\circ$\\
page-blocks & -0.14 & -0.27 & $\bullet$ &  0.24 &   $\circ$ &  0.11 &   $\circ$ &  0.09 &   $\circ$\\
pendigits &  1.56 &  1.57 &           &  2.49 &   $\circ$ &  2.00 &   $\circ$ &  2.04 &   $\circ$\\
postoperative-patient-data & -0.22 & -0.23 &           & -0.24 &           & -0.17 &   $\circ$ & -0.17 &   $\circ$\\
primary-tumor &  0.51 &  0.51 &           &  0.62 &   $\circ$ &  0.71 &   $\circ$ &  0.70 &   $\circ$\\
segment &  0.64 &  0.65 &           &  1.81 &   $\circ$ &  1.25 &   $\circ$ &  1.30 &   $\circ$\\
shuttle-landing-control &  0.04 &  0.05 &           & -0.01 & $\bullet$ &  0.04 &           &  0.05 &          \\
sick & -0.00 & -0.00 &           &  0.08 &   $\circ$ &  0.04 &   $\circ$ &  0.02 &   $\circ$\\
solar-flare-2 &  1.21 &  1.17 & $\bullet$ &  1.32 &   $\circ$ &  1.28 &   $\circ$ &  1.33 &   $\circ$\\
sonar & -0.68 & -0.68 &           &  0.09 &   $\circ$ & -0.33 &   $\circ$ & -0.30 &   $\circ$\\
soybean &  3.10 &  3.09 &           &  3.31 &   $\circ$ &  3.23 &   $\circ$ &  3.27 &   $\circ$\\
spambase &  0.02 &  0.01 & $\bullet$ &  0.28 &   $\circ$ &  0.17 &   $\circ$ &  0.15 &   $\circ$\\
tae & -0.02 & -0.03 &           & -0.02 &           &  0.03 &   $\circ$ &  0.03 &   $\circ$\\
tic-tac-toe &  0.11 &  0.11 &           &  0.16 &   $\circ$ &  0.12 &   $\circ$ &  0.13 &   $\circ$\\
trains & -0.73 & -0.37 &   $\circ$ &  0.32 &   $\circ$ &  0.01 &   $\circ$ & -0.11 &   $\circ$\\
vehicle & -1.48 & -1.49 &           &  0.51 &   $\circ$ & -0.68 &   $\circ$ & -0.79 &   $\circ$\\
vote &  0.05 &  0.05 &           &  0.66 &   $\circ$ &  0.28 &   $\circ$ &  0.18 &   $\circ$\\
vowel &  0.38 &  0.30 & $\bullet$ &  1.00 &   $\circ$ &  0.63 &   $\circ$ &  0.69 &   $\circ$\\
waveform-5000 &  0.31 &  0.32 &           &  0.78 &   $\circ$ &  0.60 &   $\circ$ &  0.55 &   $\circ$\\
zoo &  1.96 &  1.88 & $\bullet$ &  1.94 &           &  2.06 &   $\circ$ &  2.04 &   $\circ$\\
\hline
\multicolumn{10}{c}{$\circ$, $\bullet$ statistically significant improvement or degradation}\\
\end{tabular} \footnotesize \par}
\end{table}
\input{tables/bayes_net_mean_entropy_gain_hidden_five.tex}
\linespread{1.3}

\subsection{J48 Decision Tree}

J48 Decision Tree results for various numbers of hidden variables can be seen in figures \ref{j48pi3} - \ref{j48meg5}. As in section \ref{j48experiments}, the maximum number of iterations was set to 30 in order to obtain results in a reasonable time.

The J48 Percent Incorrect results can be seen in figures \ref{j48pi}, \ref{j48pi3} and \ref{j48pi5}. As more hidden variables were added, Case 1 became less accurate relative to Case 3. At zero hidden variables it was more accurate in 9/51 cases, while with 5 hidden variables this had decreased to to 4/51. It also remained better than Case 4 in all three experiments, although became worse relative to Case 5 as more hidden variables were added.

The J48 MAE results are shown in \ref{j48mae}, \ref{j48mae3} and \ref{j48mae5}. Adding hidden variables seemed to have a much smaller effect on this metric when using J48 rather than the other classifiers. Only small changes were observed, probably due to the much lower cap on the max iterations. Case 1 began to perform slightly better than Case 3 as hidden variables were added, while it also began to perform slightly worse relative to Cases 4 and 5.

The J48 RMSE results (figures \ref{j48rmse}, \ref{j48rmse3} and \ref{j48rmse5}) show that adding more hidden variables with this classifier caused the RMSE to increase on average. Case 1 initially performed better than Case 5 in 21/51 cases, while with five hidden variables added in only performed better in 4/51. The decrease in relative performance was less pronounced between Case 3 and Case 4, although decreases were still observed in both cases.

The J48 MEG results can be seen in \ref{j48meg}, \ref{j48meg3} and \ref{j48meg5}. Case 1 saw large entropy reductions relative to Cases 3 and 5 as more hidden variables were added. Case 1 also saw large entropy \textit{gains} relative to Case 4. This is probably due to how J48 decision trees are built; imputing the most common value for missing values will heavily influence how the tree splits, and therefore decrease the entropy of the system.

\linespread{1.0}
\newpage
{\centering \footnotesize \begin{longtable}{lrr@{\hspace{0.1cm}}cr@{\hspace{0.1cm}}cr@{\hspace{0.1cm}}cr@{\hspace{0.1cm}}c}
\caption{\label{j48pi3}J48 Decision Tree Percent Incorrect - Three Hidden Variables}
\\
\hline
Dataset & (1)& (2) & & (3) & & (4) & & (5) & \\
\hline
audiology & 30.27 & 30.48 &           & 32.94 &   $\circ$ & 35.33 &   $\circ$ & 32.94 &   $\circ$\\
autos & 13.84 & 14.12 &           & 13.02 &           & 14.68 &           & 13.02 &          \\
balance-scale & 28.88 & 28.81 &           & 29.02 &           & 29.56 &           & 29.02 &          \\
breast-cancer & 25.97 & 27.26 &   $\circ$ & 26.80 &   $\circ$ & 26.68 &           & 26.80 &   $\circ$\\
bridges-version1 & 43.00 & 44.01 &           & 41.77 &           & 45.89 &   $\circ$ & 41.77 &          \\
car & 16.25 & 16.52 &           & 16.73 &   $\circ$ & 20.18 &   $\circ$ & 16.73 &   $\circ$\\
cmc & 49.71 & 50.30 &           & 50.17 &           & 50.40 &           & 50.17 &          \\
colic & 18.74 & 17.91 & $\bullet$ & 17.32 & $\bullet$ & 18.98 &           & 17.32 & $\bullet$\\
cylinder-bands & 34.53 & 34.14 &           & 34.84 &           & 35.73 &           & 34.84 &          \\
dermatology & 11.00 & 11.63 &           & 10.04 &           & 13.33 &   $\circ$ & 10.04 &          \\
diabetes & 28.94 & 28.87 &           & 28.61 &           & 27.73 & $\bullet$ & 28.61 &          \\
ecoli & 34.69 & 34.53 &           & 34.92 &           & 37.23 &   $\circ$ & 34.92 &          \\
flags & 41.71 & 42.19 &           & 41.94 &           & 44.77 &   $\circ$ & 41.94 &          \\
glass & 41.70 & 42.73 &           & 38.89 & $\bullet$ & 41.35 &           & 38.89 & $\bullet$\\
haberman & 27.08 & 26.57 &           & 26.57 &           & 26.57 &           & 26.57 &          \\
hayes-roth-train & 40.97 & 41.15 &           & 37.95 & $\bullet$ & 42.43 &           & 37.95 & $\bullet$\\
heart-h & 21.69 & 21.28 &           & 22.94 &   $\circ$ & 23.77 &   $\circ$ & 22.94 &   $\circ$\\
heart-statlog & 31.00 & 30.00 &           & 32.32 &           & 30.04 &           & 32.32 &          \\
hepatitis & 23.33 & 21.75 &           & 22.10 &           & 23.28 &           & 22.10 &          \\
hypothyroid &  7.73 &  7.76 &           &  7.73 &           &  7.73 &           &  7.73 &          \\
ionosphere & 20.92 & 19.21 & $\bullet$ & 19.37 & $\bullet$ & 21.55 &           & 19.37 & $\bullet$\\
iris & 24.61 & 25.76 &           & 20.50 & $\bullet$ & 22.24 & $\bullet$ & 20.50 & $\bullet$\\
kr-vs-kp &  5.99 &  5.93 &           &  6.09 &           &  5.88 &           &  6.09 &          \\
labor & 23.60 & 26.00 &           & 27.80 &   $\circ$ & 21.40 &           & 27.80 &   $\circ$\\
letter & 49.51 & 49.39 &           & 47.39 & $\bullet$ & 51.04 &   $\circ$ & 47.39 & $\bullet$\\
liver-disorders & 39.27 & 39.84 &           & 40.16 &           & 38.54 &           & 40.16 &          \\
lung-cancer & 58.40 & 56.15 &           & 67.29 &   $\circ$ & 55.14 &           & 67.29 &   $\circ$\\
lymph & 20.47 & 20.03 &           & 21.75 &           & 23.36 &   $\circ$ & 21.75 &          \\
molecular-biology-promoters & 21.91 & 21.11 &           & 20.22 &           & 26.85 &   $\circ$ & 20.22 &          \\
mushroom &  2.20 &  3.56 &   $\circ$ &  0.86 & $\bullet$ &  0.57 & $\bullet$ &  0.86 & $\bullet$\\
nursery & 12.32 & 12.48 &   $\circ$ & 12.58 &   $\circ$ & 14.34 &   $\circ$ & 12.58 &   $\circ$\\
optdigits & 17.22 & 17.25 &           & 13.90 & $\bullet$ & 23.70 &   $\circ$ & 13.90 & $\bullet$\\
page-blocks &  7.30 &  7.23 &           &  7.25 &           &  7.27 &           &  7.25 &          \\
pendigits & 15.90 & 15.92 &           & 11.80 & $\bullet$ & 16.89 &   $\circ$ & 11.80 & $\bullet$\\
postoperative-patient-data & 29.61 & 29.35 &           & 29.24 &           & 29.24 &           & 29.24 &          \\
primary-tumor & 62.48 & 63.10 &           & 60.68 & $\bullet$ & 61.71 &           & 60.68 & $\bullet$\\
segment & 26.84 & 26.79 &           & 21.77 & $\bullet$ & 25.39 & $\bullet$ & 21.77 & $\bullet$\\
shuttle-landing-control & 42.00 & 42.00 &           & 40.00 &           & 41.33 &           & 40.00 &          \\
sick &  6.20 &  6.25 &           &  6.19 &           &  6.19 &           &  6.19 &          \\
solar-flare-2 & 30.35 & 30.02 &           & 28.17 & $\bullet$ & 29.45 & $\bullet$ & 28.17 & $\bullet$\\
sonar & 33.83 & 33.20 &           & 30.93 & $\bullet$ & 32.63 &           & 30.93 & $\bullet$\\
soybean & 18.22 & 18.28 &           & 16.24 & $\bullet$ & 18.71 &           & 16.24 & $\bullet$\\
spambase & 22.31 & 22.72 &   $\circ$ & 21.79 & $\bullet$ & 22.36 &           & 21.79 & $\bullet$\\
tae & 51.06 & 52.79 &           & 48.07 & $\bullet$ & 51.07 &           & 48.07 & $\bullet$\\
tic-tac-toe & 21.97 & 22.35 &           & 21.55 &           & 21.98 &           & 21.55 &          \\
trains &  2.00 &  9.00 &           &  0.00 &           &  0.00 &           &  0.00 &          \\
vehicle & 40.60 & 40.90 &           & 38.88 & $\bullet$ & 42.76 &   $\circ$ & 38.88 & $\bullet$\\
vote &  8.31 &  8.49 &           &  6.34 & $\bullet$ &  7.03 & $\bullet$ &  6.34 & $\bullet$\\
vowel & 50.98 & 50.63 &           & 52.36 &   $\circ$ & 54.88 &   $\circ$ & 52.36 &   $\circ$\\
waveform-5000 & 27.60 & 27.58 &           & 25.72 & $\bullet$ & 32.41 &   $\circ$ & 25.72 & $\bullet$\\
zoo & 16.26 & 17.61 &           & 15.39 &           & 18.76 &   $\circ$ & 15.39 &          \\
\hline
\multicolumn{10}{c}{$\circ$, $\bullet$ statistically significant improvement or degradation}\\
\end{longtable} \footnotesize \par}
\newpage
{\centering \footnotesize \begin{longtable}{lrr@{\hspace{0.1cm}}cr@{\hspace{0.1cm}}cr@{\hspace{0.1cm}}cr@{\hspace{0.1cm}}c}
\caption{\label{j48pi5}J48 Decision Tree Percent Incorrect - Five Hidden Variables}
\\
\hline
Dataset & (1)& (2) & & (3) & & (4) & & (5) & \\
\hline
audiology & 30.39 & 30.37 &          & 32.94 &   $\circ$ & 35.33 &   $\circ$ & 32.94 &   $\circ$\\
autos & 13.30 & 14.44 &          & 13.02 &           & 14.68 &           & 13.02 &          \\
balance-scale & 29.11 & 29.60 &          & 29.02 &           & 29.56 &           & 29.02 &          \\
breast-cancer & 26.50 & 27.36 &          & 26.80 &           & 26.68 &           & 26.80 &          \\
bridges-version1 & 42.44 & 44.02 &          & 41.77 &           & 45.89 &   $\circ$ & 41.77 &          \\
car & 16.38 & 16.55 &          & 16.73 &           & 20.18 &   $\circ$ & 16.73 &          \\
cmc & 50.50 & 49.60 &          & 50.17 &           & 50.40 &           & 50.17 &          \\
colic & 18.22 & 18.68 &          & 17.32 & $\bullet$ & 18.98 &           & 17.32 & $\bullet$\\
cylinder-bands & 33.26 & 33.44 &          & 34.84 &           & 35.73 &   $\circ$ & 34.84 &          \\
dermatology & 11.06 & 11.75 &          & 10.04 &           & 13.33 &   $\circ$ & 10.04 &          \\
diabetes & 30.00 & 29.46 &          & 28.61 & $\bullet$ & 27.73 & $\bullet$ & 28.61 & $\bullet$\\
ecoli & 34.40 & 35.19 &          & 34.92 &           & 37.23 &   $\circ$ & 34.92 &          \\
flags & 41.15 & 41.64 &          & 41.94 &           & 44.77 &   $\circ$ & 41.94 &          \\
glass & 41.41 & 41.42 &          & 38.89 & $\bullet$ & 41.35 &           & 38.89 & $\bullet$\\
haberman & 26.57 & 26.75 &          & 26.57 &           & 26.57 &           & 26.57 &          \\
hayes-roth-train & 41.14 & 42.24 &          & 37.95 & $\bullet$ & 42.43 &           & 37.95 & $\bullet$\\
heart-h & 21.84 & 22.30 &          & 22.94 &           & 23.77 &   $\circ$ & 22.94 &          \\
heart-statlog & 31.25 & 31.33 &          & 32.32 &           & 30.04 &           & 32.32 &          \\
hepatitis & 23.06 & 21.59 &          & 22.10 &           & 23.28 &           & 22.10 &          \\
hypothyroid &  7.73 &  7.74 &          &  7.73 &           &  7.73 &           &  7.73 &          \\
ionosphere & 20.19 & 19.78 &          & 19.37 &           & 21.55 &           & 19.37 &          \\
iris & 23.70 & 23.70 &          & 20.50 & $\bullet$ & 22.24 &           & 20.50 & $\bullet$\\
kr-vs-kp &  6.05 &  6.06 &          &  6.09 &           &  5.88 &           &  6.09 &          \\
labor & 26.60 & 24.80 &          & 27.80 &           & 21.40 & $\bullet$ & 27.80 &          \\
letter & 49.51 & 49.41 &          & 47.39 & $\bullet$ & 51.04 &   $\circ$ & 47.39 & $\bullet$\\
liver-disorders & 40.29 & 39.14 &          & 40.16 &           & 38.54 & $\bullet$ & 40.16 &          \\
lung-cancer & 58.23 & 53.31 &          & 67.29 &   $\circ$ & 55.14 &           & 67.29 &   $\circ$\\
lymph & 20.62 & 20.67 &          & 21.75 &           & 23.36 &   $\circ$ & 21.75 &          \\
molecular-biology-promoters & 20.36 & 21.81 &          & 20.22 &           & 26.85 &   $\circ$ & 20.22 &          \\
mushroom &  2.14 &  3.60 &  $\circ$ &  0.86 & $\bullet$ &  0.57 & $\bullet$ &  0.86 & $\bullet$\\
nursery & 12.35 & 12.61 &  $\circ$ & 12.58 &   $\circ$ & 14.34 &   $\circ$ & 12.58 &   $\circ$\\
optdigits & 17.07 & 17.18 &          & 13.90 & $\bullet$ & 23.70 &   $\circ$ & 13.90 & $\bullet$\\
page-blocks &  7.17 &  7.32 &          &  7.25 &           &  7.27 &           &  7.25 &          \\
pendigits & 15.92 & 16.05 &          & 11.80 & $\bullet$ & 16.89 &   $\circ$ & 11.80 & $\bullet$\\
postoperative-patient-data & 29.24 & 29.61 &          & 29.24 &           & 29.24 &           & 29.24 &          \\
primary-tumor & 61.65 & 62.10 &          & 60.68 &           & 61.71 &           & 60.68 &          \\
segment & 26.88 & 27.00 &          & 21.77 & $\bullet$ & 25.39 & $\bullet$ & 21.77 & $\bullet$\\
shuttle-landing-control & 42.00 & 42.00 &          & 40.00 &           & 41.33 &           & 40.00 &          \\
sick &  6.19 &  6.32 &          &  6.19 &           &  6.19 &           &  6.19 &          \\
solar-flare-2 & 30.25 & 29.73 &          & 28.17 & $\bullet$ & 29.45 & $\bullet$ & 28.17 & $\bullet$\\
sonar & 32.45 & 31.76 &          & 30.93 &           & 32.63 &           & 30.93 &          \\
soybean & 18.14 & 18.25 &          & 16.24 & $\bullet$ & 18.71 &           & 16.24 & $\bullet$\\
spambase & 22.08 & 22.60 &  $\circ$ & 21.79 & $\bullet$ & 22.36 &   $\circ$ & 21.79 & $\bullet$\\
tae & 51.13 & 51.57 &          & 48.07 & $\bullet$ & 51.07 &           & 48.07 & $\bullet$\\
tic-tac-toe & 22.10 & 22.22 &          & 21.55 &           & 21.98 &           & 21.55 &          \\
trains &  2.00 &  3.00 &          &  0.00 &           &  0.00 &           &  0.00 &          \\
vehicle & 40.60 & 40.30 &          & 38.88 & $\bullet$ & 42.76 &   $\circ$ & 38.88 & $\bullet$\\
vote &  8.44 &  8.31 &          &  6.34 & $\bullet$ &  7.03 & $\bullet$ &  6.34 & $\bullet$\\
vowel & 50.87 & 51.51 &          & 52.36 &   $\circ$ & 54.88 &   $\circ$ & 52.36 &   $\circ$\\
waveform-5000 & 27.54 & 27.48 &          & 25.72 & $\bullet$ & 32.41 &   $\circ$ & 25.72 & $\bullet$\\
zoo & 18.35 & 18.56 &          & 15.39 & $\bullet$ & 18.76 &           & 15.39 & $\bullet$\\
\hline
\multicolumn{10}{c}{$\circ$, $\bullet$ statistically significant improvement or degradation}\\
\end{longtable} \footnotesize \par}
\input{tables/j48_mean_absolute_error_hidden_three.tex}
\newpage
{\centering \footnotesize \begin{longtable}{lrr@{\hspace{0.1cm}}cr@{\hspace{0.1cm}}cr@{\hspace{0.1cm}}cr@{\hspace{0.1cm}}c}
\caption{\label{j48mae5}J48 Decision Tree Mean Absolute Error - Five Hidden Variables}
\\
\hline
Dataset & (1)& (2) & & (3) & & (4) & & (5) & \\
\hline
audiology & 0.03 & 0.03 &           & 0.04 &   $\circ$ & 0.04 &   $\circ$ & 0.04 &   $\circ$\\
autos & 0.18 & 0.19 &           & 0.19 &           & 0.21 &   $\circ$ & 0.19 &          \\
balance-scale & 0.27 & 0.27 &           & 0.29 &   $\circ$ & 0.28 &   $\circ$ & 0.29 &   $\circ$\\
breast-cancer & 0.37 & 0.37 &           & 0.38 &   $\circ$ & 0.37 &           & 0.38 &   $\circ$\\
bridges-version1 & 0.16 & 0.17 &           & 0.17 &   $\circ$ & 0.18 &   $\circ$ & 0.17 &   $\circ$\\
car & 0.10 & 0.10 &           & 0.12 &   $\circ$ & 0.12 &   $\circ$ & 0.12 &   $\circ$\\
cmc & 0.39 & 0.39 &           & 0.40 &   $\circ$ & 0.39 &   $\circ$ & 0.40 &   $\circ$\\
colic & 0.26 & 0.26 &           & 0.28 &   $\circ$ & 0.27 &   $\circ$ & 0.28 &   $\circ$\\
cylinder-bands & 0.37 & 0.37 &           & 0.40 &   $\circ$ & 0.39 &   $\circ$ & 0.40 &   $\circ$\\
dermatology & 0.05 & 0.05 &           & 0.07 &   $\circ$ & 0.07 &   $\circ$ & 0.07 &   $\circ$\\
diabetes & 0.37 & 0.36 &           & 0.38 &   $\circ$ & 0.36 &           & 0.38 &   $\circ$\\
ecoli & 0.11 & 0.11 &           & 0.11 &   $\circ$ & 0.11 &   $\circ$ & 0.11 &   $\circ$\\
flags & 0.12 & 0.12 &           & 0.13 &   $\circ$ & 0.13 &   $\circ$ & 0.13 &   $\circ$\\
glass & 0.15 & 0.15 &           & 0.15 &           & 0.15 &           & 0.15 &          \\
haberman & 0.38 & 0.38 &           & 0.39 &   $\circ$ & 0.38 & $\bullet$ & 0.39 &   $\circ$\\
hayes-roth-train & 0.26 & 0.26 &           & 0.26 &           & 0.27 &   $\circ$ & 0.26 &          \\
heart-h & 0.11 & 0.11 &           & 0.12 &   $\circ$ & 0.13 &   $\circ$ & 0.12 &   $\circ$\\
heart-statlog & 0.37 & 0.37 &           & 0.40 &   $\circ$ & 0.38 &           & 0.40 &   $\circ$\\
hepatitis & 0.28 & 0.27 &           & 0.29 &           & 0.28 &           & 0.29 &          \\
hypothyroid & 0.07 & 0.07 &   $\circ$ & 0.07 &   $\circ$ & 0.07 &           & 0.07 &   $\circ$\\
ionosphere & 0.24 & 0.24 &           & 0.27 &   $\circ$ & 0.26 &   $\circ$ & 0.27 &   $\circ$\\
iris & 0.20 & 0.21 &           & 0.22 &   $\circ$ & 0.23 &   $\circ$ & 0.22 &   $\circ$\\
kr-vs-kp & 0.09 & 0.09 & $\bullet$ & 0.14 &   $\circ$ & 0.13 &   $\circ$ & 0.14 &   $\circ$\\
labor & 0.27 & 0.25 &           & 0.32 &   $\circ$ & 0.22 & $\bullet$ & 0.32 &   $\circ$\\
letter & 0.05 & 0.05 & $\bullet$ & 0.05 &   $\circ$ & 0.05 &   $\circ$ & 0.05 &   $\circ$\\
liver-disorders & 0.45 & 0.45 &           & 0.46 &   $\circ$ & 0.46 &           & 0.46 &   $\circ$\\
lung-cancer & 0.39 & 0.36 &           & 0.44 &   $\circ$ & 0.37 &           & 0.44 &   $\circ$\\
lymph & 0.13 & 0.13 &           & 0.15 &   $\circ$ & 0.14 &   $\circ$ & 0.15 &   $\circ$\\
molecular-biology-promoters & 0.24 & 0.26 &           & 0.27 &   $\circ$ & 0.30 &   $\circ$ & 0.27 &   $\circ$\\
mushroom & 0.05 & 0.05 &   $\circ$ & 0.03 & $\bullet$ & 0.05 &   $\circ$ & 0.03 & $\bullet$\\
nursery & 0.06 & 0.06 &   $\circ$ & 0.09 &   $\circ$ & 0.10 &   $\circ$ & 0.09 &   $\circ$\\
optdigits & 0.05 & 0.05 &           & 0.05 &   $\circ$ & 0.06 &   $\circ$ & 0.05 &   $\circ$\\
page-blocks & 0.05 & 0.05 &   $\circ$ & 0.05 &   $\circ$ & 0.05 &   $\circ$ & 0.05 &   $\circ$\\
pendigits & 0.05 & 0.05 & $\bullet$ & 0.05 &   $\circ$ & 0.06 &   $\circ$ & 0.05 &   $\circ$\\
postoperative-patient-data & 0.28 & 0.28 &           & 0.28 &   $\circ$ & 0.28 &           & 0.28 &   $\circ$\\
primary-tumor & 0.06 & 0.06 &           & 0.07 &   $\circ$ & 0.06 &           & 0.07 &   $\circ$\\
segment & 0.10 & 0.10 &           & 0.10 &   $\circ$ & 0.11 &   $\circ$ & 0.10 &   $\circ$\\
shuttle-landing-control & 0.49 & 0.49 &           & 0.49 &           & 0.49 &           & 0.49 &          \\
sick & 0.11 & 0.11 &           & 0.12 &   $\circ$ & 0.11 & $\bullet$ & 0.12 &   $\circ$\\
solar-flare-2 & 0.12 & 0.12 &           & 0.13 &   $\circ$ & 0.13 &   $\circ$ & 0.13 &   $\circ$\\
sonar & 0.34 & 0.34 &           & 0.35 &           & 0.34 &           & 0.35 &          \\
soybean & 0.02 & 0.03 &           & 0.03 &   $\circ$ & 0.03 &   $\circ$ & 0.03 &   $\circ$\\
spambase & 0.31 & 0.32 &   $\circ$ & 0.32 &   $\circ$ & 0.33 &   $\circ$ & 0.32 &   $\circ$\\
tae & 0.41 & 0.40 &           & 0.40 &           & 0.41 &   $\circ$ & 0.40 &          \\
tic-tac-toe & 0.28 & 0.29 &           & 0.30 &   $\circ$ & 0.28 &           & 0.30 &   $\circ$\\
trains & 0.02 & 0.03 &           & 0.00 &           & 0.00 &           & 0.00 &          \\
vehicle & 0.23 & 0.23 &           & 0.25 &   $\circ$ & 0.25 &   $\circ$ & 0.25 &   $\circ$\\
vote & 0.11 & 0.10 &           & 0.11 &           & 0.13 &   $\circ$ & 0.11 &          \\
vowel & 0.10 & 0.11 &           & 0.12 &   $\circ$ & 0.11 &   $\circ$ & 0.12 &   $\circ$\\
waveform-5000 & 0.21 & 0.21 &           & 0.22 &   $\circ$ & 0.24 &   $\circ$ & 0.22 &   $\circ$\\
zoo & 0.07 & 0.07 &           & 0.07 &           & 0.07 &           & 0.07 &          \\
\hline
\multicolumn{10}{c}{$\circ$, $\bullet$ statistically significant improvement or degradation}\\
\end{longtable} \footnotesize \par}
\newpage
{\centering \footnotesize \begin{longtable}{lrr@{\hspace{0.1cm}}cr@{\hspace{0.1cm}}cr@{\hspace{0.1cm}}cr@{\hspace{0.1cm}}c}
\caption{\label{j48rmse3}J48 Root Mean Squared Error - Three Hidden Variables}
\\
\hline
Dataset & (1)& (2) & & (3) & & (4) & & (5) & \\
\hline
audiology & 0.14 & 0.14 &           & 0.14 &   $\circ$ & 0.15 &   $\circ$ & 0.14 &   $\circ$\\
autos & 0.33 & 0.34 &           & 0.33 &           & 0.35 &   $\circ$ & 0.33 &          \\
balance-scale & 0.37 & 0.38 &           & 0.37 &           & 0.38 &           & 0.37 &          \\
breast-cancer & 0.44 & 0.45 &   $\circ$ & 0.44 &           & 0.45 &           & 0.44 &          \\
bridges-version1 & 0.32 & 0.33 &           & 0.31 & $\bullet$ & 0.33 &   $\circ$ & 0.31 & $\bullet$\\
car & 0.23 & 0.23 &           & 0.23 &   $\circ$ & 0.26 &   $\circ$ & 0.23 &   $\circ$\\
cmc & 0.46 & 0.46 &           & 0.45 & $\bullet$ & 0.46 &   $\circ$ & 0.45 & $\bullet$\\
colic & 0.39 & 0.38 & $\bullet$ & 0.37 & $\bullet$ & 0.40 &   $\circ$ & 0.37 & $\bullet$\\
cylinder-bands & 0.50 & 0.50 &           & 0.48 & $\bullet$ & 0.53 &   $\circ$ & 0.48 & $\bullet$\\
dermatology & 0.17 & 0.17 &           & 0.16 &           & 0.19 &   $\circ$ & 0.16 &          \\
diabetes & 0.44 & 0.44 &           & 0.44 & $\bullet$ & 0.44 &           & 0.44 & $\bullet$\\
ecoli & 0.24 & 0.24 &           & 0.23 & $\bullet$ & 0.24 &   $\circ$ & 0.23 & $\bullet$\\
flags & 0.28 & 0.28 &           & 0.27 & $\bullet$ & 0.30 &   $\circ$ & 0.27 & $\bullet$\\
glass & 0.29 & 0.29 &           & 0.28 & $\bullet$ & 0.30 &   $\circ$ & 0.28 & $\bullet$\\
haberman & 0.44 & 0.44 &           & 0.44 &           & 0.44 &           & 0.44 &          \\
hayes-roth-train & 0.37 & 0.37 &           & 0.36 & $\bullet$ & 0.38 &   $\circ$ & 0.36 & $\bullet$\\
heart-h & 0.25 & 0.25 &           & 0.26 &           & 0.27 &   $\circ$ & 0.26 &          \\
heart-statlog & 0.47 & 0.46 &           & 0.47 &           & 0.47 &           & 0.47 &          \\
hepatitis & 0.43 & 0.42 &           & 0.41 & $\bullet$ & 0.44 &           & 0.41 & $\bullet$\\
hypothyroid & 0.19 & 0.19 &           & 0.19 & $\bullet$ & 0.19 &   $\circ$ & 0.19 & $\bullet$\\
ionosphere & 0.40 & 0.39 &           & 0.38 & $\bullet$ & 0.43 &   $\circ$ & 0.38 & $\bullet$\\
iris & 0.33 & 0.33 &           & 0.32 & $\bullet$ & 0.34 &   $\circ$ & 0.32 & $\bullet$\\
kr-vs-kp & 0.21 & 0.21 &           & 0.22 &   $\circ$ & 0.24 &   $\circ$ & 0.22 &   $\circ$\\
labor & 0.41 & 0.42 &           & 0.45 &   $\circ$ & 0.40 &           & 0.45 &   $\circ$\\
letter & 0.16 & 0.16 &           & 0.15 & $\bullet$ & 0.16 &   $\circ$ & 0.15 & $\bullet$\\
liver-disorders & 0.50 & 0.50 &           & 0.49 & $\bullet$ & 0.50 &           & 0.49 & $\bullet$\\
lung-cancer & 0.55 & 0.54 &           & 0.57 &           & 0.55 &           & 0.57 &          \\
lymph & 0.29 & 0.28 &           & 0.29 &           & 0.31 &   $\circ$ & 0.29 &          \\
molecular-biology-promoters & 0.42 & 0.41 &           & 0.40 &           & 0.47 &   $\circ$ & 0.40 &          \\
mushroom & 0.14 & 0.15 &   $\circ$ & 0.09 & $\bullet$ & 0.10 & $\bullet$ & 0.09 & $\bullet$\\
nursery & 0.18 & 0.18 &   $\circ$ & 0.19 &   $\circ$ & 0.22 &   $\circ$ & 0.19 &   $\circ$\\
optdigits & 0.16 & 0.16 &           & 0.15 & $\bullet$ & 0.20 &   $\circ$ & 0.15 & $\bullet$\\
page-blocks & 0.16 & 0.16 &           & 0.16 &   $\circ$ & 0.16 &   $\circ$ & 0.16 &   $\circ$\\
pendigits & 0.15 & 0.15 &           & 0.14 & $\bullet$ & 0.17 &   $\circ$ & 0.14 & $\bullet$\\
postoperative-patient-data & 0.38 & 0.38 &           & 0.38 &           & 0.38 &           & 0.38 &          \\
primary-tumor & 0.20 & 0.20 &           & 0.19 & $\bullet$ & 0.20 &   $\circ$ & 0.19 & $\bullet$\\
segment & 0.23 & 0.23 &           & 0.21 & $\bullet$ & 0.24 &   $\circ$ & 0.21 & $\bullet$\\
shuttle-landing-control & 0.50 & 0.50 &           & 0.50 &           & 0.50 &           & 0.50 &          \\
sick & 0.24 & 0.24 &           & 0.24 &           & 0.24 &           & 0.24 &          \\
solar-flare-2 & 0.25 & 0.25 &           & 0.25 & $\bullet$ & 0.26 &   $\circ$ & 0.25 & $\bullet$\\
sonar & 0.53 & 0.52 &           & 0.48 & $\bullet$ & 0.54 &           & 0.48 & $\bullet$\\
soybean & 0.12 & 0.12 &           & 0.11 & $\bullet$ & 0.13 &   $\circ$ & 0.11 & $\bullet$\\
spambase & 0.40 & 0.40 &   $\circ$ & 0.39 & $\bullet$ & 0.41 &   $\circ$ & 0.39 & $\bullet$\\
tae & 0.46 & 0.47 &           & 0.45 & $\bullet$ & 0.47 &           & 0.45 & $\bullet$\\
tic-tac-toe & 0.40 & 0.41 &           & 0.39 & $\bullet$ & 0.41 &   $\circ$ & 0.39 & $\bullet$\\
trains & 0.02 & 0.08 &           & 0.00 &           & 0.00 &           & 0.00 &          \\
vehicle & 0.36 & 0.36 &           & 0.35 & $\bullet$ & 0.39 &   $\circ$ & 0.35 & $\bullet$\\
vote & 0.24 & 0.25 &   $\circ$ & 0.23 & $\bullet$ & 0.25 &   $\circ$ & 0.23 & $\bullet$\\
vowel & 0.25 & 0.25 &           & 0.24 & $\bullet$ & 0.27 &   $\circ$ & 0.24 & $\bullet$\\
waveform-5000 & 0.38 & 0.38 &           & 0.35 & $\bullet$ & 0.43 &   $\circ$ & 0.35 & $\bullet$\\
zoo & 0.18 & 0.19 &           & 0.18 &           & 0.20 &   $\circ$ & 0.18 &          \\
\hline
\multicolumn{10}{c}{$\circ$, $\bullet$ statistically significant improvement or degradation}\\
\end{longtable} \footnotesize \par}
\newpage
{\centering \footnotesize \begin{longtable}{lrr@{\hspace{0.1cm}}cr@{\hspace{0.1cm}}cr@{\hspace{0.1cm}}cr@{\hspace{0.1cm}}c}
\caption{\label{j48rmse5}J48 Root Mean Squared Error - Five Hidden Variables}
\\
\hline
Dataset & (1)& (2) & & (3) & & (4) & & (5) & \\
\hline
audiology & 0.14 & 0.14 &           & 0.14 &   $\circ$ & 0.15 &   $\circ$ & 0.14 &   $\circ$\\
autos & 0.34 & 0.34 &           & 0.33 &           & 0.35 &           & 0.33 &          \\
balance-scale & 0.37 & 0.38 &           & 0.37 &           & 0.38 &           & 0.37 &          \\
breast-cancer & 0.45 & 0.45 &           & 0.44 &           & 0.45 &           & 0.44 &          \\
bridges-version1 & 0.32 & 0.32 &           & 0.31 &           & 0.33 &   $\circ$ & 0.31 &          \\
car & 0.23 & 0.23 &           & 0.23 &   $\circ$ & 0.26 &   $\circ$ & 0.23 &   $\circ$\\
cmc & 0.46 & 0.46 & $\bullet$ & 0.45 & $\bullet$ & 0.46 &           & 0.45 & $\bullet$\\
colic & 0.39 & 0.39 &           & 0.37 & $\bullet$ & 0.40 &   $\circ$ & 0.37 & $\bullet$\\
cylinder-bands & 0.49 & 0.50 &           & 0.48 & $\bullet$ & 0.53 &   $\circ$ & 0.48 & $\bullet$\\
dermatology & 0.17 & 0.17 &           & 0.16 &           & 0.19 &   $\circ$ & 0.16 &          \\
diabetes & 0.45 & 0.44 & $\bullet$ & 0.44 & $\bullet$ & 0.44 & $\bullet$ & 0.44 & $\bullet$\\
ecoli & 0.24 & 0.24 &   $\circ$ & 0.23 & $\bullet$ & 0.24 &   $\circ$ & 0.23 & $\bullet$\\
flags & 0.28 & 0.28 &           & 0.27 &           & 0.30 &   $\circ$ & 0.27 &          \\
glass & 0.29 & 0.29 &           & 0.28 & $\bullet$ & 0.30 &   $\circ$ & 0.28 & $\bullet$\\
haberman & 0.44 & 0.44 &           & 0.44 & $\bullet$ & 0.44 &   $\circ$ & 0.44 & $\bullet$\\
hayes-roth-train & 0.37 & 0.37 &           & 0.36 & $\bullet$ & 0.38 &   $\circ$ & 0.36 & $\bullet$\\
heart-h & 0.25 & 0.26 &           & 0.26 &           & 0.27 &   $\circ$ & 0.26 &          \\
heart-statlog & 0.47 & 0.47 &           & 0.47 &           & 0.47 &           & 0.47 &          \\
hepatitis & 0.43 & 0.42 &           & 0.41 & $\bullet$ & 0.44 &           & 0.41 & $\bullet$\\
hypothyroid & 0.19 & 0.19 &           & 0.19 &           & 0.19 &           & 0.19 &          \\
ionosphere & 0.39 & 0.40 &           & 0.38 & $\bullet$ & 0.43 &   $\circ$ & 0.38 & $\bullet$\\
iris & 0.32 & 0.33 &           & 0.32 &           & 0.34 &   $\circ$ & 0.32 &          \\
kr-vs-kp & 0.21 & 0.21 &           & 0.22 &   $\circ$ & 0.24 &   $\circ$ & 0.22 &   $\circ$\\
labor & 0.43 & 0.41 &           & 0.45 &           & 0.40 &           & 0.45 &          \\
letter & 0.16 & 0.16 &           & 0.15 & $\bullet$ & 0.16 &   $\circ$ & 0.15 & $\bullet$\\
liver-disorders & 0.50 & 0.49 &           & 0.49 & $\bullet$ & 0.50 &           & 0.49 & $\bullet$\\
lung-cancer & 0.55 & 0.53 &           & 0.57 &           & 0.55 &           & 0.57 &          \\
lymph & 0.29 & 0.28 &           & 0.29 &           & 0.31 &   $\circ$ & 0.29 &          \\
molecular-biology-promoters & 0.40 & 0.42 &           & 0.40 &           & 0.47 &   $\circ$ & 0.40 &          \\
mushroom & 0.14 & 0.15 &   $\circ$ & 0.09 & $\bullet$ & 0.10 & $\bullet$ & 0.09 & $\bullet$\\
nursery & 0.18 & 0.18 &   $\circ$ & 0.19 &   $\circ$ & 0.22 &   $\circ$ & 0.19 &   $\circ$\\
optdigits & 0.16 & 0.16 &           & 0.15 & $\bullet$ & 0.20 &   $\circ$ & 0.15 & $\bullet$\\
page-blocks & 0.16 & 0.16 &   $\circ$ & 0.16 &           & 0.16 &   $\circ$ & 0.16 &          \\
pendigits & 0.15 & 0.15 &   $\circ$ & 0.14 & $\bullet$ & 0.17 &   $\circ$ & 0.14 & $\bullet$\\
postoperative-patient-data & 0.38 & 0.38 &           & 0.38 &           & 0.38 &           & 0.38 &          \\
primary-tumor & 0.20 & 0.20 &           & 0.19 & $\bullet$ & 0.20 &   $\circ$ & 0.19 & $\bullet$\\
segment & 0.23 & 0.23 &           & 0.21 & $\bullet$ & 0.24 &   $\circ$ & 0.21 & $\bullet$\\
shuttle-landing-control & 0.50 & 0.50 &           & 0.50 &           & 0.50 &           & 0.50 &          \\
sick & 0.24 & 0.24 &           & 0.24 & $\bullet$ & 0.24 &   $\circ$ & 0.24 & $\bullet$\\
solar-flare-2 & 0.25 & 0.25 &           & 0.25 & $\bullet$ & 0.26 &   $\circ$ & 0.25 & $\bullet$\\
sonar & 0.51 & 0.51 &           & 0.48 & $\bullet$ & 0.54 &   $\circ$ & 0.48 & $\bullet$\\
soybean & 0.12 & 0.12 &           & 0.11 & $\bullet$ & 0.13 &   $\circ$ & 0.11 & $\bullet$\\
spambase & 0.40 & 0.40 &   $\circ$ & 0.39 & $\bullet$ & 0.41 &   $\circ$ & 0.39 & $\bullet$\\
tae & 0.47 & 0.46 &           & 0.45 & $\bullet$ & 0.47 &           & 0.45 & $\bullet$\\
tic-tac-toe & 0.40 & 0.41 &           & 0.39 & $\bullet$ & 0.41 &   $\circ$ & 0.39 & $\bullet$\\
trains & 0.02 & 0.03 &           & 0.00 &           & 0.00 &           & 0.00 &          \\
vehicle & 0.36 & 0.36 &           & 0.35 & $\bullet$ & 0.39 &   $\circ$ & 0.35 & $\bullet$\\
vote & 0.24 & 0.24 &           & 0.23 & $\bullet$ & 0.25 &           & 0.23 & $\bullet$\\
vowel & 0.25 & 0.25 &           & 0.24 & $\bullet$ & 0.27 &   $\circ$ & 0.24 & $\bullet$\\
waveform-5000 & 0.37 & 0.37 &           & 0.35 & $\bullet$ & 0.43 &   $\circ$ & 0.35 & $\bullet$\\
zoo & 0.20 & 0.20 &           & 0.18 & $\bullet$ & 0.20 &           & 0.18 & $\bullet$\\
\hline
\multicolumn{10}{c}{$\circ$, $\bullet$ statistically significant improvement or degradation}\\
\end{longtable} \footnotesize \par}
\newpage
{\centering \footnotesize \begin{longtable}{lrr@{\hspace{0.1cm}}cr@{\hspace{0.1cm}}cr@{\hspace{0.1cm}}cr@{\hspace{0.1cm}}c}
\caption{\label{j48meg3}J48 Decision Tree Mean Entropy Gain - Three Hidden Variables}
\\
\hline
Dataset & (1)& (2) & & (3) & & (4) & & (5) & \\
\hline
audiology & -108.27 & -111.93 &           &  -87.04 &  $\circ$ & -158.79 & $\bullet$ &  -87.04 &  $\circ$\\
autos &  -13.12 &  -17.24 &           &   -3.68 &  $\circ$ &  -21.39 &           &   -3.68 &  $\circ$\\
balance-scale &   -1.44 &   -2.01 &           &    0.25 &  $\circ$ &   -2.59 &           &    0.25 &  $\circ$\\
breast-cancer &  -11.36 &  -12.58 &           &   -1.20 &  $\circ$ &  -16.63 &           &   -1.20 &  $\circ$\\
bridges-version1 & -134.90 & -145.54 &           &  -77.69 &  $\circ$ & -133.17 &           &  -77.69 &  $\circ$\\
car &  -15.90 &  -16.45 &           &   -1.06 &  $\circ$ &  -39.71 & $\bullet$ &   -1.06 &  $\circ$\\
cmc &  -22.03 &  -21.96 &           &   -0.50 &  $\circ$ &  -31.94 & $\bullet$ &   -0.50 &  $\circ$\\
colic &  -27.45 &  -20.50 &   $\circ$ &   -0.69 &  $\circ$ &  -36.02 & $\bullet$ &   -0.69 &  $\circ$\\
cylinder-bands &  -53.56 &  -54.23 &           &   -1.51 &  $\circ$ & -101.43 & $\bullet$ &   -1.51 &  $\circ$\\
dermatology &  -21.83 &  -23.49 &           &   -0.63 &  $\circ$ &  -55.02 & $\bullet$ &   -0.63 &  $\circ$\\
diabetes &   -4.72 &   -1.45 &   $\circ$ &    0.11 &  $\circ$ &   -2.54 &           &    0.11 &  $\circ$\\
ecoli &  -36.23 &  -35.86 &           &  -21.92 &  $\circ$ &  -51.93 & $\bullet$ &  -21.92 &  $\circ$\\
flags & -179.68 & -185.81 &           &  -59.40 &  $\circ$ & -262.88 & $\bullet$ &  -59.40 &  $\circ$\\
glass &  -79.32 &  -79.87 &           &  -20.88 &  $\circ$ & -134.49 & $\bullet$ &  -20.88 &  $\circ$\\
haberman &   -0.01 &   -0.00 &           &    0.00 &          &   -0.00 &           &    0.00 &         \\
hayes-roth-train &  -23.77 &  -21.93 &           &    0.32 &  $\circ$ &  -24.76 &           &    0.32 &  $\circ$\\
heart-h &  -20.45 &  -23.32 &           &   -0.60 &  $\circ$ &  -21.34 &           &   -0.60 &  $\circ$\\
heart-statlog &  -20.16 &  -13.23 &           &    0.08 &  $\circ$ &  -24.89 &           &    0.08 &  $\circ$\\
hepatitis &  -32.65 &  -32.56 &           &   -5.45 &  $\circ$ &  -56.53 & $\bullet$ &   -5.45 &  $\circ$\\
hypothyroid &   -0.00 &    0.01 &           &   -0.00 &  $\circ$ &   -0.00 & $\bullet$ &   -0.00 &  $\circ$\\
ionosphere &  -39.49 &  -36.41 &           &   -2.47 &  $\circ$ &  -86.84 & $\bullet$ &   -2.47 &  $\circ$\\
iris &  -15.26 &  -13.70 &           &    0.06 &  $\circ$ &  -14.53 &           &    0.06 &  $\circ$\\
kr-vs-kp &    0.01 &   -0.03 &           &    0.73 &  $\circ$ &   -4.54 & $\bullet$ &    0.73 &  $\circ$\\
labor &  -42.63 &  -64.07 &           &  -14.96 &          &  -78.97 & $\bullet$ &  -14.96 &         \\
letter &  -48.05 &  -46.34 &   $\circ$ &   -2.75 &  $\circ$ & -123.96 & $\bullet$ &   -2.75 &  $\circ$\\
liver-disorders &   -2.12 &   -2.82 &           &   -0.34 &          &   -7.05 & $\bullet$ &   -0.34 &         \\
lung-cancer & -304.56 & -282.51 &           & -285.77 &          & -349.74 &           & -285.77 &         \\
lymph &  -61.28 &  -63.72 &           &  -14.26 &  $\circ$ &  -98.35 & $\bullet$ &  -14.26 &  $\circ$\\
molecular-biology-promoters &  -96.84 &  -87.74 &           &  -10.05 &  $\circ$ & -120.80 &           &  -10.05 &  $\circ$\\
mushroom &    0.91 &    0.90 & $\bullet$ &    0.94 &  $\circ$ &    0.12 & $\bullet$ &    0.94 &  $\circ$\\
nursery &   -3.53 &   -3.26 &           &    1.23 &  $\circ$ &   -8.25 & $\bullet$ &    1.23 &  $\circ$\\
optdigits &  -46.04 &  -45.95 &           &   -5.72 &  $\circ$ & -172.03 & $\bullet$ &   -5.72 &  $\circ$\\
page-blocks &   -0.42 &   -0.44 &           &   -0.10 &  $\circ$ &   -1.16 & $\bullet$ &   -0.10 &  $\circ$\\
pendigits &   -9.69 &   -9.74 &           &    1.74 &  $\circ$ &  -52.19 & $\bullet$ &    1.74 &  $\circ$\\
postoperative-patient-data &   -5.38 &   -1.27 &           &   -0.01 &          &   -0.01 &           &   -0.01 &         \\
primary-tumor & -295.99 & -280.61 &           & -120.47 &  $\circ$ & -366.49 & $\bullet$ & -120.47 &  $\circ$\\
segment &  -10.11 &  -11.44 &           &    0.95 &  $\circ$ &  -51.51 & $\bullet$ &    0.95 &  $\circ$\\
shuttle-landing-control &   -0.03 &   -0.03 &           &   -0.00 &          &   -0.01 &           &   -0.00 &         \\
sick &   -0.09 &   -0.12 &           &    0.00 &          &   -0.00 &           &    0.00 &         \\
solar-flare-2 &  -22.42 &  -20.61 &           &   -1.14 &  $\circ$ &  -45.09 & $\bullet$ &   -1.14 &  $\circ$\\
sonar & -148.45 & -139.16 &           &  -26.02 &  $\circ$ & -213.09 & $\bullet$ &  -26.02 &  $\circ$\\
soybean &  -23.97 &  -25.22 &           &   -4.49 &  $\circ$ & -112.98 & $\bullet$ &   -4.49 &  $\circ$\\
spambase &    0.06 &    0.09 &           &    0.28 &  $\circ$ &    0.15 &           &    0.28 &  $\circ$\\
tae &  -15.98 &  -22.25 &           &    0.09 &  $\circ$ &  -18.33 &           &    0.09 &  $\circ$\\
tic-tac-toe &  -28.11 &  -28.61 &           &    0.12 &  $\circ$ &  -34.08 & $\bullet$ &    0.12 &  $\circ$\\
trains &    0.97 &  -42.07 & $\bullet$ &    1.01 &          &    1.01 &           &    1.01 &         \\
vehicle &  -34.63 &  -36.74 &           &   -3.43 &  $\circ$ & -125.50 & $\bullet$ &   -3.43 &  $\circ$\\
vote &   -1.82 &   -4.57 & $\bullet$ &   -0.43 &          &  -19.92 & $\bullet$ &   -0.43 &         \\
vowel &  -99.00 &  -97.54 &           &  -15.15 &  $\circ$ & -292.68 & $\bullet$ &  -15.15 &  $\circ$\\
waveform-5000 &  -49.15 &  -47.75 &           &   -0.12 &  $\circ$ & -165.15 & $\bullet$ &   -0.12 &  $\circ$\\
zoo &  -39.30 &  -46.47 &           &  -19.36 &  $\circ$ & -111.23 & $\bullet$ &  -19.36 &  $\circ$\\
\hline
\multicolumn{10}{c}{$\circ$, $\bullet$ statistically significant improvement or degradation}\\
\end{longtable} \footnotesize \par}
\newpage
{\centering \footnotesize \begin{longtable}{lrr@{\hspace{0.1cm}}cr@{\hspace{0.1cm}}cr@{\hspace{0.1cm}}cr@{\hspace{0.1cm}}c}
\caption{\label{j48meg5}J48 Decision Tree Mean Entropy Gain - Five Hidden Variables}
\\
\hline
Dataset & (1)& (2) & & (3) & & (4) & & (5) & \\
\hline
audiology & -105.03 & -110.89 &           &  -87.04 &  $\circ$ & -158.79 & $\bullet$ &  -87.04 &  $\circ$\\
autos &  -18.34 &  -17.84 &           &   -3.68 &  $\circ$ &  -21.39 &           &   -3.68 &  $\circ$\\
balance-scale &   -1.26 &   -1.45 &           &    0.25 &  $\circ$ &   -2.59 &           &    0.25 &  $\circ$\\
breast-cancer &  -12.60 &  -15.42 &           &   -1.20 &  $\circ$ &  -16.63 &           &   -1.20 &  $\circ$\\
bridges-version1 & -148.10 & -107.96 &   $\circ$ &  -77.69 &  $\circ$ & -133.17 &           &  -77.69 &  $\circ$\\
car &  -16.04 &  -15.51 &           &   -1.06 &  $\circ$ &  -39.71 & $\bullet$ &   -1.06 &  $\circ$\\
cmc &  -22.04 &  -18.29 &           &   -0.50 &  $\circ$ &  -31.94 & $\bullet$ &   -0.50 &  $\circ$\\
colic &  -22.84 &  -21.19 &           &   -0.69 &  $\circ$ &  -36.02 & $\bullet$ &   -0.69 &  $\circ$\\
cylinder-bands &  -51.38 &  -49.18 &           &   -1.51 &  $\circ$ & -101.43 & $\bullet$ &   -1.51 &  $\circ$\\
dermatology &  -19.87 &  -23.44 &           &   -0.63 &  $\circ$ &  -55.02 & $\bullet$ &   -0.63 &  $\circ$\\
diabetes &   -2.87 &   -1.77 &           &    0.11 &  $\circ$ &   -2.54 &           &    0.11 &  $\circ$\\
ecoli &  -32.71 &  -33.77 &           &  -21.92 &  $\circ$ &  -51.93 & $\bullet$ &  -21.92 &  $\circ$\\
flags & -176.06 & -184.33 &           &  -59.40 &  $\circ$ & -262.88 & $\bullet$ &  -59.40 &  $\circ$\\
glass &  -75.58 &  -66.34 &           &  -20.88 &  $\circ$ & -134.49 & $\bullet$ &  -20.88 &  $\circ$\\
haberman &   -0.00 &   -0.00 &           &    0.00 &  $\circ$ &   -0.00 & $\bullet$ &    0.00 &  $\circ$\\
hayes-roth-train &  -18.23 &  -17.24 &           &    0.32 &  $\circ$ &  -24.76 &           &    0.32 &  $\circ$\\
heart-h &  -20.47 &  -20.90 &           &   -0.60 &  $\circ$ &  -21.34 &           &   -0.60 &  $\circ$\\
heart-statlog &  -18.45 &  -18.48 &           &    0.08 &  $\circ$ &  -24.89 &           &    0.08 &  $\circ$\\
hepatitis &  -43.57 &  -31.08 &           &   -5.45 &  $\circ$ &  -56.53 &           &   -5.45 &  $\circ$\\
hypothyroid &    0.00 &    0.00 &           &   -0.00 &          &   -0.00 &           &   -0.00 &         \\
ionosphere &  -34.09 &  -36.48 &           &   -2.47 &  $\circ$ &  -86.84 & $\bullet$ &   -2.47 &  $\circ$\\
iris &  -16.90 &  -16.13 &           &    0.06 &  $\circ$ &  -14.53 &           &    0.06 &  $\circ$\\
kr-vs-kp &    0.04 &    0.08 &           &    0.73 &  $\circ$ &   -4.54 & $\bullet$ &    0.73 &  $\circ$\\
labor &  -49.10 &  -44.76 &           &  -14.96 &  $\circ$ &  -78.97 &           &  -14.96 &  $\circ$\\
letter &  -45.92 &  -45.06 &           &   -2.75 &  $\circ$ & -123.96 & $\bullet$ &   -2.75 &  $\circ$\\
liver-disorders &   -3.54 &   -2.46 &           &   -0.34 &  $\circ$ &   -7.05 &           &   -0.34 &  $\circ$\\
lung-cancer & -351.02 & -315.18 &           & -285.77 &          & -349.74 &           & -285.77 &         \\
lymph &  -62.13 &  -51.21 &           &  -14.26 &  $\circ$ &  -98.35 & $\bullet$ &  -14.26 &  $\circ$\\
molecular-biology-promoters &  -88.35 &  -98.66 &           &  -10.05 &  $\circ$ & -120.80 & $\bullet$ &  -10.05 &  $\circ$\\
mushroom &    0.91 &    0.90 & $\bullet$ &    0.94 &  $\circ$ &    0.12 & $\bullet$ &    0.94 &  $\circ$\\
nursery &   -2.22 &   -2.39 &           &    1.23 &  $\circ$ &   -8.25 & $\bullet$ &    1.23 &  $\circ$\\
optdigits &  -45.50 &  -45.83 &           &   -5.72 &  $\circ$ & -172.03 & $\bullet$ &   -5.72 &  $\circ$\\
page-blocks &   -0.44 &   -0.35 &           &   -0.10 &  $\circ$ &   -1.16 & $\bullet$ &   -0.10 &  $\circ$\\
pendigits &   -9.20 &   -9.41 &           &    1.74 &  $\circ$ &  -52.19 & $\bullet$ &    1.74 &  $\circ$\\
postoperative-patient-data &   -4.03 &   -2.70 &           &   -0.01 &          &   -0.01 &           &   -0.01 &         \\
primary-tumor & -288.14 & -269.52 &           & -120.47 &  $\circ$ & -366.49 & $\bullet$ & -120.47 &  $\circ$\\
segment &  -10.68 &   -9.55 &           &    0.95 &  $\circ$ &  -51.51 & $\bullet$ &    0.95 &  $\circ$\\
shuttle-landing-control &   -0.02 &   -0.03 &           &   -0.00 &          &   -0.01 &           &   -0.00 &         \\
sick &   -0.00 &   -0.18 &           &    0.00 &  $\circ$ &   -0.00 & $\bullet$ &    0.00 &  $\circ$\\
solar-flare-2 &  -19.83 &  -18.59 &           &   -1.14 &  $\circ$ &  -45.09 & $\bullet$ &   -1.14 &  $\circ$\\
sonar & -135.92 & -118.66 &           &  -26.02 &  $\circ$ & -213.09 & $\bullet$ &  -26.02 &  $\circ$\\
soybean &  -20.99 &  -22.94 &           &   -4.49 &  $\circ$ & -112.98 & $\bullet$ &   -4.49 &  $\circ$\\
spambase &    0.16 &    0.12 &           &    0.28 &          &    0.15 &           &    0.28 &         \\
tae &  -16.80 &   -6.08 &   $\circ$ &    0.09 &  $\circ$ &  -18.33 &           &    0.09 &  $\circ$\\
tic-tac-toe &  -21.80 &  -20.57 &           &    0.12 &  $\circ$ &  -34.08 & $\bullet$ &    0.12 &  $\circ$\\
trains &    0.97 &   -9.78 &           &    1.01 &          &    1.01 &           &    1.01 &         \\
vehicle &  -31.99 &  -34.90 &           &   -3.43 &  $\circ$ & -125.50 & $\bullet$ &   -3.43 &  $\circ$\\
vote &   -4.28 &   -4.28 &           &   -0.43 &  $\circ$ &  -19.92 & $\bullet$ &   -0.43 &  $\circ$\\
vowel &  -97.20 & -101.43 &           &  -15.15 &  $\circ$ & -292.68 & $\bullet$ &  -15.15 &  $\circ$\\
waveform-5000 &  -45.64 &  -45.07 &           &   -0.12 &  $\circ$ & -165.15 & $\bullet$ &   -0.12 &  $\circ$\\
zoo &  -61.66 &  -49.83 &           &  -19.36 &  $\circ$ & -111.23 & $\bullet$ &  -19.36 &  $\circ$\\
\hline
\multicolumn{10}{c}{$\circ$, $\bullet$ statistically significant improvement or degradation}\\
\end{longtable} \footnotesize \par}
\linespread{1.3}

\section{Conclusions}

Adding hidden variables did not appear to be particularly helpful for the iterative training method which was implemented in this work. While adding more hidden variables did further reduce the entropy gains of the classifier, it also decreased classifier accuracy in every test case, and also generally increased root mean squared error. While this method may prove useful for other classifiers that were not tested. it does not appear useful for those that were.