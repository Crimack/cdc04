% intro.tex
% Too short probably, maybe more background and some citations
\newpage
\chapter{Introduction}\label{intro}
The reliability of a machine learning model is heavily dependent on the data which was used to train it. Training data sets with missing values are therefore an interesting problem since there is no way of knowing what any given missing value could have been, and so the only two useful approaches involve either making statistical approximations of missing values or ignoring any record which contains missing data. Both mentioned approaches have positives as well as negatives, and may produce false outcomes. Ignoring a record prevents the use of unreliable data, but also may lead to valuable data in the other, non-missing attributes of the same record being discounted. The process of attempting to fill in a missing value is known as \textit{imputation}. This has no guarantee of being correct, but it may be worth the risk to gain value from the other complete attributes.  It is therefore extremely important that any imputation is carried out with care, and it must be very likely to be correct.

All implemented code for this project, as well as some additional utility material, can be found in the Git repository~\cite{gitlab}. It aims to investigate the usage of machine learning techniques on incomplete sets of discrete data. It will attempt to impute any missing values in the training data, by iteratively building and applying models until no further change is observed. It is expected that predicted missing values will trend towards the mean, leading to more reliable classification when this trained model is then applied to target data.  Given sufficient time, it is also intended that this approach can be used to infer 'hidden variables', similarly to how neural networks produce hidden layers to create relationships between data~\cite{Touretzky:1989}. This will be achieved by adding a new attribute to a data set, and initially filling it full of random valid data. We will then repeatedly build, apply and rebuild models to classify this attribute until it stops changing, in an identical manner to how missing data is imputed in the training set. New data will be added to the training data set by this method, and this project aims to determine whether or not these hidden variables lead to a greater classification accuracy. 
