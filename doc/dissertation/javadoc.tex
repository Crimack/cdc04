\documentclass[11pt,a4paper]{report}
\usepackage{color}
\usepackage{ifthen}
\usepackage{makeidx}
\usepackage{ifpdf}
\usepackage[headings]{fullpage}
\usepackage{listings}
\lstset{language=Java,breaklines=true}
\ifpdf \usepackage[pdftex, pdfpagemode={UseOutlines},bookmarks,colorlinks,linkcolor={blue},plainpages=false,pdfpagelabels,citecolor={red},breaklinks=true]{hyperref}
  \usepackage[pdftex]{graphicx}
  \pdfcompresslevel=9
  \DeclareGraphicsRule{*}{mps}{*}{}
\else
  \usepackage[dvips]{graphicx}
\fi

\newcommand{\entityintro}[3]{%
  \hbox to \hsize{%
    \vbox{%
      \hbox to .2in{}%
    }%
    {\bf  #1}%
    \dotfill\pageref{#2}%
  }
  \makebox[\hsize]{%
    \parbox{.4in}{}%
    \parbox[l]{5in}{%
      \vspace{1mm}%
      #3%
      \vspace{1mm}%
    }%
  }%
}
\newcommand{\refdefined}[1]{
\expandafter\ifx\csname r@#1\endcsname\relax
\relax\else
{$($in \ref{#1}, page \pageref{#1}$)$}\fi}
\date{10 May 2017}
\title{CDC04 TeXDoclet}
\author{Christopher McKee}
\chardef\textbackslash=`\\
\makeindex
\begin{document}
\maketitle
\sloppy
\addtocontents{toc}{\protect\markboth{Contents}{Contents}}
\tableofcontents
\chapter*{Class Hierarchy}{
\thispagestyle{empty}
\markboth{Class Hierarchy}{Class Hierarchy}
\addcontentsline{toc}{chapter}{Class Hierarchy}
\section*{Classes}
{\raggedright
\hspace{0.0cm} $\bullet$ java.lang.Object {\tiny \refdefined{java.lang.Object}} \\
\hspace{1.0cm} $\bullet$ cdc04.StateAnalyser {\tiny \refdefined{cdc04.StateAnalyser}} \\
\hspace{1.0cm} $\bullet$ weka.classifiers.AbstractClassifier {\tiny \refdefined{weka.classifiers.AbstractClassifier}} \\
\hspace{2.0cm} $\bullet$ weka.classifiers.SingleClassifierEnhancer {\tiny \refdefined{weka.classifiers.SingleClassifierEnhancer}} \\
\hspace{3.0cm} $\bullet$ weka.classifiers.meta.ProjectClassifier {\tiny \refdefined{weka.classifiers.meta.ProjectClassifier}} \\
}
}
\chapter{Package cdc04}{
\label{cdc04}\hskip -.05in
\hbox to \hsize{\textit{ Package Contents\hfil Page}}
\vskip .13in
\hbox{{\bf  Classes}}
\entityintro{StateAnalyser}{cdc04.StateAnalyser}{Class written to record a series of Weka instances, and to determine differences between them.}
\vskip .1in
\vskip .1in
\section{\label{cdc04.StateAnalyser}\index{StateAnalyser}Class StateAnalyser}{
\vskip .1in 
Class written to record a series of Weka instances, and to determine differences between them.\vskip .1in 
\subsection{Declaration}{
\begin{lstlisting}[frame=none]
public class StateAnalyser
 extends java.lang.Object implements java.io.Serializable\end{lstlisting}
\subsection{Constructor summary}{
\begin{verse}
{\bf StateAnalyser()} Constructs a new instance with no recorded instances\\
\end{verse}
}
\subsection{Method summary}{
\begin{verse}
{\bf addInstances(Instances)} Adds an Instances object to be tracked\\
{\bf getNumberDifferences()} Returns an integer representation of the number of rows which have changed between the previous two iterations of the classifier A negative return values shows that the number of differences could not be calculated, since there are not at least two iterations present.\\
{\bf getNumberIterations()} Returns the number of instances which are currently contained in the tracker for analysis\\
\end{verse}
}
\subsection{Constructors}{
\vskip -2em
\begin{itemize}
\item{ 
\index{StateAnalyser()}
{\bf  StateAnalyser}\\
\begin{lstlisting}[frame=none]
public StateAnalyser()\end{lstlisting} %end signature
\begin{itemize}
\item{
{\bf  Description}

Constructs a new instance with no recorded instances
}
\end{itemize}
}%end item
\end{itemize}
}
\subsection{Methods}{
\vskip -2em
\begin{itemize}
\item{ 
\index{addInstances(Instances)}
{\bf  addInstances}\\
\begin{lstlisting}[frame=none]
public void addInstances(weka.core.Instances toAdd)\end{lstlisting} %end signature
\begin{itemize}
\item{
{\bf  Description}

Adds an Instances object to be tracked
}
\item{
{\bf  Parameters}
  \begin{itemize}
   \item{
\texttt{toAdd} -- Instances object to be added}
  \end{itemize}
}%end item
\end{itemize}
}%end item
\item{ 
\index{getNumberDifferences()}
{\bf  getNumberDifferences}\\
\begin{lstlisting}[frame=none]
public int getNumberDifferences()\end{lstlisting} %end signature
\begin{itemize}
\item{
{\bf  Description}

Returns an integer representation of the number of rows which have changed between the previous two iterations of the classifier A negative return values shows that the number of differences could not be calculated, since there are not at least two iterations present.
}
\item{{\bf  Returns} -- 
int representing number of differences between previous two iterations 
}%end item
\end{itemize}
}%end item
\item{ 
\index{getNumberIterations()}
{\bf  getNumberIterations}\\
\begin{lstlisting}[frame=none]
public int getNumberIterations()\end{lstlisting} %end signature
\begin{itemize}
\item{
{\bf  Description}

Returns the number of instances which are currently contained in the tracker for analysis
}
\item{{\bf  Returns} -- 
the current number of recorded instances 
}%end item
\end{itemize}
}%end item
\end{itemize}
}
}
}
\chapter{Package weka.classifiers.meta}{
\label{weka.classifiers.meta}\hskip -.05in
\hbox to \hsize{\textit{ Package Contents\hfil Page}}
\vskip .13in
\hbox{{\bf  Classes}}
\entityintro{ProjectClassifier}{weka.classifiers.meta.ProjectClassifier}{A classifier which is iteratively trained, imputing missing values into copies of the training data until no further change is observed.}
\vskip .1in
\vskip .1in
\section{\label{weka.classifiers.meta.ProjectClassifier}\index{ProjectClassifier}Class ProjectClassifier}{
\vskip .1in 
A classifier which is iteratively trained, imputing missing values into copies of the training data until no further change is observed. Builds one learner per attribute, and therefore can take quite a while to run. Valid options are:\texttt{\small
\mbox{}\newline \phantom{ }-W\phantom{ }classifier}\mbox{}\newline
\texttt{\small \phantom{ }Full\phantom{ }path\phantom{ }to\phantom{ }the\phantom{ }target\phantom{ }classifier\phantom{ }to\phantom{ }use,\phantom{ }e.g.\phantom{ }weka.classifiers.trees.J48}\mbox{}\newline
\texttt{\small \phantom{ }}
\texttt{\small
\mbox{}\newline \phantom{ }-S}\mbox{}\newline
\texttt{\small \phantom{ }Defines\phantom{ }whether\phantom{ }or\phantom{ }not\phantom{ }the\phantom{ }classifier\phantom{ }will\phantom{ }impute\phantom{ }a\phantom{ }value\phantom{ }for\phantom{ }the\phantom{ }class}\mbox{}\newline
\texttt{\small \phantom{ }attribute\phantom{ }as\phantom{ }it\phantom{ }trains.}\mbox{}\newline
\texttt{\small \phantom{ }}
\texttt{\small
\mbox{}\newline \phantom{ }-R}\mbox{}\newline
\texttt{\small \phantom{ }If\phantom{ }set,\phantom{ }the\phantom{ }classifiers\phantom{ }will\phantom{ }be\phantom{ }trained\phantom{ }with\phantom{ }any\phantom{ }missing\phantom{ }arguments\phantom{ }filled\phantom{ }in}\mbox{}\newline
\texttt{\small \phantom{ }by\phantom{ }random\phantom{ }data.\phantom{ }The\phantom{ }classifier\phantom{ }will\phantom{ }then\phantom{ }only\phantom{ }iterate\phantom{ }once.}\mbox{}\newline
\texttt{\small \phantom{ }}
\texttt{\small
\mbox{}\newline \phantom{ }-M\phantom{ }integer}\mbox{}\newline
\texttt{\small \phantom{ }Sets\phantom{ }the\phantom{ }maximum\phantom{ }number\phantom{ }of\phantom{ }times\phantom{ }that\phantom{ }a\phantom{ }particular\phantom{ }classifier\phantom{ }will\phantom{ }iterate}\mbox{}\newline
\texttt{\small \phantom{ }before\phantom{ }determining\phantom{ }that\phantom{ }it\phantom{ }is\phantom{ }trained.}\mbox{}\newline
\texttt{\small \phantom{ }}
Options after -- are passed to the currently selected classifier.\vskip .1in 
\subsection{Declaration}{
\begin{lstlisting}[frame=none]
public class ProjectClassifier
 extends weka.classifiers.SingleClassifierEnhancer implements weka.classifiers.IterativeClassifier\end{lstlisting}
\subsection{Constructor summary}{
\begin{verse}
{\bf ProjectClassifier()} Constructor\\
\end{verse}
}
\subsection{Method summary}{
\begin{verse}
{\bf buildClassifier(Instances)} Builds a set of classifiers based on the training data.\\
{\bf classifierOptionsTipText()} Tip text to be displayed in the GUI for this property\\
{\bf classifyInstance(Instance)} Classifies an instance.\\
{\bf defaultClassifierString()} String describing default classifier.\\
{\bf distributionForInstance(Instance)} Returns class probabilities for an instance.\\
{\bf done()} Method called when iteration has terminated.\\
{\bf getCapabilities()} Returns default capabilities of the classifier.\\
{\bf getClassifierOptions()} Gets classifier options\\
{\bf getMaxIterations()} Get the value of m\_MaxIterations\\
{\bf getNumHiddenVariables()} \\
{\bf getOptions()} Gets the current settings of the Classifier.\\
{\bf getRandomData()} Get the value of m\_RandomData\\
{\bf getSupervised()} Get the value of m\_Supervised\\
{\bf globalInfo()} Global information about the class\\
{\bf initializeClassifier(Instances)} Makes copies of the training data which can be mutated, and initialise the array of Classifier objects\\
{\bf listOptions()} Returns an enumeration describing the available options.\\
{\bf main(String\lbrack \rbrack )} Main method for testing this class.\\
{\bf maxIterationsTipText()} Tip text to be displayed in the GUI for this property\\
{\bf next()} Retrains each of the classifiers, then attempts to impute missing data in a copy of the training data.\\
{\bf numHiddenVariablesTipText()} \\
{\bf randomDataTipText()} Tip text to be displayed in the GUI for this property\\
{\bf setClassifierOptions(String\lbrack \rbrack )} Sets classifier options\\
{\bf setMaxIterations(int)} Set the value of m\_MaxIterations.\\
{\bf setNumHiddenVariables(int)} \\
{\bf setOptions(String\lbrack \rbrack )} Parses a given list of options.\\
{\bf setRandomData(boolean)} Set the value of m\_RandomData\\
{\bf setSupervised(boolean)} Set the value of m\_Supervised\\
{\bf supervisedTipText()} Tip text to be displayed in the GUI for this property\\
\end{verse}
}
\subsection{Constructors}{
\vskip -2em
\begin{itemize}
\item{ 
\index{ProjectClassifier()}
{\bf  ProjectClassifier}\\
\begin{lstlisting}[frame=none]
public ProjectClassifier()\end{lstlisting} %end signature
\begin{itemize}
\item{
{\bf  Description}

Constructor
}
\end{itemize}
}%end item
\end{itemize}
}
\subsection{Methods}{
\vskip -2em
\begin{itemize}
\item{ 
\index{buildClassifier(Instances)}
{\bf  buildClassifier}\\
\begin{lstlisting}[frame=none]
public void buildClassifier(weka.core.Instances instances) throws java.lang.Exception\end{lstlisting} %end signature
\begin{itemize}
\item{
{\bf  Description}

Builds a set of classifiers based on the training data. These are iteratively trained on copies of the data.
}
\item{
{\bf  Parameters}
  \begin{itemize}
   \item{
\texttt{instances} -- the Instances object which comprises the training data}
  \end{itemize}
}%end item
\item{{\bf  Throws}
  \begin{itemize}
   \item{\vskip -.6ex \texttt{java.lang.Exception} -- exception thrown is raised to a Weka error handler}
  \end{itemize}
}%end item
\end{itemize}
}%end item
\item{ 
\index{classifierOptionsTipText()}
{\bf  classifierOptionsTipText}\\
\begin{lstlisting}[frame=none]
public java.lang.String classifierOptionsTipText()\end{lstlisting} %end signature
\begin{itemize}
\item{
{\bf  Description}

Tip text to be displayed in the GUI for this property
}
\item{{\bf  Returns} -- 
tip text to be displayed in the GUI 
}%end item
\end{itemize}
}%end item
\item{ 
\index{classifyInstance(Instance)}
{\bf  classifyInstance}\\
\begin{lstlisting}[frame=none]
public double classifyInstance(weka.core.Instance instance) throws java.lang.Exception\end{lstlisting} %end signature
\begin{itemize}
\item{
{\bf  Description}

Classifies an instance.
}
\item{
{\bf  Parameters}
  \begin{itemize}
   \item{
\texttt{instance} -- the instance to classify}
  \end{itemize}
}%end item
\item{{\bf  Returns} -- 
the classification for the instance 
}%end item
\item{{\bf  Throws}
  \begin{itemize}
   \item{\vskip -.6ex \texttt{java.lang.Exception} -- if instance can't be classified successfully}
  \end{itemize}
}%end item
\end{itemize}
}%end item
\item{ 
\index{defaultClassifierString()}
{\bf  defaultClassifierString}\\
\begin{lstlisting}[frame=none]
protected java.lang.String defaultClassifierString()\end{lstlisting} %end signature
\begin{itemize}
\item{
{\bf  Description}

String describing default classifier.
}
\end{itemize}
}%end item
\item{ 
\index{distributionForInstance(Instance)}
{\bf  distributionForInstance}\\
\begin{lstlisting}[frame=none]
public double[] distributionForInstance(weka.core.Instance instance) throws java.lang.Exception\end{lstlisting} %end signature
\begin{itemize}
\item{
{\bf  Description}

Returns class probabilities for an instance.
}
\item{
{\bf  Parameters}
  \begin{itemize}
   \item{
\texttt{instance} -- the instance to calculate the class probabilities for}
  \end{itemize}
}%end item
\item{{\bf  Returns} -- 
the class probabilities 
}%end item
\item{{\bf  Throws}
  \begin{itemize}
   \item{\vskip -.6ex \texttt{java.lang.Exception} -- if distribution can't be computed successfully}
  \end{itemize}
}%end item
\end{itemize}
}%end item
\item{ 
\index{done()}
{\bf  done}\\
\begin{lstlisting}[frame=none]
public void done() throws java.lang.Exception\end{lstlisting} %end signature
\begin{itemize}
\item{
{\bf  Description}

Method called when iteration has terminated. Imputes class values if m\_Supervised is set.
}
\end{itemize}
}%end item
\item{ 
\index{getCapabilities()}
{\bf  getCapabilities}\\
\begin{lstlisting}[frame=none]
public weka.core.Capabilities getCapabilities()\end{lstlisting} %end signature
\begin{itemize}
\item{
{\bf  Description}

Returns default capabilities of the classifier.
}
\item{{\bf  Returns} -- 
the capabilities of this classifier 
}%end item
\end{itemize}
}%end item
\item{ 
\index{getClassifierOptions()}
{\bf  getClassifierOptions}\\
\begin{lstlisting}[frame=none]
public java.lang.String[] getClassifierOptions()\end{lstlisting} %end signature
\begin{itemize}
\item{
{\bf  Description}

Gets classifier options
}
\item{{\bf  Returns} -- 
array of String objects to be passed to each classifier 
}%end item
\end{itemize}
}%end item
\item{ 
\index{getMaxIterations()}
{\bf  getMaxIterations}\\
\begin{lstlisting}[frame=none]
public int getMaxIterations()\end{lstlisting} %end signature
\begin{itemize}
\item{
{\bf  Description}

Get the value of m\_MaxIterations
}
\item{{\bf  Returns} -- 
value of m\_MaxIterations 
}%end item
\end{itemize}
}%end item
\item{ 
\index{getNumHiddenVariables()}
{\bf  getNumHiddenVariables}\\
\begin{lstlisting}[frame=none]
public int getNumHiddenVariables()\end{lstlisting} %end signature
}%end item
\item{ 
\index{getOptions()}
{\bf  getOptions}\\
\begin{lstlisting}[frame=none]
public java.lang.String[] getOptions()\end{lstlisting} %end signature
\begin{itemize}
\item{
{\bf  Description}

Gets the current settings of the Classifier.
}
\item{{\bf  Returns} -- 
an array of strings suitable for passing to setOptions 
}%end item
\end{itemize}
}%end item
\item{ 
\index{getRandomData()}
{\bf  getRandomData}\\
\begin{lstlisting}[frame=none]
public boolean getRandomData()\end{lstlisting} %end signature
\begin{itemize}
\item{
{\bf  Description}

Get the value of m\_RandomData
}
\item{{\bf  Returns} -- 
value of m\_RandomData 
}%end item
\end{itemize}
}%end item
\item{ 
\index{getSupervised()}
{\bf  getSupervised}\\
\begin{lstlisting}[frame=none]
public boolean getSupervised()\end{lstlisting} %end signature
\begin{itemize}
\item{
{\bf  Description}

Get the value of m\_Supervised
}
\item{{\bf  Returns} -- 
value of m\_Supervised 
}%end item
\end{itemize}
}%end item
\item{ 
\index{globalInfo()}
{\bf  globalInfo}\\
\begin{lstlisting}[frame=none]
public java.lang.String globalInfo()\end{lstlisting} %end signature
\begin{itemize}
\item{
{\bf  Description}

Global information about the class
}
\item{{\bf  Returns} -- 
information about the classifier which is displayed in the CLI/GUI 
}%end item
\end{itemize}
}%end item
\item{ 
\index{initializeClassifier(Instances)}
{\bf  initializeClassifier}\\
\begin{lstlisting}[frame=none]
public void initializeClassifier(weka.core.Instances instances) throws java.lang.Exception\end{lstlisting} %end signature
\begin{itemize}
\item{
{\bf  Description}

Makes copies of the training data which can be mutated, and initialise the array of Classifier objects
}
\item{
{\bf  Parameters}
  \begin{itemize}
   \item{
\texttt{instances} -- the training data}
  \end{itemize}
}%end item
\end{itemize}
}%end item
\item{ 
\index{listOptions()}
{\bf  listOptions}\\
\begin{lstlisting}[frame=none]
public java.util.Enumeration listOptions()\end{lstlisting} %end signature
\begin{itemize}
\item{
{\bf  Description}

Returns an enumeration describing the available options.
}
\item{{\bf  Returns} -- 
an enumeration of all the available options. 
}%end item
\end{itemize}
}%end item
\item{ 
\index{main(String\lbrack \rbrack )}
{\bf  main}\\
\begin{lstlisting}[frame=none]
public static void main(java.lang.String[] args)\end{lstlisting} %end signature
\begin{itemize}
\item{
{\bf  Description}

Main method for testing this class.
}
\item{
{\bf  Parameters}
  \begin{itemize}
   \item{
\texttt{args} -- the options}
  \end{itemize}
}%end item
\end{itemize}
}%end item
\item{ 
\index{maxIterationsTipText()}
{\bf  maxIterationsTipText}\\
\begin{lstlisting}[frame=none]
public java.lang.String maxIterationsTipText()\end{lstlisting} %end signature
\begin{itemize}
\item{
{\bf  Description}

Tip text to be displayed in the GUI for this property
}
\item{{\bf  Returns} -- 
tip text to be displayed in the GUI 
}%end item
\end{itemize}
}%end item
\item{ 
\index{next()}
{\bf  next}\\
\begin{lstlisting}[frame=none]
public boolean next() throws java.lang.Exception\end{lstlisting} %end signature
\begin{itemize}
\item{
{\bf  Description}

Retrains each of the classifiers, then attempts to impute missing data in a copy of the training data. Does not iterate again if the results of current iteration match the results of the previous iteration, or the max number of iterations has been reached.
}
\item{{\bf  Returns} -- 
true if another iteration should be performed, otherwise false. 
}%end item
\end{itemize}
}%end item
\item{ 
\index{numHiddenVariablesTipText()}
{\bf  numHiddenVariablesTipText}\\
\begin{lstlisting}[frame=none]
public java.lang.String numHiddenVariablesTipText()\end{lstlisting} %end signature
}%end item
\item{ 
\index{randomDataTipText()}
{\bf  randomDataTipText}\\
\begin{lstlisting}[frame=none]
public java.lang.String randomDataTipText()\end{lstlisting} %end signature
\begin{itemize}
\item{
{\bf  Description}

Tip text to be displayed in the GUI for this property
}
\item{{\bf  Returns} -- 
tip text to be displayed in the GUI 
}%end item
\end{itemize}
}%end item
\item{ 
\index{setClassifierOptions(String\lbrack \rbrack )}
{\bf  setClassifierOptions}\\
\begin{lstlisting}[frame=none]
public void setClassifierOptions(java.lang.String[] classifierOptions)\end{lstlisting} %end signature
\begin{itemize}
\item{
{\bf  Description}

Sets classifier options
}
\item{
{\bf  Parameters}
  \begin{itemize}
   \item{
\texttt{classifierOptions} -- array of String objects to be passed to each classifier}
  \end{itemize}
}%end item
\end{itemize}
}%end item
\item{ 
\index{setMaxIterations(int)}
{\bf  setMaxIterations}\\
\begin{lstlisting}[frame=none]
public void setMaxIterations(int maxIterations)\end{lstlisting} %end signature
\begin{itemize}
\item{
{\bf  Description}

Set the value of m\_MaxIterations. Defaults to Integer.MAX\_VALUE if value less than 0 is supplied.
}
\item{
{\bf  Parameters}
  \begin{itemize}
   \item{
\texttt{maxIterations} -- new value of m\_MaxIterations}
  \end{itemize}
}%end item
\end{itemize}
}%end item
\item{ 
\index{setNumHiddenVariables(int)}
{\bf  setNumHiddenVariables}\\
\begin{lstlisting}[frame=none]
public void setNumHiddenVariables(int numHiddenVariables)\end{lstlisting} %end signature
}%end item
\item{ 
\index{setOptions(String\lbrack \rbrack )}
{\bf  setOptions}\\
\begin{lstlisting}[frame=none]
public void setOptions(java.lang.String[] options) throws java.lang.Exception\end{lstlisting} %end signature
\begin{itemize}
\item{
{\bf  Description}

Parses a given list of options. Valid options are:

-W classifier\mbox{}\newline Full path to the target classifier to use, e.g. weka.classifiers.trees.J48

-S\mbox{}\newline Defines whether or not the classifier will impute a value for the class attribute as it trains.

-R\mbox{}\newline If set, the classifiers will be trained with any missing arguments filled in by random data. The classifier will then only iterate once.

-M integer\mbox{}\newline Sets the maximum number of times that a particular classifier will iterate before determining that it is trained.

Options after -- are passed to the currently selected classifier.
}
\item{
{\bf  Parameters}
  \begin{itemize}
   \item{
\texttt{options} -- The list of options as an array of Strings}
  \end{itemize}
}%end item
\item{{\bf  Throws}
  \begin{itemize}
   \item{\vskip -.6ex \texttt{java.lang.Exception} -- if an option is not supported}
  \end{itemize}
}%end item
\end{itemize}
}%end item
\item{ 
\index{setRandomData(boolean)}
{\bf  setRandomData}\\
\begin{lstlisting}[frame=none]
public void setRandomData(boolean randomData)\end{lstlisting} %end signature
\begin{itemize}
\item{
{\bf  Description}

Set the value of m\_RandomData
}
\item{
{\bf  Parameters}
  \begin{itemize}
   \item{
\texttt{randomData} -- new value of m\_RandomData}
  \end{itemize}
}%end item
\end{itemize}
}%end item
\item{ 
\index{setSupervised(boolean)}
{\bf  setSupervised}\\
\begin{lstlisting}[frame=none]
public void setSupervised(boolean supervised)\end{lstlisting} %end signature
\begin{itemize}
\item{
{\bf  Description}

Set the value of m\_Supervised
}
\item{
{\bf  Parameters}
  \begin{itemize}
   \item{
\texttt{supervised} -- the new value of m\_Supervised}
  \end{itemize}
}%end item
\end{itemize}
}%end item
\item{ 
\index{supervisedTipText()}
{\bf  supervisedTipText}\\
\begin{lstlisting}[frame=none]
public java.lang.String supervisedTipText()\end{lstlisting} %end signature
\begin{itemize}
\item{
{\bf  Description}

Tip text to be displayed in the GUI for this property
}
\item{{\bf  Returns} -- 
tip text to be displayed in the GUI 
}%end item
\end{itemize}
}%end item
\end{itemize}
}
\subsection{Members inherited from class SingleClassifierEnhancer }{
\texttt{weka.classifiers.SingleClassifierEnhancer} {\small 
\refdefined{weka.classifiers.SingleClassifierEnhancer}}
{\small 

\vskip -2em
\begin{itemize}
\item{\vskip -1.5ex 
\texttt{public String {\bf  classifierTipText}()
}%end signature
}%end item
\item{\vskip -1.5ex 
\texttt{protected String {\bf  defaultClassifierOptions}()
}%end signature
}%end item
\item{\vskip -1.5ex 
\texttt{protected String {\bf  defaultClassifierString}()
}%end signature
}%end item
\item{\vskip -1.5ex 
\texttt{public Capabilities {\bf  getCapabilities}()
}%end signature
}%end item
\item{\vskip -1.5ex 
\texttt{public Classifier {\bf  getClassifier}()
}%end signature
}%end item
\item{\vskip -1.5ex 
\texttt{protected String {\bf  getClassifierSpec}()
}%end signature
}%end item
\item{\vskip -1.5ex 
\texttt{public String {\bf  getOptions}()
}%end signature
}%end item
\item{\vskip -1.5ex 
\texttt{public Enumeration {\bf  listOptions}()
}%end signature
}%end item
\item{\vskip -1.5ex 
\texttt{protected {\bf  m\_Classifier}}%end signature
}%end item
\item{\vskip -1.5ex 
\texttt{public void {\bf  postExecution}() throws java.lang.Exception
}%end signature
}%end item
\item{\vskip -1.5ex 
\texttt{public void {\bf  preExecution}() throws java.lang.Exception
}%end signature
}%end item
\item{\vskip -1.5ex 
\texttt{public void {\bf  setClassifier}(\texttt{Classifier} {\bf  arg0})
}%end signature
}%end item
\item{\vskip -1.5ex 
\texttt{public void {\bf  setOptions}(\texttt{java.lang.String\lbrack \rbrack } {\bf  arg0}) throws java.lang.Exception
}%end signature
}%end item
\end{itemize}
}
\subsection{Members inherited from class AbstractClassifier }{
\texttt{weka.classifiers.AbstractClassifier} {\small 
\refdefined{weka.classifiers.AbstractClassifier}}
{\small 

\vskip -2em
\begin{itemize}
\item{\vskip -1.5ex 
\texttt{public static {\bf  BATCH\_SIZE\_DEFAULT}}%end signature
}%end item
\item{\vskip -1.5ex 
\texttt{public String {\bf  batchSizeTipText}()
}%end signature
}%end item
\item{\vskip -1.5ex 
\texttt{public double {\bf  classifyInstance}(\texttt{weka.core.Instance} {\bf  arg0}) throws java.lang.Exception
}%end signature
}%end item
\item{\vskip -1.5ex 
\texttt{public String {\bf  debugTipText}()
}%end signature
}%end item
\item{\vskip -1.5ex 
\texttt{public double {\bf  distributionForInstance}(\texttt{weka.core.Instance} {\bf  arg0}) throws java.lang.Exception
}%end signature
}%end item
\item{\vskip -1.5ex 
\texttt{public double {\bf  distributionsForInstances}(\texttt{weka.core.Instances} {\bf  arg0}) throws java.lang.Exception
}%end signature
}%end item
\item{\vskip -1.5ex 
\texttt{public String {\bf  doNotCheckCapabilitiesTipText}()
}%end signature
}%end item
\item{\vskip -1.5ex 
\texttt{public static Classifier {\bf  forName}(\texttt{java.lang.String} {\bf  arg0},
\texttt{java.lang.String\lbrack \rbrack } {\bf  arg1}) throws java.lang.Exception
}%end signature
}%end item
\item{\vskip -1.5ex 
\texttt{public String {\bf  getBatchSize}()
}%end signature
}%end item
\item{\vskip -1.5ex 
\texttt{public Capabilities {\bf  getCapabilities}()
}%end signature
}%end item
\item{\vskip -1.5ex 
\texttt{public boolean {\bf  getDebug}()
}%end signature
}%end item
\item{\vskip -1.5ex 
\texttt{public boolean {\bf  getDoNotCheckCapabilities}()
}%end signature
}%end item
\item{\vskip -1.5ex 
\texttt{public int {\bf  getNumDecimalPlaces}()
}%end signature
}%end item
\item{\vskip -1.5ex 
\texttt{public String {\bf  getOptions}()
}%end signature
}%end item
\item{\vskip -1.5ex 
\texttt{public String {\bf  getRevision}()
}%end signature
}%end item
\item{\vskip -1.5ex 
\texttt{public boolean {\bf  implementsMoreEfficientBatchPrediction}()
}%end signature
}%end item
\item{\vskip -1.5ex 
\texttt{public Enumeration {\bf  listOptions}()
}%end signature
}%end item
\item{\vskip -1.5ex 
\texttt{protected {\bf  m\_BatchSize}}%end signature
}%end item
\item{\vskip -1.5ex 
\texttt{protected {\bf  m\_Debug}}%end signature
}%end item
\item{\vskip -1.5ex 
\texttt{protected {\bf  m\_DoNotCheckCapabilities}}%end signature
}%end item
\item{\vskip -1.5ex 
\texttt{protected {\bf  m\_numDecimalPlaces}}%end signature
}%end item
\item{\vskip -1.5ex 
\texttt{public static Classifier {\bf  makeCopies}(\texttt{Classifier} {\bf  arg0},
\texttt{int} {\bf  arg1}) throws java.lang.Exception
}%end signature
}%end item
\item{\vskip -1.5ex 
\texttt{public static Classifier {\bf  makeCopy}(\texttt{Classifier} {\bf  arg0}) throws java.lang.Exception
}%end signature
}%end item
\item{\vskip -1.5ex 
\texttt{public static {\bf  NUM\_DECIMAL\_PLACES\_DEFAULT}}%end signature
}%end item
\item{\vskip -1.5ex 
\texttt{public String {\bf  numDecimalPlacesTipText}()
}%end signature
}%end item
\item{\vskip -1.5ex 
\texttt{public void {\bf  postExecution}() throws java.lang.Exception
}%end signature
}%end item
\item{\vskip -1.5ex 
\texttt{public void {\bf  preExecution}() throws java.lang.Exception
}%end signature
}%end item
\item{\vskip -1.5ex 
\texttt{public void {\bf  run}(\texttt{java.lang.Object} {\bf  arg0},
\texttt{java.lang.String\lbrack \rbrack } {\bf  arg1}) throws java.lang.Exception
}%end signature
}%end item
\item{\vskip -1.5ex 
\texttt{public static void {\bf  runClassifier}(\texttt{Classifier} {\bf  arg0},
\texttt{java.lang.String\lbrack \rbrack } {\bf  arg1})
}%end signature
}%end item
\item{\vskip -1.5ex 
\texttt{public void {\bf  setBatchSize}(\texttt{java.lang.String} {\bf  arg0})
}%end signature
}%end item
\item{\vskip -1.5ex 
\texttt{public void {\bf  setDebug}(\texttt{boolean} {\bf  arg0})
}%end signature
}%end item
\item{\vskip -1.5ex 
\texttt{public void {\bf  setDoNotCheckCapabilities}(\texttt{boolean} {\bf  arg0})
}%end signature
}%end item
\item{\vskip -1.5ex 
\texttt{public void {\bf  setNumDecimalPlaces}(\texttt{int} {\bf  arg0})
}%end signature
}%end item
\item{\vskip -1.5ex 
\texttt{public void {\bf  setOptions}(\texttt{java.lang.String\lbrack \rbrack } {\bf  arg0}) throws java.lang.Exception
}%end signature
}%end item
\end{itemize}
}
}
}
\printindex
\end{document}
