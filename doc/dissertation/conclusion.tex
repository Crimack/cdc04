% conclusion.tex
\newpage
\chapter{Conclusion}\label{conclusion}
This dissertation has implemented a novel method of training models for use in classifying nominal data sets, and has experimented with using this method together with a number of established classification techniques. Through these experiments it has been shown that the implemented method of iteratively imputing and retraining classifiers generally leads to a better result when measuring mean absolute error, but also causes lower classification accuracy and a worse root mean squared error value. Despite these broad trends, no concrete conclusion has been reached and there is potential for further analysis in another body of work. This could include experimentation with other classifiers or experimentation with the maximum iteration limit.

As a project, all objectives have been fulfilled. A functional, distributable Weka plugin has been developed which integrates into the system through its package manager and passes all unit tests imposed by the system. It is usable from both the CLI and GUI, and allows for easy configuration of its parameters and the parameters of its internal classifiers. It has also been used for experiments as intended, and some minor insights have been gleaned from these.